\chapter*{序言}

在波涛汹涌的洋面上,发光的浮游生物在黑暗的水域里飘动。它们的体内有发光物质,一旦感知到周围环境的变化时,这些发光物质就会被触发、闪烁,
仿佛黑暗中摇曳的点点星辰;它们相互呼应,把孤立的星辰连成一片星海。数学最为遥远的源头,可能就像这些浮游生物一样,
源自于生命对于自然界律动和模式的感知、反映与应答。

伴随着脑容量不断扩大,这种生命和环境的互动模式随着时间的推移和生物演化逐渐复杂化。从几亿年前的昆虫开始,动物已经开始展示出令人惊讶的数学能力。
此时承载数学能力的媒介是生物个体本身。数学的演化要通过生物的演化来实现,这种演化的速度是极其缓慢的。

更新世后期,伴随着智人语言能力的发展,这种原始的感知与能力借助符号的力量,终于脱离了身体的限制,开始不断加速迭代。
在这一阶段,承载数学能力的媒介是符号,数学开始成为一种跨越代际的文化现象。在几十万年的时间尺度内,数学的文化层层累进,发展成一棵参天大树。
它深刻地影响着人类的行为,塑造着人类的精神生活、制度规范和物质世界。
