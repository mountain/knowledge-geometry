\documentclass{article}

\usepackage{arxiv}

\usepackage[utf8]{inputenc} % allow utf-8 input
\usepackage[T1]{fontenc}    % use 8-bit T1 fonts
\usepackage{hyperref}       % hyperlinks
\usepackage{url}            % simple URL typesetting
\usepackage{booktabs}       % professional-quality tables
\usepackage{amsfonts}       % blackboard math symbols
\usepackage{nicefrac}       % compact symbols for 1/2, etc.
\usepackage{microtype}      % microtypography

\usepackage{graphicx}
\usepackage{amsmath}

\newtheorem{definition}{Definition}
\newtheorem{lemma}{Lemma}
\newtheorem{proposition}{Proposition}
\newtheorem{program}{Program}
\newtheorem{convention}{Convention}

\title{Geometry of computation: An emembeding from addition-multiplication tree into hyperbolic surface}

%\date{September 9, 1985}	% Here you can change the date presented in the paper title
%\date{} 					% Or removing it

\author{
  Mingli~Yuan \\
  AI Lab \\
  ColorfulClouds Tech.\\
  Beijing, 100083 \\
  \texttt{mingli.yuan@caiyunapp.com} \\
  %% examples of more authors
  %% \AND
  %% Coauthor \\
  %% Affiliation \\
  %% Address \\
  %% \texttt{email} \\
  %% \And
  %% Coauthor \\
  %% Affiliation \\
  %% Address \\
  %% \texttt{email} \\
  %% \And
  %% Coauthor \\
  %% Affiliation \\
  %% Address \\
  %% \texttt{email} \\
}

% Uncomment to remove the date
%\date{}

% Uncomment to override  the `A preprint' in the header
%\renewcommand{\headeright}{Technical Report}
%\renewcommand{\undertitle}{Technical Report}

\begin{document}
\maketitle

\begin{abstract}
    We introduce an new mathematical object Suan system $\mathfrak{S}$ which is a geometry interpretion of computation
    over addition and multiplication on real numbers. After we defined a local polar coordinate system on
    $\mathfrak{S}$, a flow equation can be conducted, and we show surreal number can be embedded properly into
    $\mathfrak{S}$ to support the perspective of geometry of computation.
\end{abstract}

\keywords{addition-multiplication tree \and hyperbolic surface \and surreal number \and geometry of computation}

\section{An embeding from addition-multiplication tree into hyperbolic surface}\label{sec:aefamtihs}

\section{Geometry interpretion of computation}\label{sec:gioc}

\section{Local polar coordinate system and flow equation}\label{sec:lpcsafe}

\section{An emembeding of surreal number}\label{sec:aeosn}


\bibliographystyle{unsrt}
\bibliography{references}

\end{document}
