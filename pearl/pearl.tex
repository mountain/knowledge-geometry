\documentclass{article}

\usepackage{arxiv}

\usepackage[utf8]{inputenc} % allow utf-8 input
\usepackage[T1]{fontenc}    % use 8-bit T1 fonts
\usepackage{hyperref}       % hyperlinks
\usepackage{url}            % simple URL typesetting
\usepackage{booktabs}       % professional-quality tables
\usepackage{amsfonts}       % blackboard math symbols
\usepackage{nicefrac}       % compact symbols for 1/2, etc.
\usepackage{microtype}      % microtypography

\usepackage{graphicx}
\usepackage{amsmath}

\newtheorem{definition}{Definition}
\newtheorem{lemma}{Lemma}
\newtheorem{proposition}{Proposition}
\newtheorem{program}{Program}
\newtheorem{convention}{Convention}

\title{Toward geometry of computation: An embedding from addition-multiplication tree into hyperbolic surface}

%\date{September 9, 1985}	% Here you can change the date presented in the paper title
%\date{} 					% Or removing it

\author{
  Mingli~Yuan \\
  AI Lab \\
  ColorfulClouds Tech.\\
  Beijing, 100083 \\
  \texttt{mingli.yuan@caiyunapp.com} \\
  %% examples of more authors
  %% \AND
  %% Coauthor \\
  %% Affiliation \\
  %% Address \\
  %% \texttt{email} \\
  %% \And
  %% Coauthor \\
  %% Affiliation \\
  %% Address \\
  %% \texttt{email} \\
  %% \And
  %% Coauthor \\
  %% Affiliation \\
  %% Address \\
  %% \texttt{email} \\
}

% Uncomment to remove the date
%\date{}

% Uncomment to override  the `A preprint' in the header
%\renewcommand{\headeright}{Technical Report}
%\renewcommand{\undertitle}{Technical Report}

\begin{document}
\maketitle

\begin{abstract}
    We introduce an new mathematical system Suan $\mathfrak{S}$ which is an embedding of addition-multiplication
    computation into hyperbolic surface. After a local polar coordinate on $\mathfrak{S}$ is defined, a flow
    equation can be conducted, and it shows surreal number can be embedded properly into $\mathfrak{S}$.
\end{abstract}

\keywords{addition-multiplication tree \and hyperbolic surface \and surreal number \and geometry of computation}

\section{Computation as an addition-multiplication tree}\label{sec:caaamt}

Traditionally, computation was engaged by apply a serial of addition and multiplication on numbers. Any
addition-multiplication expression can be formed into a binary tree, while pure numbers are leaves and addition or
multiplication operators are at branches.


\section{An embedding from addition-multiplication tree into hyperbolic surface}\label{sec:aefamtihs}

\section{Local polar coordinate and flow equation}\label{sec:lpcsafe}

Given any point $x_0$ on $\mathfrak{S}$, an addtional axis $A$ pass $x_0$, then for any nearby point $x_1$, two
horocycle $B$ and $\bar{B}$ exists to connect with $x_0$ and $x_1$, so we can consider a local polar coordinate
$\mathfrak{C}_{x_0}$, in which $x_1$ decided by:
\begin{itemize}
    \item the angle $\theta$ between $A$ and $B$
    \item the length $\rho$ of arc between $x_0$ and $x_1$ along $B$
\end{itemize}

$\mathfrak{C}_{x_0}$ has a counterpart $\bar{\mathfrak{C}_{x_0}}$, in which $x_1$ decided by:
\begin{itemize}
    \item the angle $\bar{\theta}$ between $A$ and $\bar{B}$
    \item the length $\bar{\rho}$ of arc between $x_0$ and $x_1$ along $\bar{B}$
\end{itemize}

\section{An embedding of surreal number}\label{sec:aeosn}

\section{Geometry of computation}\label{sec:gioc}


\bibliographystyle{unsrt}
\bibliography{references}

\end{document}
