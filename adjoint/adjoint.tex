\documentclass{article}

\usepackage{arxiv}

\usepackage[utf8]{inputenc} % allow utf-8 input
\usepackage[T1]{fontenc}    % use 8-bit T1 fonts
\usepackage{hyperref}       % hyperlinks
\usepackage{url}            % simple URL typesetting
\usepackage{booktabs}       % professional-quality tables
\usepackage{amsfonts}       % blackboard math symbols
\usepackage{nicefrac}       % compact symbols for 1/2, etc.
\usepackage{microtype}      % microtypography

\usepackage{graphicx}
\usepackage{amsmath}

\newtheorem{definition}{Definition}
\newtheorem{lemma}{Lemma}
\newtheorem{proposition}{Proposition}
\newtheorem{program}{Program}
\newtheorem{convention}{Convention}

\title{Notes on geometrical interpretation of adjoint equation}

%\date{September 9, 1985}	% Here you can change the date presented in the paper title
%\date{} 					% Or removing it

\author{
  Mingli~Yuan \\
  AI Lab \\
  Beijing ColorfulClouds Tech.\\
  Beijing, 100083 \\
  \texttt{mingli.yuan@caiyunapp.com} \\
  %% examples of more authors
  %% \AND
  %% Coauthor \\
  %% Affiliation \\
  %% Address \\
  %% \texttt{email} \\
  %% \And
  %% Coauthor \\
  %% Affiliation \\
  %% Address \\
  %% \texttt{email} \\
  %% \And
  %% Coauthor \\
  %% Affiliation \\
  %% Address \\
  %% \texttt{email} \\
}

% Uncomment to remove the date
%\date{}

% Uncomment to override  the `A preprint' in the header
%\renewcommand{\headeright}{Technical Report}
%\renewcommand{\undertitle}{Technical Report}

\begin{document}
\maketitle

\begin{abstract}
Traditionally, adjoint equation was introduced via different approches.
After recalling the geometrical interpretation of adjoint operators in linear cases, we reveal that adjoint equation shares the same interpretation.
Then a new dual-number-based definition of adjoint equation is given, and the equivalence of these different approches is proofed.
\end{abstract}

\keywords{adjoint \and adjoint equation \and geometrical interpretation \and dual number}

\section{Introduction}

Adjoint is a fundamental concept\cite{Daz1953MethodsOM}\cite{Marchuk1995}, and adjoint equation shows its importance in domains of optimal control\cite{Liberzon2012CalculusOV}, sensitivity analysis\cite{hall1983physical}, and data assimilation\cite{Errico1997WhatIA}. Also, recent contribution\cite{Chen2018NeuralOD} shows adjoint equation can play a role in the intersection area of deep learning and differential equations.

In linear cases, there is an apparent geometrical interpretation of adjoint operators, and in this paper, we want to reveal that adjoint equation shares the same interpretation.

\subsection{Definition and geometrical interpretation of adjoint operators in linear case}

Formally, we can define a pair of adjoint operators in a dual system.

\begin{definition}
\label{d0}
A dual system $ \langle X, Y \rangle $ is a bilinear mapping $ \langle , \rangle : X \times Y \to F $ where $X$, $Y$ are two vector space and $ F $ is a field.
\end{definition}

Geometrically, We can interpret that $ X $ is a bottom space which holds the points, $ Y $ is a frame space which holds the coordinate frames, and the value of $ \langle x, y \rangle $ is the coordinate of $x$ under a frame $y$.

\begin{definition}
\label{d1}
For two dual system $ \langle X_1, Y_1 \rangle $ and $ \langle X_2, Y_2 \rangle $ , each of the two operator $ A : X_1 \to X_2$ and $ B : Y_2 \to Y_1 $ is an adjoint of the other counterpart,
if and only if, the equation
\begin{equation}
  \label{eq:langrangeidentity}
  \langle A \phi, \psi \rangle = \langle \phi, B \psi \rangle
\end{equation}
holds for arbitrary $ \phi \in X_1 $ and $ \psi \in Y_2 $
\end{definition}

Under the above interpretation, a pair of adjoint operators are just active and passive transformations (or alibi and alias transformations)\cite{wiki:aptrans}.

The left side of Equation \ref{eq:langrangeidentity} $ \langle A \phi, \psi \rangle $ is interpreted as an active transformation(operator) $ A $ act on a point $\phi$ in bottom space $ X_1 $ and hence have a coordinate $\langle A \phi, \psi \rangle$ after the transformations;

The right side of Equation \ref{eq:langrangeidentity} $ \langle \phi, B \psi \rangle $ is interpreted as a passive transformation(operator) $ B $ act on a frame $\psi$ in frame space $ Y_2 $ and hence have another coordinate $\langle \phi, B \psi \rangle$ after the transformations.

\begin{figure}[ht]
\label{fig:alias_and_alibi}
\centering
\includegraphics[width=3.5in]{../images/adjoint/alias_and_alibi.png}
\caption{A dual system and two pair of adjoint operators\cite{wiki:aatrans}}
\end{figure}

Then, the Equation \ref{eq:langrangeidentity} just says that the two coordinate are always the same, just as illustrated in Figure \ref{fig:alias_and_alibi}.

\subsection{Problem leading to adjoint equation}

A control system can be described with a state variable $\mathbf{x} \in \Omega$ and a control variable $\mathbf{u} \in \Upsilon$,
and the state variable $\mathbf{x}$ bounded to a constraint which governed by an evolutionary differential equation:

\begin{equation}
\begin{array}{ll}
\dot{\mathbf{x}} = f(\mathbf{x}, \mathbf{u}, t; \theta) & (\mathbf{x}, \mathbf{u}, t) \in \Omega \times \Upsilon \times [0, 1] \\
\mathbf{x} = g(\mathbf{x}, \mathbf{u}, t; \theta) & (\mathbf{x}, \mathbf{u}, t) \in \partial \Omega \times \Upsilon \times [0, 1] \\
\end{array}
\end{equation}

For the reason of simplification, we denote $\mathbf{\nu}_t$ as variable $\mathbf{\nu}$ at time $t$,
so $\mathbf{\nu}_0$ is the initial value and $\mathbf{\nu}_1$ is the terminal value.

Now we can define an optimization problem, find the control trajectory $\mathbf{u}:[0,1] \mapsto \Upsilon$ to minimize the cost:

\begin{equation}
J = \psi(\mathbf{x}_0) + \phi(\mathbf{x}_1) + \int\limits_{0}^{1} L(\mathbf{x}, \mathbf{u}, t) dt
\end{equation}

Or, rewrite in a compact form

\begin{equation}
\begin{array}{rcll}
\min &~& J(\mathbf{u}, \mathbf{x}, \mathbf{x}_0, \mathbf{x}_1) & \\
\mathrm{s.t.} &~& \dot{\mathbf{x}} = f(\mathbf{x}, \mathbf{u}, t; \theta) & (\mathbf{x}, \mathbf{u}, t) \in \Omega \times \Upsilon \times [0, 1] \\
&~& \mathbf{x} = g(\mathbf{x}, \mathbf{u}, t; \theta) & (\mathbf{x}, \mathbf{u}, t) \in \partial \Omega \times \Upsilon \times [0, 1] \\
\end{array}
\end{equation}

\subsection{Traditional approach to introduce adjoint equation}


\section{Geometrical interpretation of adjoint equation}

\subsection{Additional and multiplicational point of view}

\subsection{Forward and backward propagation of a disturbance}

\subsection{The geometrical interpretation}

\section{Reformulate with dual number}

\section{Reformulate disturbance as a space}

the orbit of a disturbance is a point

\section{Applications}

\bibliographystyle{unsrt}
\bibliography{references}

\end{document}
