\documentclass[a4paper,12pt]{article}
\usepackage{amsmath, amsthm}
\usepackage{datetime}
\usepackage{framed}
\usepackage{enumitem}
\usepackage{fancyref}
\usepackage{wrapfig}
\usepackage{pifont}
\usepackage{appendix}
\usepackage{caption}
\usepackage{xcolor}
\usepackage[stable]{footmisc}
\usepackage{multicol}
\usepackage{csquotes}

\usepackage{amsthm}
\usepackage{amssymb}
\usepackage{amsfonts}
\usepackage{amsmath}
\usepackage{mathtools}

\usepackage{tikz}
\usepackage{pgf}
\usepgflibrary{fpu}
\usepackage{qtree}
\usetikzlibrary{angles,fit,arrows,calc,math,intersections,through,backgrounds}
\usepackage{pgfplots}
\usepackage{tkz-euclide}

\usepackage{listings}
\lstset{
  basicstyle=\itshape,
  xleftmargin=3em,
  literate={->}{$\rightarrow$}{2}
           {α}{$\alpha$}{1}
           {δ}{$\delta$}{1}
}


\usepackage{csquotes}
\renewcommand{\mkbegdispquote}[2]{\itshape}

\newdateformat{nianyueri}{ \THEYEAR 年 \THEMONTH 月 \THEDAY 日 }

\usepackage{xstring}
\usepackage{catchfile}
\CatchFileDef{\HEAD}{../.git/refs/heads/master}{}
\newcommand{\gitrevision}{%
  \StrLeft{\HEAD}{7}%
}

\usepackage{data/quiver}
\usepackage{data/circledsteps}
\usepackage[top=1in,bottom=1in,left=1in,right=1in]{geometry} % 用于设置页面布局
\usepackage{xeCJK} % 用于使用本地字体
\usepackage[super, square, sort&compress]{natbib} % 处理参考文献
\usepackage{titlesec, titletoc} % 设置章节标题及页眉页脚
%\usepackage{xCJKnumb} % 中英文数字转换
\usepackage{amssymb}
\usepackage{amsmath} % 在公式中用\text{文本}输入中文
\usepackage{diagbox}
\usepackage{multirow} % 表格中使用多行
\usepackage{booktabs} % 表格中使用\toprule等命令
\usepackage{rotating} % 使用sidewaystable环境旋转表格
\usepackage{tabularx}
\usepackage{graphicx} % 处理图片
\usepackage{footnote} % 增强的脚注功能,可添加表格脚注
\usepackage{threeparttable} % 添加真正的表格脚注,示例见README
\usepackage{hyperref} % 添加pdf书签

\usepackage{tikz}
\usetikzlibrary{shapes,arrows,shadows}

% 字体设置
\setmainfont{Times New Roman}
\setsansfont[Scale=MatchLowercase,Mapping=tex-text]{PT Sans}
\setmonofont[Scale=MatchLowercase]{PT Mono}
\setCJKmainfont[ItalicFont={Kaiti SC}, BoldFont={Heiti SC}]{Songti SC}
\setCJKsansfont{Heiti SC}
\setCJKmonofont{Songti SC}
% \setCJKmainfont[BoldFont={FZXiaoBiaoSong-B05S}]{Songti SC}
% \setCJKfamilyfont{kai}[BoldFont=Heiti SC]{Kaiti SC}
% \setCJKfamilyfont{song}[BoldFont=Heiti SC]{Songti SC}
% \setCJKfamilyfont{hei}[BoldFont=Heiti SC]{Heiti SC}
% \setCJKfamilyfont{fsong}[BoldFont=Heiti SC]{Songti SC}
% \newcommand{\kai}[1]{{\CJKfamily{kai}#1}}
% \newcommand{\hei}[1]{{\CJKfamily{hei}#1}}
% \setromanfont[Mapping=tex-text]{TeXGyrePagella}
% \setsansfont[Scale=MatchLowercase,Mapping=tex-text]{TeXGyrePagella}
% \setmonofont[Scale=MatchLowercase]{Courier New}
%%设置常用中文字号,方便调用
\newcommand{\erhao}{\fontsize{22pt}{\baselineskip}\selectfont}
\newcommand{\xiaoerhao}{\fontsize{18pt}{\baselineskip}\selectfont}
\newcommand{\sanhao}{\fontsize{16pt}{\baselineskip}\selectfont}
\newcommand{\xiaosanhao}{\fontsize{15pt}{\baselineskip}\selectfont}
\newcommand{\sihao}{\fontsize{14pt}{\baselineskip}\selectfont}
\newcommand{\xiaosihao}{\fontsize{12pt}{\baselineskip}\selectfont}
\newcommand{\wuhao}{\fontsize{10.5pt}{\baselineskip}\selectfont}
\newcommand{\xiaowuhao}{\fontsize{9pt}{\baselineskip}\selectfont}
\newcommand{\liuhao}{\fontsize{7.5pt}{\baselineskip}\selectfont}

% 章节标题显示方式及页眉页脚设置
% \item xCJKnumb是自己额外安装的包
% \item titleformat命令定义标题的形式
% \item titlespacing定义标题距左、上、下的距离
\titleformat{\section}{\raggedright\large\bfseries}{\thesection}{1em}{}
\titleformat{\subsection}{\raggedright\normalsize\bfseries}{\thesubsection}{1em}{}
\titlespacing{\section}{0pt}{*0}{*2}
\titlespacing{\subsection}{0pt}{*0}{*1}
% 由于默认的2em缩进不够,所以我手动调整了,但是在windows下似乎2.2就差不多了,或者是article中没有这个问题
\setlength{\parindent}{2.2em}

% 设置表格标题前后间距
\setlength{\abovecaptionskip}{0pt}
\setlength{\belowcaptionskip}{0pt}


\renewcommand{\refname}{\bfseries{参~考~文~献}} %将Reference改为参考文献(用于 article)
% \renewcommand{\bibname}{参~考~文~献} %将bibiography改为参考文献(用于 book)
\renewcommand{\baselinestretch}{1.38} %设置行间距
\renewcommand{\figurename}{\small\ttfamily 图}
\renewcommand{\tablename}{\small\ttfamily 表}


\usepackage{stmaryrd}
\usepackage{mathtools}
\usepackage{wasysym}
\usepackage{textcomp}
\usepackage{subfiles}

\newtheorem{problem}{问题}
\numberwithin{problem}{section}
\newtheorem{definition}{定义}
\numberwithin{definition}{section}
\newtheorem{lemma}{引理}
\numberwithin{lemma}{section}
\newtheorem{proposition}{命题}
\numberwithin{proposition}{section}
\newtheorem{theorem}{定理}
\numberwithin{theorem}{section}
\newtheorem{grammar}{文法}
\numberwithin{grammar}{section}
\newtheorem{program}{程序}
\numberwithin{program}{section}
\newtheorem{convention}{约定}
\numberwithin{convention}{section}
\newtheorem{corollary}{推论}
\numberwithin{corollary}{section}
\renewcommand*{\proofname}{证明}

\xeCJKsetwidth{‘’“”}{1em}

\title{教育背景}
\date{\nianyueri\today}
\author{苑明理}

\begin{document}

\begingroup
\let\newpage\relax
\maketitle
\endgroup

\centerline{\rule{13cm}{0.4pt}}
\renewcommand{\contentsname}{\hfill\bfseries 目录\hfill}
\setcounter{tocdepth}{2}
\tableofcontents
\centerline{\rule{13cm}{0.4pt}}

\section{本科阶段}

求学的本科阶段,我爱泡图书馆,图书馆里的数学与物理部分都被我翻遍了。因为是 1990 年代初,当时国内翻译的教材和读物还有限,选择比较少,但我仍然注
重多本经典教材彼此参照,数学与物理的相互印证。

在低年级阶段,数学分析的学习,除了课堂跟随老师,我自己又以 R. Courant 和 F. John 的《微积分和数学分析引论》为底本,结合华罗庚的
《高等数学引论》,同时还与《费曼物理学讲义》彼此对照。我至今记得费曼讲的场论基础清晰而直白,可以快速进入。高等代数的学习,老师以北大的《高等代数》
为教材,我自己又选择许以超老师的《线性代数与矩阵论》作为参照。

进到高年级,因为学校里没有开设类似课程,我自修了陈维桓老师的《微分几何》、 J. R. Munkres 的《拓扑学》,读了 V. I. Arnold 的《常微分方程》。
同时,我还注重数学史的学习,M. Kline 的《古今数学思想》常常摆在案头。有些大家的普及性小册子也在我的阅读范围内,比如冯承天先生译的 H. Weyl 的
《对称》,齐民友先生译的 V. I. Arnold 的《初等拓扑的直观概念》,等等。

在高中的时候,我自己曾经提问过一个问题,光线射入到球面里会出来吗?大一的时候,我了解到了一定可以射出来,证明竟然用到抽屉原理或者鸽笼原理。但后来
发现,入射椭圆的光线,其中过焦点的那些竟然可以出不来,当时很震惊这样的结果。高年级的时候,从 Arnold 的书里,我知道了庞加莱回归定理。这个大一时
候的问题给我留下了很深的印象,于是在大四毕业时,我选择研究这个问题作为本科毕业论文。我边思考问题边学习,最后用了射影几何的齐次坐标,可以方便的表
示切线,花了一个月的时间推导出这个环形上的动力系统,它的相空间被过焦点的射线分为了三个不变子集;剩下了一个月时间,想去证明三个不变子集中有两个有
遍历性,但是太难,没有结果。

\section{硕士阶段}

硕士阶段的求学,我在张乃孝老师门下,学习软件的形式化方法,当时对程序设计语言非常感兴趣,而张老师有对领域语言的研究。因为想体会一下一个领域语
言的设计和实现究竟是怎么样的?我的硕士论文偏工程性质,设计和部分实现了一个工作流语言。匆匆三年一晃而过,很多功课只是开了个头,简单记录一下对日后
工作和思考有帮助的学习内容。

\subsection{代数语义初步}

张乃孝老师有形式语义的讨论班,我在其中介绍代数语义的初步内容。整个介绍中,我对多调代数(many-sorted algebra)如何通过项代数(term algebra)
获得初始语义(initial algebra)的手法印象特别深刻,这个手法在后面数理逻辑的学习里也遇到了。这些内容影响了我对算术和逻辑的理解。

\subsection{程序设计语言}

程序设计语言这门课是裘宗燕老师授课,通过这门课基本了解了程序设计语言里各种概念的历史源流和作用。
我很好的完成了大作业,给出了 Self 语言的一个详细介绍,是课程里唯一获得 90 分以上的同学。这些学习和我后面对
元对象协议(MOP,meta-object protocol)的探索有关系。

\subsection{可计算理论初步}

当时旁听了计算机科学系的计算理论的课程,基本理解了自动机和形式语言的初步框架。在参考其他文献时,发现了 Neil D. Jones 的
\textit{Computability and Complexity: From a Programming Perspective} 这本书。作者新鲜的讲法让我非常兴奋,很多定理如果
基于图灵机去证明,会绕来绕去,特别繁复,在这本书里用 While 语言的程序去证明就显得清晰易懂。

\subsection{数理逻辑初步}

因为程序精化(refinement)的课程里有 Hoare logic 的内容,我去自修了一部分数理逻辑的内容。本科时也有数理逻辑的初阶课程,所以原以为数理逻辑只是
一些推演规则的体系。但当时看了王浩先生的《数理逻辑通俗讲话》才理解到逻辑学起源于对形式系统和客观世界之间关系的探讨,其中对形式系统的有效性和完全
性的讨论是代数味道非常浓厚的部分。后来参考到了李未老师的《数理逻辑》,书中给出了谓词逻辑完全性的一种证明,是通过 Hintikka 集来处理的。
这个处理手法,和代数语义里通过项代数获得初始语义的方法,是如出一辙的。这背后有很浓厚的哲学意味,后来有哲学系同学告诉我,这和康德处理唯理论和经验
论之争的思路是类似的。

\section{日积月累}

离开学校进入工业界,缺少了系统学习的大块时间,但也能在实践中锻炼。除了知识之外,工作中也需要很多软技能,包括领域里的方向判断、
分析复杂问题的能力、团队的沟通与协作、领导和协调的能力,这些都是在日积月累里慢慢提升。

\subsection{工程和管理实践}

2005年左右,我开始接触函数式编程;2008年参与了 eBay 的 V4 系统里 Java 到 JavaScript 的编译器工作,负责了一部分库的设计。
2010年左右,我试验了元对象协议(MOP,Meta-Object protocol)的一些实现。

2011年开始转向机器学习,2012年负责组建了果壳网的机器学习团队,带领团队设计开发了网页推荐系统、全文检索系统和日志处理系统。

2014年和朋友创建了彩云科技,在深度学习基础上,开发了短临预报系统。先后负责,发布系统的设计开发、整体架构的规划、部分算法的研发,
也参与了一部分融资和商业发展的工作。

\subsection{有效过程与自然法则}

2012年我参与了北京本地的一个科学社群—集智俱乐部的研读活动,负责讲解了可计算性的内容。在过程中,我读到了 R. Rosen 的《有效过程与自然法则》
这篇文章。2021年,我利用了零零散散的时间,把这篇文章翻译成中文,请参见附件二。


\end{document}




