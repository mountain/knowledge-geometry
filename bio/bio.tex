\documentclass[a4paper,12pt]{article}
\usepackage{amsmath, amsthm}
\usepackage{datetime}
\usepackage{framed}
\usepackage{enumitem}
\usepackage{fancyref}
\usepackage{wrapfig}
\usepackage{pifont}
\usepackage{appendix}
\usepackage{caption}
\usepackage{xcolor}
\usepackage[stable]{footmisc}
\usepackage{multicol}
\usepackage{csquotes}

\usepackage{amsthm}
\usepackage{amssymb}
\usepackage{amsfonts}
\usepackage{amsmath}
\usepackage{mathtools}

\usepackage{tikz}
\usepackage{pgf}
\usepgflibrary{fpu}
\usepackage{qtree}
\usetikzlibrary{angles,fit,arrows,calc,math,intersections,through,backgrounds}
\usepackage{pgfplots}
\usepackage{tkz-euclide}

\usepackage{csquotes}
\renewcommand{\mkbegdispquote}[2]{\itshape}

\newdateformat{nianyueri}{ \THEYEAR 年 \THEMONTH 月 \THEDAY 日 }

\usepackage{xstring}
\usepackage{catchfile}
\CatchFileDef{\HEAD}{../.git/refs/heads/master}{}
\newcommand{\gitrevision}{%
  \StrLeft{\HEAD}{7}%
}

\usepackage{data/quiver}
\usepackage{data/circledsteps}
\usepackage[top=1in,bottom=1in,left=1in,right=1in]{geometry} % 用于设置页面布局
\usepackage{xeCJK} % 用于使用本地字体
\usepackage[super, square, sort&compress]{natbib} % 处理参考文献
\usepackage{titlesec, titletoc} % 设置章节标题及页眉页脚
%\usepackage{xCJKnumb} % 中英文数字转换
\usepackage{amssymb}
\usepackage{amsmath} % 在公式中用\text{文本}输入中文
\usepackage{diagbox}
\usepackage{multirow} % 表格中使用多行
\usepackage{booktabs} % 表格中使用\toprule等命令
\usepackage{rotating} % 使用sidewaystable环境旋转表格
\usepackage{tabularx}
\usepackage{graphicx} % 处理图片
\usepackage{footnote} % 增强的脚注功能,可添加表格脚注
\usepackage{threeparttable} % 添加真正的表格脚注,示例见README
\usepackage{hyperref} % 添加pdf书签

\usepackage{tikz}
\usetikzlibrary{shapes,arrows,shadows}

% 字体设置
\setmainfont{Times New Roman}
\setsansfont[Scale=MatchLowercase,Mapping=tex-text]{PT Sans}
\setmonofont[Scale=MatchLowercase]{PT Mono}
\setCJKmainfont[ItalicFont={Kaiti SC}, BoldFont={Heiti SC}]{Songti SC}
\setCJKsansfont{Heiti SC}
\setCJKmonofont{Songti SC}
% \setCJKmainfont[BoldFont={FZXiaoBiaoSong-B05S}]{Songti SC}
% \setCJKfamilyfont{kai}[BoldFont=Heiti SC]{Kaiti SC}
% \setCJKfamilyfont{song}[BoldFont=Heiti SC]{Songti SC}
% \setCJKfamilyfont{hei}[BoldFont=Heiti SC]{Heiti SC}
% \setCJKfamilyfont{fsong}[BoldFont=Heiti SC]{Songti SC}
% \newcommand{\kai}[1]{{\CJKfamily{kai}#1}}
% \newcommand{\hei}[1]{{\CJKfamily{hei}#1}}
% \setromanfont[Mapping=tex-text]{TeXGyrePagella}
% \setsansfont[Scale=MatchLowercase,Mapping=tex-text]{TeXGyrePagella}
% \setmonofont[Scale=MatchLowercase]{Courier New}
%%设置常用中文字号,方便调用
\newcommand{\erhao}{\fontsize{22pt}{\baselineskip}\selectfont}
\newcommand{\xiaoerhao}{\fontsize{18pt}{\baselineskip}\selectfont}
\newcommand{\sanhao}{\fontsize{16pt}{\baselineskip}\selectfont}
\newcommand{\xiaosanhao}{\fontsize{15pt}{\baselineskip}\selectfont}
\newcommand{\sihao}{\fontsize{14pt}{\baselineskip}\selectfont}
\newcommand{\xiaosihao}{\fontsize{12pt}{\baselineskip}\selectfont}
\newcommand{\wuhao}{\fontsize{10.5pt}{\baselineskip}\selectfont}
\newcommand{\xiaowuhao}{\fontsize{9pt}{\baselineskip}\selectfont}
\newcommand{\liuhao}{\fontsize{7.5pt}{\baselineskip}\selectfont}

% 章节标题显示方式及页眉页脚设置
% \item xCJKnumb是自己额外安装的包
% \item titleformat命令定义标题的形式
% \item titlespacing定义标题距左、上、下的距离
\titleformat{\section}{\raggedright\large\bfseries}{\thesection}{1em}{}
\titleformat{\subsection}{\raggedright\normalsize\bfseries}{\thesubsection}{1em}{}
\titlespacing{\section}{0pt}{*0}{*2}
\titlespacing{\subsection}{0pt}{*0}{*1}
% 由于默认的2em缩进不够,所以我手动调整了,但是在windows下似乎2.2就差不多了,或者是article中没有这个问题
\setlength{\parindent}{2.2em}

% 设置表格标题前后间距
\setlength{\abovecaptionskip}{0pt}
\setlength{\belowcaptionskip}{0pt}


\renewcommand{\refname}{\bfseries{参~考~文~献}} %将Reference改为参考文献(用于 article)
% \renewcommand{\bibname}{参~考~文~献} %将bibiography改为参考文献(用于 book)
\renewcommand{\baselinestretch}{1.38} %设置行间距
\renewcommand{\figurename}{\small\ttfamily 图}
\renewcommand{\tablename}{\small\ttfamily 表}


\usepackage{stmaryrd}
\usepackage{mathtools}
\usepackage{wasysym}
\usepackage{textcomp}
\usepackage{subfiles}

\newtheorem{definition}{定义}
\numberwithin{definition}{section}
\newtheorem{lemma}{引理}
\numberwithin{lemma}{section}
\newtheorem{proposition}{命题}
\numberwithin{proposition}{section}
\newtheorem{theorem}{定理}
\numberwithin{theorem}{section}
\newtheorem{grammar}{文法}
\numberwithin{grammar}{section}
\newtheorem{program}{程序}
\numberwithin{program}{section}
\newtheorem{convention}{约定}
\numberwithin{convention}{section}
\newtheorem{corollary}{推论}
\numberwithin{corollary}{section}
\renewcommand*{\proofname}{证明}

\xeCJKsetwidth{‘’“”}{1em}

\title{本人自述}
\date{\nianyueri\today}
\author{苑明理}

\begin{document}

\maketitle

\centerline{\rule{13cm}{0.4pt}}
\renewcommand{\contentsname}{\hfill\bfseries 目录\hfill}
\setcounter{tocdepth}{2}
\tableofcontents
\centerline{\rule{13cm}{0.4pt}}
\newpage

本人苑明理,蜗居于北京的一位求学者。古人论学,有生而知之、学而知之和困而知之的分别,我大概属于困而知之的这一类。
不过,我认为古人的顺序说反了,并不符合知识创造的规律。学问发端于问,万事创制艰难,筚路蓝缕才能写成文章;
等到一种学问有了体系和沿承,学而知之的比例才会越来越大;但学问万古常新,不因循旧论而有新意的地方,还又会不断从新的困惑中产生。
我今年四十六岁,在这番年纪,始有志于学,原因在哪里?我会在下面逐一解释。

\begin{enumerate}
    \item 个人经历:简述个人的学习和工作经历,着重写和一个数学问题结缘的过程
    \item 教育背景:本科、硕士、工作各个阶段的学习情况,学习方法和一些心得
    \item 研究的问题与计划:
    \item 博士期间的学习:结合攻克问题的方向,尽快补齐知识和技巧上的短板。
    \item 更多的思考:更多地思考背景,一部分志趣所在,或许会和更长时间段上的工作有关系。
\end{enumerate}

\newpage

\section{个人经历}

1994年我就读于山东工业大学计算数学专业(后并入山东大学),1998年获得学士学位。其后短暂工作了两年,用了三个月的准备考入了北京大学数学系,
2001年9月入校学习。2002年10月份,我在北大读应用数学硕士时,和新加坡的友人一起汉化了维基百科,所以我是中文维基百科的发起人之一。
此后多年也致力于跨越地域的分歧,共同创造一个属于中文的互联网百科全书。2005年 1 月我获得硕士学位,离开北大到了上海,先后在 HP、eBay 任职,
从事工程技术工作;2010年,人生低谷中困惑的我,开始了一个浪漫的尝试—瓦克星计划,用计算机去模拟一个虚拟星球。2012 年我回到北京,成家立业。
经过一年的中转,在 2014 年和朋友一起创立了彩云科技,把最新的深度学习和天气预报结合起来,是国内最早把短临预报大规模推向公众的单位。
我也是公司的联合创始人,先后任 CTO 和数据科学家。

或许可以这样一直走下去,但内心敏感的我常有困惑,2016年 40 岁时我曾留下一段文字

\begin{displayquote}
“十余年程序员生涯,游走于工程、语言、知识、智能、数学这些关键词之间,内心对世界充满很多好奇和困惑,尝听友人讲大刘《山》的故事,
也期自己可凭蛮力,凿空厚壁,得见星空。”
\end{displayquote}

大刘《山》的故事是小说家刘慈欣的一篇短篇小说,是一个现代版的望洋兴叹的故事。当时的我已经开始和一个数学问题结缘了,
那个时候认识还不深。我在尚未完成的附件一里有段文字写前后的经历:

\begin{displayquote}
    事情可以追溯到 2015 年底,当时我在上海短居。在考虑用词向量技术表达整数的时候,
    我发现可以构造一种有趣的、双曲平面上的离散结构。当看到那些蜿蜒到无穷的折线和
    均匀的数值分布时,我意识到可能发现了一个有着丰富内涵的数学结构。

    其后的几年我一直在反复琢磨它。 2019 年 10 月底我去美国出差,11 月初在回国的飞机上,
    我偶然发现了这种构造方式可以推广到无穷小。一两天后,我得到了双曲平面上无穷小生成
    结构的流方程。和中山大学的蒋文峰老师讨论后,他给到我很多好的建议,鼓励我一步步走
    下去。

    2019 年底到 2020 年夏,我在努力把这个结果应用到一种重要的神经网络上—残差网络,
    可以得到一类特殊的网络架构来解决演化问题。几个月的努力,逐步得到了一些有趣的神
    经网络架构,虽然在一些演化问题上它还没有达到工业界的最佳水平,但也有着良好的性能,
    至少可以说明这种几何化的想法在实践中是可行的。这些应用角度的努力,也可以从理论角
    度给予解读,是在探索另外一种微积分的可能性。

    2020 年底,在和英国的刘宇老师讨论后,他问了我几个特别好的问题,促使我重新思考,
    结果发现无穷小生成元的选择是自由的,这样就导出了一个非常深刻的管型构造, 这个
    管型构造能和多项式联系起来,是一种纤维结构。但是这些几何构造还仅仅存在于设想中,
    它还欠缺一个严密的理论基础和经过证明的实例。

    于是,在 2021 年初,我从公司(彩云科技)争取了三个月的时间,在实习生张乐和张怀公的
    帮助下,我们一起找到了第一个严格的、可以解算的实例(张乐的工作),也找到了一般情况
    下的表述方法(张怀公的工作)。
\end{displayquote}

经过六、七年的探索,我觉得要做一个不容易的人生选择,我的选择是回到数学。人要对事情有判断和基于判断的胆识。基于我个人的理解,我认为这个选择值得。

\newpage

\section{教育背景}

\subsection{本科阶段}

求学的本科阶段,我爱泡图书馆,图书馆里的数学与物理部分都被我翻遍了。因为是 1990 年代初,当时国内翻译的教材和读物还有限,选择比较少,但我仍然注
重多本经典教材彼此参照,数学与物理的相互印证。

在低年级阶段,数学分析的学习,除了课堂跟随老师,我自己又以 Richard Courant 和 Fritz John 的《微积分和数学分析引论》为底本,结合华罗庚的
《高等数学引论》,同时还与《费曼物理学讲义》彼此对照。我至今记得费曼讲的场论基础清晰而直白,可以快速进入。高等代数的学习,老师以北大的《高等代数》
为教材,我自己又选择许以超老师的《线性代数与矩阵论》作为参照。

进到高年级,因为学校里没有开设类似课程,我自修了陈维桓老师的《微分几何》、 J. R. Munkres 的《拓扑学》,读了 V. I. Arnold 的《常微分方程》。
同时,我还注重数学史的学习,M. Kline 的《古今数学思想》常常摆在案头。有些大家的普及性小册子也在我的阅读范围内,比如冯承天先生译的 H. Weyl 的
《对称》,齐民友先生译的 V. I. Arnold 的《初等拓扑的直观概念》,等等。

在高中的时候,我自己曾经提问过一个问题,光线射入到球面里会出来吗?大一的时候,我了解到了一定可以射出来,证明竟然用到抽屉原理或者鸽笼原理。但后来
发现,入射椭圆的光线,其中过焦点的那些竟然可以出不来,当时很震惊这样的结果。高年级的时候,从 Arnold 的书里,我知道了庞加莱回归定理。这个大一时
候的问题给我留下了很深的印象,于是在大四毕业时,我选择研究这个问题作为本科毕业论文。我边思考问题边学习,最后用了射影几何的齐次坐标,可以方便的表
示切线,花了一个月的时间推导出这个环形上的动力系统,它的相空间被过焦点的射线分为了三个不变子集;剩下了一个月时间,想去证明三个不变子集中有两个有
遍历性,但是太难,没有结果。

\subsection{硕士阶段}

硕士阶段的求学,我在张乃孝老师门下,学习软件的形式化方法,当时对程序设计语言非常感兴趣,而张老师有对领域语言的研究。因为想体会一下一个领域语
言的设计和实现究竟是怎么样的?我的硕士论文偏工程性质,设计和部分实现了一个工作流语言。匆匆三年一晃而过,很过功课只是开了个头,简单记录一下对日后
工作和思考有帮助的学习内容。

\subsubsection{代数语义初步}

张乃孝老师有形式语义的讨论班,我在其中介绍代数语义的初步内容。整个介绍中,我对多调代数(many-sorted algebra)如何通过项代数(term algebra)
获得初始语义(initial algebra)的手法印象特别深刻,这个手法在后面数理逻辑的学习里也遇到了。这些内容影响了我对算术和逻辑的理解。

\subsubsection{程序设计语言}

程序设计语言这门课是裘宗燕老师授课,通过这门课基本了解了程序设计语言里各种概念的历史源流和作用。裘老师治学严谨,深受同学爱戴。
我很好的完成了大作业,给出了 Self 语言的一个详细介绍,是课程里唯一获得 90 分以上的同学。这些学习和我后面对
元对象协议(MOP,meta-object protocol)的探索有关系。

\subsubsection{可计算理论初步}

当时旁听了计算机科学系的计算理论的课程,基本理解了自动机和形式语言的初步框架。在参考其他文献时,发现了 Neil D. Jones 的
\textit{Computability and Complexity: From a Programming Perspective} 这本书。作者新鲜的讲法让我非常兴奋,很多定理如果
基于图灵机去证明,会绕来绕去,特别繁复,在这本书里用 While 语言的程序去证明就显得清晰易懂。

\subsubsection{数理逻辑初步}

因为程序精化(refinement)的课程里有 Hoare logic 的内容,我去自修了一部分数理逻辑的内容。本科时也有数理逻辑的初阶课程,所以原以为数理逻辑只是
一些推演规则的体系。但当时看了王浩先生的《数理逻辑通俗讲话》才理解到逻辑学起源于对形式系统和客观世界之间关系的探讨,其中对形式系统的有效性和完全
性的讨论是代数味道非常浓厚的部分。后来参考到了李未老师的《数理逻辑》,书中给出了谓词逻辑完全性的一种证明,是通过 Hintikka 集来处理的。
这个处理手法,和代数语义里通过项代数获得初始语义的方法,是如出一辙的。这背后有很浓厚的哲学意味,后来有哲学系同学告诉我,这和康德处理唯理论和经验
论之争的思路是类似的。

\subsection{日积月累}

\subsubsection{编程的工程实践与探索}

函数式编程、编译器与库的设计、MOP

\subsubsection{机器学习的工程实践}

\subsubsection{短临预报系统里的问题}

\subsubsection{有效过程与自然法则}

\newpage

\section{研究的问题与计划}

\subsection{研究题目}

\subsection{异同辨析}

与目前 SL(2) 研究异同的辨析

\subsection{当前进展}

\subsection{研究意义}

分析部分

几何部分

计算部分

\subsection{研究路径}

基础部分

通路群如何引入拓扑构成拓扑群,甚至进一步也是李群?交换子的作用?
已经有文献给出了拓扑群生成元的研究

分析部分

变分法的计算技巧,狄利克莱能量的计算

几何部分

计算部分

\newpage

\section{博士期间的学习}

黎曼流形

黎曼单值化定理的证明和计算

拓扑群、李群、SL(2)

变分法

\newpage

\section{更多的思考}

\subsection{学习与停机问题}

\subsection{机器能否数数的思考}

2021 年 1 月集智年会上我做了一个小的分享,去探讨机器能不能数数的问题。

表面上看似乎不是问题,机器本来不就是高级版本的计算器嘛!实则不然,我其实在这里提了
一个不同于图灵测试的智能体的创造性测试问题。图灵测试是以人类智能为标准的,而这里的
机器数数的测试问题则是以一种超出了人类智能的方式来设问的。

在演示文档中,我有一层层的剥开这个问题,先考察生物和认知里的数,继而从目前人类的形
式定义去考察,发现想把这个问题定义清楚都比较困难。 困难在于,无法简单的排除掉智能体
可以创造出和人类不同的数的形式,但这种情况下,你无法给出一个数的标准定义,从而也很
难给出能不能数数问题的测试标准。 最后一段,我尝试用强化学习的思路,创造一个生存环境,
智能体必须通过计数的方法来获得更优的选择,然后不限定智能体的策略的制定,如此从侧面
去考察数的创造。

这种强化学习的思路,最终还是通过报酬或者激励的量化来驱动策略的搜索,也就是把数的创造
问题化归为一个势能面上的寻优。这背后是一种\textbf{几何式的元数学}观点。

扩展开来,这里面其实有个很深的智能体之间彼此理解的问题。
我对沙漠蚁计步的能力就很好奇:沙漠蚁在觅食时可以走很复杂的路线,但是它们一旦归巢,
就会直奔巢穴的位置。 沙漠蚁在这里展现出来的空间路径积分的能力,其实不比人类最开始掰
手指头数那么几百个数要低。我想的是数学能力其实是一种具身性的能力,而数学形式也受具
身性的影响。 想要隔着物种或者其他的隔阂去理解的话,必须得深入到那个具身环境里去理
解。但我倒不是否定柏拉图理念式的数学,我是想探索这些不同具身环境下的数学形态的某种
公约数。当生物逐渐从昆虫演化到了哺乳动物,它们在海马体里出现了导航神经细胞,这些细
胞有六边形的活跃模式。那么这种神经层面上六边形活跃模式对应的几何, 怎么和我们心理
层面的几何挂上钩的呢?从数学角度如何理解呢?所以还可以跨越神经、认知和数学,去思考
更多。

上面提到的公约数可能也是理解人类符号化知识的一个有趣切入的方向。考察这种异与同背后,
其实还有一个问题是什么是真实?有没有这种真实? 如果没有这种真实,智能体之间能互相理
解吗?这背后的指涉,远不止在具身性数学的探讨,如今混乱的人类世界或许也需要深刻反思
这个问题。

\subsection{环形的推理结构}

因为上面提到了多种数学形态,和寻求公约数。我在想另外一种推理的方式。我们学习实数理论
或者复变函数时,不少教科书就采用了一种环式的论证,它不是循环论证,而是通过一个环形的
蕴含来论证一定条件下的概念间的等价性的。

我在想,这种环式的推理是可以被挖掘的,有趣的一个问题,这个环包围的空有没有意义呢?
你可以想到同伦或者某种语义的空间。它就不单纯是推理的链条了。相反,我们要跳出推理链条,
从链条的外面来看。

有趣的是,在算术表达式的几何里,我就是从词向量这种语义空间的构造开始的。而得到的算术
表达式空间,可以被赋值函数的等值线与梯度分解。我们知道等值线就是逻辑上的等值关系,
梯度可以理解成一种特殊的推理机制。既然算术表达式被几何化了,那么是否有可能逻辑也可以
被几何化?

\subsection{向下去思考}

\section{结语}

背水一战



\end{document}




