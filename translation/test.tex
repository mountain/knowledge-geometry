\documentclass[a4paper,12pt]{article}
\usepackage{amsmath, amsthm}
\usepackage{datetime}
\usepackage{framed}
\usepackage{enumitem}
\usepackage{fancyref}
\usepackage{wrapfig}
\usepackage{pifont}
\usepackage{appendix}
\usepackage{caption}
\usepackage{xcolor}
\usepackage[stable]{footmisc}
\usepackage{multicol}
\usepackage{csquotes}

\usepackage{amsthm}
\usepackage{amssymb}
\usepackage{amsfonts}
\usepackage{amsmath}
\usepackage{mathtools}

\usepackage{tikz}
\usepackage{pgf}
\usepgflibrary{fpu}
\usepackage{qtree}
\usetikzlibrary{angles,fit,arrows,calc,math,intersections,through,backgrounds}
\usepackage{pgfplots}
\usepackage{tkz-euclide}

\usepackage{listings}
\lstset{
  basicstyle=\itshape,
  xleftmargin=3em,
  literate={->}{$\rightarrow$}{2}
           {α}{$\alpha$}{1}
           {δ}{$\delta$}{1}
}


\usepackage{csquotes}
\renewcommand{\mkbegdispquote}[2]{\itshape}

\newdateformat{nianyueri}{ \THEYEAR 年 \THEMONTH 月 \THEDAY 日 }

\usepackage{xstring}
\usepackage{catchfile}
\CatchFileDef{\HEAD}{../.git/refs/heads/master}{}
\newcommand{\gitrevision}{%
  \StrLeft{\HEAD}{7}%
}

\usepackage{data/quiver}
\usepackage{data/circledsteps}
\usepackage[top=1in,bottom=1in,left=1in,right=1in]{geometry} % 用于设置页面布局
\usepackage{xeCJK} % 用于使用本地字体
\usepackage[super, square, sort&compress]{natbib} % 处理参考文献
\usepackage{titlesec, titletoc} % 设置章节标题及页眉页脚
\usepackage{amssymb}
\usepackage{amsmath} % 在公式中用\text{文本}输入中文
\usepackage{diagbox}
\usepackage{multirow} % 表格中使用多行
\usepackage{booktabs} % 表格中使用\toprule等命令
\usepackage{rotating} % 使用sidewaystable环境旋转表格
\usepackage{tabularx}
\usepackage{graphicx} % 处理图片
\usepackage{footnote} % 增强的脚注功能,可添加表格脚注
\usepackage{threeparttable} % 添加真正的表格脚注,示例见README
\usepackage{hyperref} % 添加pdf书签

\usepackage{tikz}
\usetikzlibrary{shapes,arrows,shadows}


% 字体设置
\setmainfont{Times New Roman}
\setsansfont[Scale=MatchLowercase,Mapping=tex-text]{PT Sans}
\setmonofont[Scale=MatchLowercase]{PT Mono}
\setCJKmainfont[ItalicFont={FZKai-Z03}, BoldFont={FZHei-B01}]{FZShuSong-Z01}
\setCJKsansfont{FZHei-B01}
\setCJKmonofont{FZShuSong-Z01}

\newcommand{\song}{\CJKfamily{song}} % 宋体
\newcommand{\fs}{\CJKfamily{fs}} % 仿宋体
\newcommand{\kai}{\CJKfamily{kai}} % 楷体
\newcommand{\hei}{\CJKfamily{hei}} % 黑体
\newcommand{\li}{\CJKfamily{li}} % 隶书
\newcommand{\you}{\CJKfamily{you}} % 幼圆
\def\songti{\song}
\def\fangsong{\fs}
\def\kaishu{\kai}
\def\heiti{\hei}
\def\lishu{\li}
\def\youyuan{\you}

%%设置常用中文字号,方便调用
\newcommand{\chuhao}{\fontsize{42pt}{\baselineskip}\selectfont}
\newcommand{\xiaochu}{\fontsize{36pt}{\baselineskip}\selectfont}
\newcommand{\yihao}{\fontsize{26pt}{\baselineskip}\selectfont}
\newcommand{\xiaoyi}{\fontsize{24pt}{\baselineskip}\selectfont}
\newcommand{\erhao}{\fontsize{22pt}{\baselineskip}\selectfont}
\newcommand{\xiaoer}{\fontsize{18pt}{\baselineskip}\selectfont}
\newcommand{\sanhao}{\fontsize{16pt}{\baselineskip}\selectfont}
\newcommand{\xiaosan}{\fontsize{15pt}{\baselineskip}\selectfont}
\newcommand{\sihao}{\fontsize{14pt}{\baselineskip}\selectfont}
\newcommand{\xiaosi}{\fontsize{12pt}{\baselineskip}\selectfont}
\newcommand{\wuhao}{\fontsize{10.5pt}{\baselineskip}\selectfont}
\newcommand{\xiaowu}{\fontsize{9pt}{\baselineskip}\selectfont}
\newcommand{\liuhao}{\fontsize{7.5pt}{\baselineskip}\selectfont}
\newcommand{\xiaoliu}{\fontsize{6.5pt}{\baselineskip}\selectfont}
\newcommand{\qihao}{\fontsize{5.5pt}{\baselineskip}\selectfont}
\newcommand{\bahao}{\fontsize{5pt}{\baselineskip}\selectfont}

% 章节标题显示方式及页眉页脚设置
% \item xCJKnumb是自己额外安装的包
% \item titleformat命令定义标题的形式
% \item titlespacing定义标题距左、上、下的距离
\titleformat{\section}{\raggedright\large\bfseries}{\thesection}{1em}{}
\titleformat{\subsection}{\raggedright\normalsize\bfseries}{\thesubsection}{1em}{}
\titlespacing{\section}{0pt}{*2}{*0}
\titlespacing{\subsection}{0pt}{*1}{*0}

% 由于默认的2em缩进不够,所以我手动调整了,但是在windows下似乎2.2就差不多了,或者是article中没有这个问题
\setlength{\parindent}{0em}
\setlength{\parskip}{0.25em}

% 设置表格标题前后间距
\setlength{\abovecaptionskip}{0pt}
\setlength{\belowcaptionskip}{0pt}

% 设置列表项目前后间距
\setlength\itemsep{0em}

\renewcommand{\refname}{\bfseries{参~考~文~献}} %将Reference改为参考文献(用于 article)
% \renewcommand{\bibname}{参~考~文~献} %将bibiography改为参考文献(用于 book)

\renewcommand{\baselinestretch}{1.4} %设置行间距
\renewcommand{\figurename}{\small\ttfamily 图}
\renewcommand{\tablename}{\small\ttfamily 表}


\usepackage{stmaryrd}
\usepackage{mathtools}
\usepackage{wasysym}
\usepackage{textcomp}
\usepackage{blindtext}
\usepackage{subfiles}

\newtheorem{problem}{问题}
\numberwithin{problem}{section}
\newtheorem{definition}{定义}
\numberwithin{definition}{section}
\newtheorem{lemma}{引理}
\numberwithin{lemma}{section}
\newtheorem{proposition}{命题}
\numberwithin{proposition}{section}
\newtheorem{theorem}{定理}
\numberwithin{theorem}{section}
\newtheorem{grammar}{文法}
\numberwithin{grammar}{section}
\newtheorem{program}{程序}
\numberwithin{program}{section}
\newtheorem{convention}{约定}
\numberwithin{convention}{section}
\newtheorem{corollary}{推论}
\numberwithin{corollary}{section}
\renewcommand*{\proofname}{证明}

\xeCJKsetwidth{‘’“”}{1em}

\title{欧拉作品试译}
\date{\nianyueri\today}
\author{苑明理}

\begin{document}

\begingroup
\let\newpage\relax
\maketitle
\endgroup

拉丁原文\footnote{出自 Euler 的论文 De Sono 的第一章《关于声音的性质和传播》}
\begin{displayquote}
Obscura admodum atque confusa fuit vererum Philosophorum soni explicatio,
quantum ex scriptis eorum nobis relictis intelligi potest. Alii, cum Epicuro sonum,
istar fluminis ex corporibus sonoris pulsatis emanare statuerunt.
Alii autem \& præprimis interpretes ARISTOTLIS latini cum illo naturam soni posuerunt in fractione æris,
quæ oritur ex collisione vehementior corporum. Inter recentiores HONORATUS FABRI atque CARTESIUS invenerunt sonum
consistere in æris tremore, de isto autem tremore pariter confuse sentiebant. Acutissimus NEUTONUS,
hanc rem accuratius expendere atque exponere aggressus est, præcipue soni propagationem explicando,
verum parum feliciori successu. Arduam ergo hanc de sone materiam, istac in dissertatione tractare,
atque pro viribus dilucidare constitui, duobus capitibus eam comprehenendo.
Priore hoc capite scilicet perpendetur, quare sonus consistat, \& quomodo ab uno loco ad alium propagetur.
In posteriore autem, tres sonum producendi : considerabuntur.
\end{displayquote}

英语译文\footnote{出自 Ian Bruce 的翻译}
\begin{displayquote}
The explanation of sound by the old philosophers was very obscure and confused, so much can be understood from their
writings that have come down to us. Some were of the opinion, like Epicures [341 - 270 B. C.], that sound emanated from
a pulsating body rather like the flow of a river; while others with the foremost of the Latin writers, believed with
Aristotle, [384 - 322 B. C.] that sounds were formed from the breaking of the air which arose from the more violent
collisions of bodies. Among the more recent commentaries, Honoré Fabri [1608 - 1688], and Descartes [1596 - 1650],
discovered that sound consisted of tremors or vibrations of the air, but their reasoning concerning these vibrations
were equally confused. Newton [1643 - 1727], with the sharpest of minds, considered the matter with more care,
and undertook to set forth an explanation especially for the propagation of sound, truly with much more success.
A determined effort has been made [by me] to grasp the difficult matters involved in an understanding of the nature of
sound, which are set out in the two chapters of this dissertation. In the first chapter it becomes apparent, after some
careful thought, what the nature of sound really is, and how it is propagated from one place to another. Moreover,
in the following chapter, three sources of sound are to be considered.
\end{displayquote}

汉语译文\footnote{在彩云小译帮助下,有 ChatGPT 第一边逐句校读,苑明理第二遍逐句校读,整理后成文}
\begin{displayquote}
古代哲学家对声音的解释非常晦涩难懂,从他们留传下来的著作中能够理解的并不多。有些人,例如伊壁鸠鲁(公元前341年-270年),
认为声音是由发出有规律的振动的物体产生的,就像河水的流动一样;而另一些人,尤其是拉丁作家中最著名的,相信与亚里士多德(公元前384年-322年)在一起,
认为声音是由更强烈的物体碰撞所产生的气体震动而形成的。在更近期的评论中,Honoré Fabri(1608年-1688年)和笛卡尔(1596年-1650年)
发现声音由空气的震颤或振动组成,但是他们关于这些振动的推理同样很混乱。牛顿(1643年-1727年)拥有极其敏锐的思维,对这个问题进行了更加仔细的探究,
尤其是关于声音的传播,他承诺提出更为成功的解释。我决心努力理解涉及声音本质的困难问题,并在本论文的两个章节中对其进行了阐述。
在第一章中,经过深思熟虑,我们将了解到声音的本质究竟是什么,以及它是如何从一个地方传播到另一个地方的。此外,在随后的章节中,我们还将探讨三种声音的来源。
\end{displayquote}

\end{document}