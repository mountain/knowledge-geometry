\documentclass[a4paper,12pt]{article}
\usepackage{datetime}
\usepackage{amsmath, amsthm, amssymb}
\usepackage[dvipsnames]{xcolor}
\usepackage{enumitem}
\usepackage{fancyref}

\usepackage[top=1in,bottom=1in,left=1in,right=1in]{geometry} % 用于设置页面布局
\usepackage{xeCJK} % 用于使用本地字体
\usepackage[super, square, sort&compress]{natbib} % 处理参考文献
\usepackage{titlesec, titletoc} % 设置章节标题及页眉页脚
%\usepackage{xCJKnumb} % 中英文数字转换
\usepackage{amssymb}
\usepackage{amsmath} % 在公式中用\text{文本}输入中文
\usepackage{diagbox}
\usepackage{multirow} % 表格中使用多行
\usepackage{booktabs} % 表格中使用\toprule等命令
\usepackage{rotating} % 使用sidewaystable环境旋转表格
\usepackage{tabularx}
\usepackage{graphicx} % 处理图片
\usepackage{footnote} % 增强的脚注功能,可添加表格脚注
\usepackage{threeparttable} % 添加真正的表格脚注,示例见README
\usepackage{hyperref} % 添加pdf书签

\usepackage{tikz}
\usetikzlibrary{shapes,arrows,shadows}

% 字体设置
\setmainfont{Times New Roman}
\setsansfont[Scale=MatchLowercase,Mapping=tex-text]{PT Sans}
\setmonofont[Scale=MatchLowercase]{PT Mono}
\setCJKmainfont[ItalicFont={Kaiti SC}, BoldFont={Heiti SC}]{Songti SC}
\setCJKsansfont{Heiti SC}
\setCJKmonofont{Songti SC}
% \setCJKmainfont[BoldFont={FZXiaoBiaoSong-B05S}]{Songti SC}
% \setCJKfamilyfont{kai}[BoldFont=Heiti SC]{Kaiti SC}
% \setCJKfamilyfont{song}[BoldFont=Heiti SC]{Songti SC}
% \setCJKfamilyfont{hei}[BoldFont=Heiti SC]{Heiti SC}
% \setCJKfamilyfont{fsong}[BoldFont=Heiti SC]{Songti SC}
% \newcommand{\kai}[1]{{\CJKfamily{kai}#1}}
% \newcommand{\hei}[1]{{\CJKfamily{hei}#1}}
% \setromanfont[Mapping=tex-text]{TeXGyrePagella}
% \setsansfont[Scale=MatchLowercase,Mapping=tex-text]{TeXGyrePagella}
% \setmonofont[Scale=MatchLowercase]{Courier New}
%%设置常用中文字号,方便调用
\newcommand{\erhao}{\fontsize{22pt}{\baselineskip}\selectfont}
\newcommand{\xiaoerhao}{\fontsize{18pt}{\baselineskip}\selectfont}
\newcommand{\sanhao}{\fontsize{16pt}{\baselineskip}\selectfont}
\newcommand{\xiaosanhao}{\fontsize{15pt}{\baselineskip}\selectfont}
\newcommand{\sihao}{\fontsize{14pt}{\baselineskip}\selectfont}
\newcommand{\xiaosihao}{\fontsize{12pt}{\baselineskip}\selectfont}
\newcommand{\wuhao}{\fontsize{10.5pt}{\baselineskip}\selectfont}
\newcommand{\xiaowuhao}{\fontsize{9pt}{\baselineskip}\selectfont}
\newcommand{\liuhao}{\fontsize{7.5pt}{\baselineskip}\selectfont}

% 章节标题显示方式及页眉页脚设置
% \item xCJKnumb是自己额外安装的包
% \item titleformat命令定义标题的形式
% \item titlespacing定义标题距左、上、下的距离
\titleformat{\section}{\raggedright\large\bfseries}{\thesection}{1em}{}
\titleformat{\subsection}{\raggedright\normalsize\bfseries}{\thesubsection}{1em}{}
\titlespacing{\section}{0pt}{*0}{*2}
\titlespacing{\subsection}{0pt}{*0}{*1}
% 由于默认的2em缩进不够,所以我手动调整了,但是在windows下似乎2.2就差不多了,或者是article中没有这个问题
\setlength{\parindent}{2.2em}

% 设置表格标题前后间距
\setlength{\abovecaptionskip}{0pt}
\setlength{\belowcaptionskip}{0pt}


\renewcommand{\refname}{\bfseries{参~考~文~献}} %将Reference改为参考文献(用于 article)
% \renewcommand{\bibname}{参~考~文~献} %将bibiography改为参考文献(用于 book)
\renewcommand{\baselinestretch}{1.38} %设置行间距
\renewcommand{\figurename}{\small\ttfamily 图}
\renewcommand{\tablename}{\small\ttfamily 表}


\setlength{\parindent}{0em}

\newcommand{\specialcell}[2][c]{%
  \begin{tabular}[#1]{@{}c@{}}#2\end{tabular}}

\newtheorem{definition}{定义}
\newtheorem{lemma}{引理}
\newtheorem{proposition}{命题}
\newtheorem{theorem}{定理}
\newtheorem{grammar}{文法}
\newtheorem{program}{程序}
\newtheorem{convention}{约定}
\renewcommand*{\proofname}{证明}

\usetikzlibrary{shapes.geometric}
\tikzset{
    turtle/.style={
        draw,
        shape border rotate=270,
        regular polygon,
        regular polygon sides=3,
        fill=gray,
        node distance=2cm,
        minimum height=4em
    }
}

\newdateformat{monthyeardate}{\THEYEAR 年 \THEMONTH 月}

\title{循环推证与知识流形}
\author{苑明理}
\date{\monthyeardate\today}

\begin{document}
\begingroup
\let\newpage\relax
\maketitle
\endgroup

\renewcommand\contentsname{目录}
\setcounter{tocdepth}{2}
\tableofcontents

\newpage

\section{从实数的完备性说起}

\subsection{实数的完备性}

经典的教科书上,往往有如下几个定理的彼此推证,下面我们从知识体系的角度系统的研究这些推证。

\begin{proposition}[Sup]
  任何有上界的非空子集都有上确界。
\end{proposition}

\begin{proposition}[Cau]
  柯西序列收敛。
\end{proposition}

\begin{proposition}[Nes]
  闭区间套的交非空。
\end{proposition}

\begin{proposition}[Mon]
  单调有界序列收敛。
\end{proposition}

\begin{proposition}[Bol]
  有界序列存在子列收敛。
\end{proposition}

\begin{proposition}[Int]
  连续函数一侧小于零,一侧大于零,则必有介值为根。
\end{proposition}

\subsection{彼此推证}

于是两两搭配得到 30 对推证,既有如下 30 个定理。

\subsubsection{以确界性为起点}

\begin{theorem}[Sup => Cau]
  如果任何有上界的非空子集都有上确界,则柯西序列收敛。
\end{theorem}
\begin{proof}
    得证。
\end{proof}

\begin{theorem}[Sup => Nes]
  如果任何有上界的非空子集都有上确界,则闭区间套的交非空。
\end{theorem}
\begin{proof}
    得证。
\end{proof}

\begin{theorem}[Sup => Mon]
  如果任何有上界的非空子集都有上确界,则有界单调序列收敛。
\end{theorem}
\begin{proof}
    得证。
\end{proof}

\begin{theorem}[Sup => Bol]
  如果任何有上界的非空子集都有上确界,则有界序列存在子列收敛。
\end{theorem}
\begin{proof}
    得证。
\end{proof}

\begin{theorem}[Sup => Int]
  如果任何有上界的非空子集都有上确界,同时连续函数一侧小于零,一侧大于零,则必有介值为根。
\end{theorem}
\begin{proof}
    得证。
\end{proof}

\subsubsection{以柯西序列为起点}

\begin{theorem}[Cau => Sup]
  如果柯西序列收敛,则任何有上界的非空子集都有上确界。
\end{theorem}
\begin{proof}
    得证。
\end{proof}

\begin{theorem}[Cau => Nes]
  如果柯西序列收敛,则闭区间套的交非空。
\end{theorem}
\begin{proof}
    得证。
\end{proof}

\begin{theorem}[Cau => Mon]
  如果柯西序列收敛,则有界单调序列收敛。
\end{theorem}
\begin{proof}
    得证。
\end{proof}

\begin{theorem}[Cau => Bol]
  如果柯西序列收敛,则有界序列存在子列收敛。
\end{theorem}
\begin{proof}
    得证。
\end{proof}

\begin{theorem}[Cau => Int]
  如果柯西序列收敛,同时连续函数一侧小于零,一侧大于零,则必有介值为根。
\end{theorem}
\begin{proof}
    得证。
\end{proof}

\subsubsection{以闭区间套为起点}

\begin{theorem}[Nes => Sup]
  如果柯西序列收敛,则任何有上界的非空子集都有上确界。
\end{theorem}
\begin{proof}
    得证。
\end{proof}

\begin{theorem}[Nes => Cau]
  如果柯西序列收敛,则闭区间套的交非空。
\end{theorem}
\begin{proof}
    得证。
\end{proof}

\begin{theorem}[Nes => Mon]
  如果柯西序列收敛,则有界单调序列收敛。
\end{theorem}
\begin{proof}
    得证。
\end{proof}

\begin{theorem}[Nes => Bol]
  如果柯西序列收敛,则有界序列存在子列收敛。
\end{theorem}
\begin{proof}
    得证。
\end{proof}

\begin{theorem}[Nes => Int]
  如果柯西序列收敛,同时连续函数一侧小于零,一侧大于零,则必有介值为根。
\end{theorem}
\begin{proof}
    得证。
\end{proof}

\subsubsection{以单调序列为起点}

\begin{theorem}[Nes => Sup]
  如果柯西序列收敛,则任何有上界的非空子集都有上确界。
\end{theorem}
\begin{proof}
    得证。
\end{proof}

\begin{theorem}[Nes => Cau]
  如果柯西序列收敛,则闭区间套的交非空。
\end{theorem}
\begin{proof}
    得证。
\end{proof}

\begin{theorem}[Nes => Mon]
  如果柯西序列收敛,则有界单调序列收敛。
\end{theorem}
\begin{proof}
    得证。
\end{proof}

\begin{theorem}[Nes => Bol]
  如果柯西序列收敛,则有界序列存在子列收敛。
\end{theorem}
\begin{proof}
    得证。
\end{proof}

\begin{theorem}[Nes => Int]
  如果柯西序列收敛,同时连续函数一侧小于零,一侧大于零,则必有介值为根。
\end{theorem}
\begin{proof}
    得证。
\end{proof}

\section{实数的构造}

\section{解析函数}

\section{思考与总结}

\end{document}
