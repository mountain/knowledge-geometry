
\documentclass[a4paper,12pt]{article}
\usepackage{hieroglf} % 添加古埃及象形文字
\usepackage{pifont} % 添加 Dingbat
\usepackage{tikz}
\usepackage{stmaryrd}
\usepackage{textcomp}
\usepackage{subfiles} % 添加绘图

\usepackage[top=1in,bottom=1in,left=1in,right=1in]{geometry} % 用于设置页面布局
\usepackage{xeCJK} % 用于使用本地字体
\usepackage[super, square, sort&compress]{natbib} % 处理参考文献
\usepackage{titlesec, titletoc} % 设置章节标题及页眉页脚
%\usepackage{xCJKnumb} % 中英文数字转换
\usepackage{amssymb}
\usepackage{amsmath} % 在公式中用\text{文本}输入中文
\usepackage{diagbox}
\usepackage{multirow} % 表格中使用多行
\usepackage{booktabs} % 表格中使用\toprule等命令
\usepackage{rotating} % 使用sidewaystable环境旋转表格
\usepackage{tabularx}
\usepackage{graphicx} % 处理图片
\usepackage{footnote} % 增强的脚注功能,可添加表格脚注
\usepackage{threeparttable} % 添加真正的表格脚注,示例见README
\usepackage{hyperref} % 添加pdf书签

\usepackage{tikz}
\usetikzlibrary{shapes,arrows,shadows}

% 字体设置
\setmainfont{Times New Roman}
\setsansfont[Scale=MatchLowercase,Mapping=tex-text]{PT Sans}
\setmonofont[Scale=MatchLowercase]{PT Mono}
\setCJKmainfont[ItalicFont={Kaiti SC}, BoldFont={Heiti SC}]{Songti SC}
\setCJKsansfont{Heiti SC}
\setCJKmonofont{Songti SC}
% \setCJKmainfont[BoldFont={FZXiaoBiaoSong-B05S}]{Songti SC}
% \setCJKfamilyfont{kai}[BoldFont=Heiti SC]{Kaiti SC}
% \setCJKfamilyfont{song}[BoldFont=Heiti SC]{Songti SC}
% \setCJKfamilyfont{hei}[BoldFont=Heiti SC]{Heiti SC}
% \setCJKfamilyfont{fsong}[BoldFont=Heiti SC]{Songti SC}
% \newcommand{\kai}[1]{{\CJKfamily{kai}#1}}
% \newcommand{\hei}[1]{{\CJKfamily{hei}#1}}
% \setromanfont[Mapping=tex-text]{TeXGyrePagella}
% \setsansfont[Scale=MatchLowercase,Mapping=tex-text]{TeXGyrePagella}
% \setmonofont[Scale=MatchLowercase]{Courier New}
%%设置常用中文字号,方便调用
\newcommand{\erhao}{\fontsize{22pt}{\baselineskip}\selectfont}
\newcommand{\xiaoerhao}{\fontsize{18pt}{\baselineskip}\selectfont}
\newcommand{\sanhao}{\fontsize{16pt}{\baselineskip}\selectfont}
\newcommand{\xiaosanhao}{\fontsize{15pt}{\baselineskip}\selectfont}
\newcommand{\sihao}{\fontsize{14pt}{\baselineskip}\selectfont}
\newcommand{\xiaosihao}{\fontsize{12pt}{\baselineskip}\selectfont}
\newcommand{\wuhao}{\fontsize{10.5pt}{\baselineskip}\selectfont}
\newcommand{\xiaowuhao}{\fontsize{9pt}{\baselineskip}\selectfont}
\newcommand{\liuhao}{\fontsize{7.5pt}{\baselineskip}\selectfont}

% 章节标题显示方式及页眉页脚设置
% \item xCJKnumb是自己额外安装的包
% \item titleformat命令定义标题的形式
% \item titlespacing定义标题距左、上、下的距离
\titleformat{\section}{\raggedright\large\bfseries}{\thesection}{1em}{}
\titleformat{\subsection}{\raggedright\normalsize\bfseries}{\thesubsection}{1em}{}
\titlespacing{\section}{0pt}{*0}{*2}
\titlespacing{\subsection}{0pt}{*0}{*1}
% 由于默认的2em缩进不够,所以我手动调整了,但是在windows下似乎2.2就差不多了,或者是article中没有这个问题
\setlength{\parindent}{2.2em}

% 设置表格标题前后间距
\setlength{\abovecaptionskip}{0pt}
\setlength{\belowcaptionskip}{0pt}


\renewcommand{\refname}{\bfseries{参~考~文~献}} %将Reference改为参考文献(用于 article)
% \renewcommand{\bibname}{参~考~文~献} %将bibiography改为参考文献(用于 book)
\renewcommand{\baselinestretch}{1.38} %设置行间距
\renewcommand{\figurename}{\small\ttfamily 图}
\renewcommand{\tablename}{\small\ttfamily 表}

\setlength{\parindent}{0em}

\newcommand{\specialcell}[2][c]{%
  \begin{tabular}[#1]{@{}c@{}}#2\end{tabular}}

\pmhgfamily

\title{蒯恩与进化}
\author{苑明理}
\date{2021年12月}

\begin{document}

\maketitle{}

\section{记法约定}

\subsection{数}

我们约定小写的希腊字母是实数,也即 $\alpha, \beta, \gamma, \cdots \in R$。

\subsection{函数}

我们大写的斜体希腊字母是函数,也即 $\varGamma , \varPhi , \cdots$ 是函数。

\subsection{向量}

星上标代表多个的意思,可以用来表达向量或者张量,于是 $\alpha^*$ 是一个向量或者张量。
同样有下标会把向量下降为实数,例如 $\alpha^*_i$ 代表一个向量 $\alpha^*$ 的第 $i$ 分量,它是一个实数。

\subsection{空间}

我们用大写拉丁字母代表某种空间,也即 $X, Y, Z \cdots$,用小写拉丁字母代表空间里的一个点,也即有 $a, b, c, \cdots \in X$。

\subsection{机制}

我们用花体字代表各种更复杂的机制,即 $\mathcal{A}, \mathcal{B}, \mathcal{C} \cdots$ 都是机制。

\section{符号约定}

我们共有两个空间和四种机制,约定如下:
\begin{itemize}
  \item 实数空间 $R$
  \item 一般的空间 $X$
  \item 演化动力学:空间 $X$ 上的演化动力学 $\mathcal{U}: X \to X$
  \item 赋值机制: $\mathcal{A}: X \to R$
  \item 赋形机制: $\mathcal{A}^{-1}: R \to X$
  \item 神经网络机制:$\mathcal{N}: R^l \times R \to R$,写法上我们采用柯里化的记法,用 $\mathcal{N}(\alpha^*)$ 表示加载了参数$\alpha^*$的神经网络,而$\mathcal{N}(\alpha^*)(\beta)$表示神经网络作用在输入$\beta$上,而$\mathcal{N}(\alpha^*)(\gamma^*)$表示神经网络成批作用在输入$\gamma^*$上。
\end{itemize}

\section{自打印网络}

张彦博给出的网络可以概括如下,给定任意函数 $\varPhi$,可以训练并得到一组参数 $\theta^*$ 和两组编码 $\eta^*, \zeta^*$ ,使得

$$
\mathcal{N}(\theta^*)(\eta^*) = \theta^*
$$

和

$$
\mathcal{N}(\theta^*)(\zeta^*) = \eta^*
$$

同时有对任意的实数 $\alpha$
$$
\mathcal{N}(\theta^*)(\alpha) = \varPhi(\alpha)
$$

\section{问题}

给定一个初值 $\eta^*_0$,求得

$$\eta^*_k=(\mathcal{A} \circ \mathcal{U} \circ \mathcal{A}^{-1})^k (\eta^*_0)$$

同时有对任意的实数 $\alpha$ 有

$$
\mathcal{N}(\theta^*)(\alpha) = \mathcal{A} \circ \mathcal{U} \circ \mathcal{A}^{-1} (\alpha)
$$



\end{document}
