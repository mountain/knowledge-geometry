\documentclass{article}

\usepackage{arxiv}

\usepackage[utf8]{inputenc} % allow utf-8 input
\usepackage[T1]{fontenc}    % use 8-bit T1 fonts
\usepackage{hyperref}       % hyperlinks
\usepackage{url}            % simple URL typesetting
\usepackage{booktabs}       % professional-quality tables
\usepackage{amsfonts}       % blackboard math symbols
\usepackage{nicefrac}       % compact symbols for 1/2, etc.
\usepackage{microtype}      % microtypography

\usepackage{graphicx}
\usepackage{amsmath}

\newtheorem{definition}{Definition}
\newtheorem{lemma}{Lemma}
\newtheorem{proposition}{Proposition}
\newtheorem{program}{Program}
\newtheorem{convention}{Convention}

\title{Suan: towards a geometry of computation}

%\date{September 9, 1985}	% Here you can change the date presented in the paper title
%\date{} 					% Or removing it

\author{
  Mingli~Yuan \\
  AI Lab \\
  ColorfulClouds Tech.\\
  Beijing, 100083 \\
  \texttt{mingli.yuan@caiyunapp.com} \\
  %% examples of more authors
  %% \AND
  %% Coauthor \\
  %% Affiliation \\
  %% Address \\
  %% \texttt{email} \\
  %% \And
  %% Coauthor \\
  %% Affiliation \\
  %% Address \\
  %% \texttt{email} \\
  %% \And
  %% Coauthor \\
  %% Affiliation \\
  %% Address \\
  %% \texttt{email} \\
}

% Uncomment to remove the date
%\date{}

% Uncomment to override  the `A preprint' in the header
%\renewcommand{\headeright}{Technical Report}
%\renewcommand{\undertitle}{Technical Report}

\begin{document}
\maketitle

\begin{abstract}
    We introduce a new mathematical system Suan $\mathfrak{S}$ which is an embedding of addition-multiplication
    computation into hyperbolic surface. After a local polar coordinate on $\mathfrak{S}$ is defined, a flow
    equation can be conducted, and it shows surreal number can be embedded properly into $\mathfrak{S}$.
\end{abstract}

\keywords{addition-multiplication tree \and hyperbolic surface \and surreal number \and geometry of computation}

\section{A new mathematical system Suan}\label{sec:anmss}

Traditionally, computation was engaged by apply a serial of addition and multiplication on numbers. Any
addition-multiplication expression can be formed into a binary tree, while pure numbers are leaves and addition or
multiplication operators are at branches.

\section{Reverse Polish notation, recursion and discrete path on $\mathfrak{S}$}\label{sec:rpnradp}

\section{Local polar coordinate and flow equation}\label{sec:lpcsafe}

Given any point $x_0$ on $\mathfrak{S}$, an additional axis $A$ pass $x_0$, then for any nearby point $x_1$, two
horocycle $B$ and $\bar{B}$ exists to connect with $x_0$ and $x_1$, so we can consider a local polar coordinate
$\mathfrak{C}_{x_0}$, in which $x_1$ decided by:
\begin{itemize}
    \item the angle $\theta$ between $A$ and $B$
    \item the length $\rho$ of arc between $x_0$ and $x_1$ along $B$
\end{itemize}

$\mathfrak{C}_{x_0}$ has a counterpart $\bar{\mathfrak{C}_{x_0}}$, in which $x_1$ decided by:
\begin{itemize}
    \item the angle $\bar{\theta}$ between $A$ and $\bar{B}$
    \item the length $\bar{\rho}$ of arc between $x_0$ and $x_1$ along $\bar{B}$
\end{itemize}

Question: what dose the two horocycles mean under the computation perspective? how dose the two horocycles contact?

Now we study infinitesimal computation and horocycle flow on $\mathfrak{S}$. By the perspective of
addition-multiplication computation, under a local coordinate $\mathfrak{C}_{x_0}$,
consider a point moving from $x_0$ and towards $x_{\delta}$ with the angle $\theta$ and a small step $\epsilon$, we have

\begin{equation}
    x_{\delta} = (x_0 + \epsilon \cos \theta)(1 + \epsilon \sin \theta)
\end{equation}

or

\begin{equation}
    x_{\delta} = x_0 (1 + \epsilon \sin \theta) + \epsilon \cos \theta
\end{equation}

And both can be simplified to

\begin{equation}
    x_{\delta} = x_0 + \epsilon (x_0 \sin \theta + \cos \theta)
\end{equation}

So we have

\begin{equation}
    \frac{1}{\delta} (x_{\delta} - x_0) = \frac{\epsilon}{\delta} (\cos \theta + x_0 \sin \theta)
\end{equation}

We denote the left side of the above equation as $dx / dt$ when $\delta$ and $\epsilon$ approach 0.

\begin{equation}
    \frac{dx}{dt} = u (\cos \theta + x \sin \theta)
\end{equation}

where $u$ is the moving speed, and by assuming $u = 1$, we have a simpler formula,

\begin{equation}
    \frac{dx}{dt} = \cos \theta + x \sin \theta
\end{equation}

Then we have the solution:

\begin{equation}
   x =  x_0 e^{t \sin \theta} + (e^{t \sin \theta} - 1) \cot \theta
\end{equation}

And then expand to

\begin{equation}
   x =  x_0 e^{t \sin \theta} + [1 + t \sin \theta + \frac{1}{2!} t\sin^2 \theta  + \frac{1}{3!} t \sin^3 \theta + ... - 1] \cot \theta
\end{equation}

Or

\begin{equation}
   x =  x_0 e^{t \sin \theta} + t \cos \theta + \frac{1}{2} \sin 2\theta (\frac{t^2}{2!} + \frac{t^3}{3!} \sin \theta + \frac{t^4}{4!} \sin^2 \theta + ...)
\end{equation}

\section{Example functions}\label{sec:ef}

\subsection{a constant k}

\begin{equation}
    0 = \cos \theta + k \sin \theta
\end{equation}

\begin{equation}
    \theta = n\pi - \arctan \frac{1}{k}
\end{equation}

\subsection{identity}

\begin{equation}
    \cos \theta + t \sin \theta - 1 = 0
\end{equation}

\begin{equation}
    \theta = 2k\pi
\end{equation}
\begin{equation}
    \theta = 2k\pi + \tan^{-1} t
\end{equation}

\subsection{inverse of x}

\begin{equation}
    \frac{d(t^{-1})}{dt} = \cos \theta + t^{-1} \sin \theta
\end{equation}

\begin{equation}
    t^2 \cos \theta + t \sin \theta + 1 = 0
\end{equation}

\subsection{square of x}

\begin{equation}
    \frac{d(t^2)}{dt} = \cos \theta + t^2 \sin \theta
\end{equation}

\begin{equation}
    \cos \theta + t^2 \sin \theta - 2 t = 0
\end{equation}

\subsection{cube of x}

\begin{equation}
    \frac{d(t^3)}{dt} = \cos \theta + t^3 \sin \theta
\end{equation}

\begin{equation}
    \cos \theta + t^3 \sin \theta - 3 t^2 = 0
\end{equation}

\subsection{sin of x}

\begin{equation}
    \frac{d(\sin t)}{dt} = \cos \theta + \sin t \sin \theta
\end{equation}

\begin{equation}
     \cos \theta + \sin t \sin \theta - \cos t = 0
\end{equation}

\subsection{exp of x}

\begin{equation}
    \frac{d(e^t)}{dt} = \cos \theta + e^t \sin \theta
\end{equation}

\begin{equation}
     \cos \theta + e^t \sin \theta - e^t = 0
\end{equation}

\section{Streamlization}\label{sec:soada}

\section{An embedding of surreal number}\label{sec:aeosn}

\section{A geometry of computation}\label{sec:gioc}

\bibliographystyle{unsrt}
\bibliography{references}

\appendix

\section{Derivation of the flow equation}

\begin{equation}
    \frac{dx}{dt} = \cos \theta + x \sin \theta
\end{equation}

We can get the solution by steps:

\begin{equation}
    \frac{dx}{\cos \theta + x \sin \theta} = dt
\end{equation}

\begin{equation}
    \frac{1}{\sin \theta} \frac{d(\cos \theta + x \sin \theta)}{\cos \theta + x \sin \theta} = dt
\end{equation}

\begin{equation}
    \frac{1}{\sin \theta} ln(\cos \theta + x \sin \theta) = t + C
\end{equation}

\begin{equation}
    \cos \theta + x \sin \theta = C e^{t \sin \theta}
\end{equation}

\begin{equation}
    \cos \theta + x_0 \sin \theta = C
\end{equation}

\begin{equation}
   x = \frac{\cos \theta + x_0 \sin \theta}{\sin \theta} e^{t \sin \theta} - \cot \theta
\end{equation}

\begin{equation}
   x = (\cot \theta + x_0) e^{t \sin \theta} - \cot \theta
\end{equation}

\begin{equation}
   x =  x_0 e^{t \sin \theta} + (e^{t \sin \theta} - 1) \cot \theta
\end{equation}

\begin{equation}
   x =  x_0 e^{t \sin \theta} + [1 + t \sin \theta + \frac{1}{2!} t\sin^2 \theta  + \frac{1}{3!} t \sin^3 \theta + ... - 1] \cot \theta
\end{equation}

\begin{equation}
   x =  x_0 e^{t \sin \theta} + t \cos \theta + \sin \theta \cos \theta (\frac{t^2}{2!} + \frac{t^3}{3!} \sin \theta + \frac{t^4}{4!} \sin^2 \theta + ...)
\end{equation}

\begin{equation}
   x =  x_0 e^{t \sin \theta} + t \cos \theta + \frac{1}{2} \sin 2\theta (\frac{t^2}{2!} + \frac{t^3}{3!} \sin \theta + \frac{t^4}{4!} \sin^2 \theta + ...)
\end{equation}

\begin{equation}
   x =  x_0 e^{t \sin \theta} + t \cos \theta + \frac{1}{2} \sin 2\theta (\frac{t^2}{2!} + \frac{t^3}{3!} \sin \theta + \frac{t^4}{4!} (\frac{1 - \cos 2\theta}{2}) + ...)
\end{equation}

When $\theta = \frac{k \pi}{2}, k = 0, 1, 2, 3...$, we have

\begin{equation}
    x = x_0 e^{t \sin \theta} + t \cos \theta
\end{equation}

rewrite it:

\begin{equation}
   x =  x_0 e^{t \sin \theta} + t \cos \theta + \frac{1}{2} \sin 2\theta \Psi(t)
\end{equation}

\begin{equation}
   \Psi(t) =  \frac{t^2}{2!} + \frac{t^3}{3!} \sin \theta + \frac{t^4}{4!} (\frac{1 - \cos 2\theta}{2}) + \frac{t^5}{5!} (\frac{3 \sin \theta - \sin 3\theta}{4}) + \frac{t^6}{6!} (\frac{3 - 4 \cos 2\theta - \cos 4\theta}{8}) + ...
\end{equation}

\begin{equation}
   \Psi(t) = c_0 + s_1 \sin \theta + c_2 \cos 2\theta + s_3 \sin 3\theta + c_4 \cos 4\theta + s_5 \sin 5\theta + c_6 \cos 6\theta + ...
\end{equation}

\begin{equation}
   c_0 = \frac{t^2}{2!} + \frac{1}{2}\frac{t^4}{4!} + \frac{3}{8} \frac{t^6}{6!} + \frac{5}{16} \frac{t^8}{8!} + \frac{35}{128} \frac{t^{10}}{10!}  ...
\end{equation}

\begin{equation}
   s_1 = \frac{t^3}{3!} + \frac{3}{4}\frac{t^5}{5!} + \frac{5}{8} \frac{t^7}{7!} + \frac{35}{64} \frac{t^9}{9!} + \frac{63}{128} \frac{t^{11}}{11!}  ...
\end{equation}

\begin{equation}
   c_2 = \frac{t^3}{3!} + \frac{3}{4}\frac{t^5}{5!} + \frac{5}{8} \frac{t^7}{7!} + \frac{35}{64} \frac{t^9}{9!} + \frac{63}{128} \frac{t^{11}}{11!}  ...
\end{equation}

\begin{equation}
   s_3 = \frac{t^3}{3!} + \frac{3}{4}\frac{t^5}{5!} + \frac{5}{8} \frac{t^7}{7!} + \frac{35}{64} \frac{t^9}{9!} + \frac{63}{128} \frac{t^{11}}{11!}  ...
\end{equation}

\begin{equation}
   c_4 = \frac{t^3}{3!} + \frac{3}{4}\frac{t^5}{5!} + \frac{5}{8} \frac{t^7}{7!} + \frac{35}{64} \frac{t^9}{9!} + \frac{63}{128} \frac{t^{11}}{11!}  ...
\end{equation}

\begin{equation}
   s_5 = \frac{t^3}{3!} + \frac{3}{4}\frac{t^5}{5!} + \frac{5}{8} \frac{t^7}{7!} + \frac{35}{64} \frac{t^9}{9!} + \frac{63}{128} \frac{t^{11}}{11!}  ...
\end{equation}

\end{document}
