\documentclass{article}

\usepackage{arxiv}

\usepackage[utf8]{inputenc} % allow utf-8 input
\usepackage[T1]{fontenc}    % use 8-bit T1 fonts
\usepackage{hyperref}       % hyperlinks
\usepackage{url}            % simple URL typesetting
\usepackage{booktabs}       % professional-quality tables
\usepackage[english]{babel}
\usepackage{amsthm}
\usepackage{amsfonts}       % blackboard math symbols
\usepackage{nicefrac}       % compact symbols for 1/2, etc.
\usepackage{microtype}      % microtypography

\usepackage{graphicx}
\usepackage{amsmath}

\newtheorem{definition}{Definition}
\newtheorem{lemma}{Lemma}
\newtheorem{proposition}{Proposition}
\newtheorem{program}{Program}
\newtheorem{convention}{Convention}
\newtheorem{theorem}{Theorem}

\usepackage{stmaryrd}
\usepackage{mathtools}

\title{Suan: towards a geometry of computation}

%\date{September 9, 1985}	% Here you can change the date presented in the paper title
%\date{} 					% Or removing it

\author{
  Mingli~Yuan \\
  AI Lab \\
  ColorfulClouds Tech.\\
  Beijing, 100083 \\
  \texttt{mingli.yuan@caiyunapp.com} \\
  %% examples of more authors
  %% \AND
  %% Coauthor \\
  %% Affiliation \\
  %% Address \\
  %% \texttt{email} \\
  %% \And
  %% Coauthor \\
  %% Affiliation \\
  %% Address \\
  %% \texttt{email} \\
  %% \And
  %% Coauthor \\
  %% Affiliation \\
  %% Address \\
  %% \texttt{email} \\
}

% Uncomment to remove the date
%\date{}

% Uncomment to override  the `A preprint' in the header
%\renewcommand{\headeright}{Technical Report}
%\renewcommand{\undertitle}{Technical Report}

\begin{document}
\maketitle

\begin{abstract}
    Begin from an embedding of addition-multiplication computation into hyperbolic surface, we define a local polar
    coordinate on hyperbolic surface and then derive a flow equation, after studying several examples, we proof the
    consistency among the discrete constructions, infinitesimal calculation and the hyperbolic geometrical structures, and
    these lead us to a new mathematical system $\mathfrak{S}$.
\end{abstract}

\keywords{hyperbolic surface \and addition-multiplication tree \and infinitesimal calculation \and Suan system \and surreal number \and geometry of computation}

\setcounter{tocdepth}{2}
\tableofcontents

\section{Some discrete constructions on addition-multiplication expressions}\label{sec:discrete}

\subsection{Addition-multiplication tree, simplified postfix notation}\label{sec:amt}

Traditionally, we can express the most frequently-used mathematical functions by Taylor series,
so we can carry the computation by applying a serial of addition and multiplication on some numbers.
Any addition-multiplication expression can be formed into a binary tree, while pure numbers are leaves,
and addition or multiplication operators are at the branches. In this article, we study general arithmetical
expressions, or addition-multiplication trees as we named; and for a specific kind of addition-multiplication tree
that holds a fixed summand $1$ or a fixed multiplier $e$, and we name them as fixed addition-multiplication tree.

As we show later, all the fixed addition-multiplication trees can be formed into a background space,
where all the addition-multiplication trees can be appropriately representated in this background space.

\begin{definition}
An addition-multiplication tree is an arithmetical expression which owns arbitrary valid summands or arbitrary valid multipliers.
\end{definition}

\begin{definition}
A fixed addition-multiplication tree is an arithmetical expression which owns fixed summands $1, -1$
or fixed multipliers $e, e^{-1}$.
\end{definition}

Here is some examples of fixed addition-multiplication trees
\begin{itemize}
    \item $1 + 1 + 1 + 1$
    \item $n$ as $1 + 1 + ... + 1$
    \item $e^n$ as $e e ... e$
    \item $(e - 1) / e $
    \item the values defined by the iteration $x_{n+1} = e (x_n + 1), x_0 = 0$
\end{itemize}

We can express fixed addition-multiplication trees in a simplified postfix notion other than the normal infix notation.

\begin{definition}
    The grammar of the simplified postfix notion
    \begin{itemize}
    \item $\langle Expression \rangle \coloneqq \langle Operand \rangle \langle Operators \rangle$
    \item $\langle Operand \rangle \coloneqq 0$
    \item $\langle Operators \rangle \coloneqq \langle Operator \rangle \langle Operators \rangle$
    \item $\langle Operator \rangle \coloneqq \epsilon | + | - | * | /$
    \end{itemize}
\end{definition}

And we can define the semantics of the simplified postfix expression layer by layer as below

\begin{definition}
    The semantics of the simplified postfix notion
    \begin{itemize}
    \item $\llbracket\langle Expression \rangle\rrbracket = \llbracket\langle Operators \rangle\rrbracket (\llbracket\langle Operand \rangle\rrbracket)$
    \item $\llbracket\langle Operand \rangle\rrbracket = 0$
    \item $\llbracket\langle Operators \rangle\rrbracket =  \llbracket\langle Operators \rangle\rrbracket \circ \llbracket\langle Operator \rangle\rrbracket$
    \item $\llbracket\langle Operator \rangle\rrbracket = \llbracket\epsilon\rrbracket | \llbracket+\rrbracket | \llbracket-\rrbracket | \llbracket*\rrbracket | \llbracket/\rrbracket$
    \item $\llbracket\epsilon\rrbracket = id$
    \item $\llbracket+\rrbracket = incr$
    \item $\llbracket-\rrbracket = decr$
    \item $\llbracket*\rrbracket = expn$
    \item $\llbracket/\rrbracket = shrn$
    \item $id(x) = x$
    \item $incr(x) = x + 1$
    \item $decr(x) = x - 1$
    \item $expn(x) = e x$
    \item $shrn(x) = \frac{x}{e}$
    \end{itemize}
\end{definition}

So we rewrite the above examples in the postfix notion correspondingly

\begin{itemize}
    \item $0++++$
    \item $0++++...+++$: totally n plus sign are in the string
    \item $0+***...***$: totally n multiply sign are in the string
    \item $0+*-/$
    \item $0+*+*+*.../$
\end{itemize}

Let us introduce the inverse notion

\begin{convention}
    The inverse notion
    \begin{itemize}
        \item $+^{-1} = -$
        \item $-^{-1} = +$
        \item $*^{-1} = /$
        \item $/^{-1} = *$
    \end{itemize}
\end{convention}

In fact, with a minor change on the grammar, we can change the initial operand to be any number.
Assume we have one expression and the operand is $\alpha$, the operations are $a_1 a_2 ... a_{n-1} a_n$, and the
result of the expression is $\beta$, hence we have $\beta = \alpha a_1 a_2 ... a_{n-1} a_n$

\begin{lemma}
\label{lemma:inverse}
If we have
$$\beta = \alpha a_1 a_2 ... a_{n-1} a_n$$
then
$$\alpha = \beta a_n^{-1} a_{n-1}^{-1} ... a_2^{-1} a_1^{-1}$$
\end{lemma}

\begin{proof}
Notice that $+, -, *, /$ are all bijections, we know that $\alpha a_1 a_2 ... a_{n-1} a_n$ is a composition of functions

$$\beta = a_n( a_{n-1}( ... a_2( a_1(\alpha) ) ... ) )$$

Considering the inverse of a composition, we have

$$\alpha = a_1^{-1}( a_2^{-1}( ... a_{n-1}^{-1}( a_n^{-1}(\beta) ) ... ) )$$

Or under the simplified postfix notion

$$\alpha = \beta a_n^{-1} a_{n-1}^{-1} ... a_2^{-1} a_1^{-1}$$

\qedhere

\end{proof}

\newpage

\section{Some geometric constructions on hyperbolic surface}\label{sec:geometry}

\subsection{Order-4 apeirogonal tiling}\label{sec:o4ataamt}

\newpage

\section{An embeding from generated expressions to hyperbolic surface}\label{sec:embeding}

\newpage

\section{A kind of new infinitesimal calculation}\label{sec:akonic}

\subsection{Local polar coordinate and flow equation}\label{sec:lpcafe}

Given any point $x_0$ on $\mathfrak{S}$, an additional axis $A$ pass $x_0$, then for any nearby point $x_1$, two
horocycle $B$ and $\bar{B}$ exists to connect with $x_0$ and $x_1$, so we can consider a local polar coordinate
$\mathfrak{C}_{x_0}$, in which $x_1$ decided by:
\begin{itemize}
    \item the angle $\theta$ between $A$ and $B$
    \item the length $\rho$ of arc between $x_0$ and $x_1$ along $B$
\end{itemize}

$\mathfrak{C}_{x_0}$ has a counterpart $\bar{\mathfrak{C}_{x_0}}$, in which $x_1$ decided by:
\begin{itemize}
    \item the angle $\bar{\theta}$ between $A$ and $\bar{B}$
    \item the length $\bar{\rho}$ of arc between $x_0$ and $x_1$ along $\bar{B}$
\end{itemize}

Question: what dose the two horocycles mean under the computation perspective? how dose the two horocycles contact?

Now we study infinitesimal calculation and horocycle flow on $\mathfrak{S}$. By the perspective of
addition-multiplication computation, under a local coordinate $\mathfrak{C}_{x_0}$,
consider a point moving from $x_0$ and towards $x_{\delta}$ with the angle $\theta$ and a small step $\epsilon$, we have

\begin{equation}
    x_{\delta} = (x_0 + \epsilon \cos \theta)(1 + \epsilon \sin \theta)
\end{equation}

or

\begin{equation}
    x_{\delta} = x_0 (1 + \epsilon \sin \theta) + \epsilon \cos \theta
\end{equation}

And both can be simplified to

\begin{equation}
    x_{\delta} = x_0 + \epsilon (x_0 \sin \theta + \cos \theta)
\end{equation}

So we have

\begin{equation}
    \frac{1}{\delta} (x_{\delta} - x_0) = \frac{\epsilon}{\delta} (\cos \theta + x_0 \sin \theta)
\end{equation}

We denote the left side of the above equation as $dx / dt$ when $\delta$ and $\epsilon$ approach 0.

\begin{equation}
    \frac{dx}{dt} = u (\cos \theta + x \sin \theta)
    \label{eqn:flow}
\end{equation}

where $u$ is the moving speed.

\subsection{background defined by flow equation}\label{sec:}

By assuming $u = 1$ and costant angle $\theta$, we have a background which can be descibed by a simpler formula,

\begin{equation}
    \frac{dx}{dt} = \cos \theta + x \sin \theta
\end{equation}

Then we have the solution:

\begin{equation}
   x = x_0 e^{t \sin \theta} + (e^{t \sin \theta} - 1) \cot \theta
\end{equation}

And then expand to

\begin{equation}
   x =  x_0 e^{t \sin \theta} + [1 + t \sin \theta + \frac{1}{2!} t\sin^2 \theta  + \frac{1}{3!} t \sin^3 \theta + ... - 1] \cot \theta
\end{equation}

Or

\begin{equation}
   x =  x_0 e^{t \sin \theta} + t \cos \theta + \frac{1}{2} \sin 2\theta (\frac{t^2}{2!} + \frac{t^3}{3!} \sin \theta + \frac{t^4}{4!} \sin^2 \theta + ...)
\end{equation}

\newpage

\subsection{Constant speed flow}\label{sec:csflow}

\subsubsection{$\sin t$}

\begin{equation}
    \frac{dx}{dt} = u (\cos \theta + x \sin \theta)
\end{equation}

\begin{equation}
    \frac{d(\sin t)}{dt} = u(\cos \theta + \sin t \sin \theta)
\end{equation}

\begin{equation}
    u(\cos \theta + \sin t \sin \theta) - \cos t = 0
\end{equation}

\begin{equation}
    u = \frac{\cos t}{\cos \theta + \sin t \sin \theta}
\end{equation}


\begin{figure}[ht]
\centering
\includegraphics[width=5.5in]{plot/sine3d.png}
\caption{3d view of velocity $u(t, \theta)$ for process $\sin t$}
\end{figure}

\begin{figure}[ht]
\centering
\includegraphics[width=5.5in]{plot/sine2d.png}
\caption{contour of velocity $u(t, \theta)$ for process $\sin t$}
\end{figure}

\subsection{Constant angle flow}

We guess constant angle flow is horocycle flow and there is a sign.

\subsubsection{Two examples of constant angle flow}

Assuming $\theta = \pi / 4$, we can conclude that

\begin{equation}
    \frac{dx}{dt} = \frac{\sqrt{2}}{2} (1 + x)
\end{equation}

\begin{equation}
    x = c e^{\frac{t}{\sqrt{2}}} - 1
\end{equation}

Then, we have below formula

\begin{equation}
    x = (x_0 + 1)e^{\frac{t}{\sqrt{2}}} - 1
\end{equation}

And a recurrence relation with $t_n = \sqrt{2} n$

\begin{equation}
    x_{n+1} = (x_n + 1) e - 1, n = 0 ... k ...
\end{equation}

or in simplified Polish notation

\begin{equation}
    x_{n+1} = [+*-] x_n, n = 0 ... k ...
\end{equation}

Geometricly, this recurrence relation defines a horecycle on $H^2$.

Assuming $\theta = \pi / 6$, we can conclude that

\begin{equation}
    \frac{dx}{dt} = \frac{1}{2} (\sqrt{3} x + 1)
\end{equation}

\begin{equation}
    x = c e^{\frac{\sqrt{3}}{2}t} - \frac{\sqrt{3}}{3}
\end{equation}

Then, we have below formula

\begin{equation}
    x = (x_0 + \frac{\sqrt{3}}{3})e^{\frac{\sqrt{3}}{2}t} - \frac{\sqrt{3}}{3}
\end{equation}

And a recurrence relation with $t_n = \frac{2\sqrt{3}}{3} n$

\begin{equation}
    x_{n+1} = (x_n + \frac{\sqrt{3}}{3}) e - \frac{\sqrt{3}}{3}, n = 0 ... k ...
\end{equation}

\subsubsection{Constant angle flow theorem}

\begin{theorem}
\label{l1}
A recurrence relation

\begin{equation}
    x_{n+1} = (x_n + \tan \theta) e - \tan \theta, n = 0 ... k ...
\end{equation}

is hold for a flow with constant angle $\theta$ and constant speed $\csc \theta$

\end{theorem}

\begin{proof}

We follow below solution of \eqref{eqn:flow}

\begin{equation}
   x = (\cot \theta + x_0) e^{t \sin \theta} - \cot \theta
\end{equation}

and re-write it as

\begin{equation}
   x = (\cot \theta + x_0) e^{t \sin \theta} - \cot \theta
\end{equation}

\end{proof}

\newpage

\section{The consistency problem}\label{sec:consistency}

The consistency of addition-multiplication tree, flow equation and hyperbolic geometrical structures

\subsection{A non-trival example of the consistency}

\subsection{How to formulate the consistency}

\subsection{A proof of the consistency}

\subsection{A new mathematical system $\mathfrak{S}$}

impact on learnability, predicability, efficency(time complexity, space complexity), energy cost

\newpage

\section{An embedding of surreal number}\label{sec:aeosn}

\section{A geometry of computation}\label{sec:gioc}

\newpage
\bibliographystyle{unsrt}
\bibliography{references}

\newpage
\appendix

\section{Derivation of the flow equation}

\begin{equation}
    \frac{dx}{dt} = \cos \theta + x \sin \theta
\end{equation}

We can get the solution by steps:

\begin{equation}
    \frac{dx}{\cos \theta + x \sin \theta} = dt
\end{equation}

\begin{equation}
    \frac{1}{\sin \theta} \frac{d(\cos \theta + x \sin \theta)}{\cos \theta + x \sin \theta} = dt
\end{equation}

\begin{equation}
    \frac{1}{\sin \theta} ln(\cos \theta + x \sin \theta) = t + C
\end{equation}

\begin{equation}
    \cos \theta + x \sin \theta = C e^{t \sin \theta}
\end{equation}

\begin{equation}
    \cos \theta + x_0 \sin \theta = C
\end{equation}

\begin{equation}
   x = \frac{\cos \theta + x_0 \sin \theta}{\sin \theta} e^{t \sin \theta} - \cot \theta
\end{equation}

\begin{equation}
   x = (\cot \theta + x_0) e^{t \sin \theta} - \cot \theta
\end{equation}

\begin{equation}
   x =  x_0 e^{t \sin \theta} + (e^{t \sin \theta} - 1) \cot \theta
\end{equation}

\begin{equation}
   x =  x_0 e^{t \sin \theta} + [1 + t \sin \theta + \frac{1}{2!} t\sin^2 \theta  + \frac{1}{3!} t \sin^3 \theta + ... - 1] \cot \theta
\end{equation}

\begin{equation}
   x =  x_0 e^{t \sin \theta} + t \cos \theta + \sin \theta \cos \theta (\frac{t^2}{2!} + \frac{t^3}{3!} \sin \theta + \frac{t^4}{4!} \sin^2 \theta + ...)
\end{equation}

\begin{equation}
   x =  x_0 e^{t \sin \theta} + t \cos \theta + \frac{1}{2} \sin 2\theta (\frac{t^2}{2!} + \frac{t^3}{3!} \sin \theta + \frac{t^4}{4!} \sin^2 \theta + ...)
\end{equation}

\begin{equation}
   x =  x_0 e^{t \sin \theta} + t \cos \theta + \frac{1}{2} \sin 2\theta (\frac{t^2}{2!} + \frac{t^3}{3!} \sin \theta + \frac{t^4}{4!} (\frac{1 - \cos 2\theta}{2}) + ...)
\end{equation}

When $\theta = \frac{k \pi}{2}, k = 0, 1, 2, 3...$, we have

\begin{equation}
    x = x_0 e^{t \sin \theta} + t \cos \theta
\end{equation}

rewrite it:

\begin{equation}
   x =  x_0 e^{t \sin \theta} + t \cos \theta + \frac{1}{2} \sin 2\theta \Psi(t)
\end{equation}

\begin{equation}
   \Psi(t) =  \frac{t^2}{2!} + \frac{t^3}{3!} \sin \theta + \frac{t^4}{4!} (\frac{1 - \cos 2\theta}{2}) + \frac{t^5}{5!} (\frac{3 \sin \theta - \sin 3\theta}{4}) + \frac{t^6}{6!} (\frac{3 - 4 \cos 2\theta - \cos 4\theta}{8}) + ...
\end{equation}

\begin{equation}
   \Psi(t) = c_0 + s_1 \sin \theta + c_2 \cos 2\theta + s_3 \sin 3\theta + c_4 \cos 4\theta + s_5 \sin 5\theta + c_6 \cos 6\theta + ...
\end{equation}

\begin{equation}
   c_0 = \frac{t^2}{2!} + \frac{1}{2}\frac{t^4}{4!} + \frac{3}{8} \frac{t^6}{6!} + \frac{5}{16} \frac{t^8}{8!} + \frac{35}{128} \frac{t^{10}}{10!}  ...
\end{equation}

\begin{equation}
   s_1 = \frac{t^3}{3!} + \frac{3}{4}\frac{t^5}{5!} + \frac{5}{8} \frac{t^7}{7!} + \frac{35}{64} \frac{t^9}{9!} + \frac{63}{128} \frac{t^{11}}{11!}  ...
\end{equation}

\begin{equation}
   c_2 = \frac{t^3}{3!} + \frac{3}{4}\frac{t^5}{5!} + \frac{5}{8} \frac{t^7}{7!} + \frac{35}{64} \frac{t^9}{9!} + \frac{63}{128} \frac{t^{11}}{11!}  ...
\end{equation}

\begin{equation}
   s_3 = \frac{t^3}{3!} + \frac{3}{4}\frac{t^5}{5!} + \frac{5}{8} \frac{t^7}{7!} + \frac{35}{64} \frac{t^9}{9!} + \frac{63}{128} \frac{t^{11}}{11!}  ...
\end{equation}

\begin{equation}
   c_4 = \frac{t^3}{3!} + \frac{3}{4}\frac{t^5}{5!} + \frac{5}{8} \frac{t^7}{7!} + \frac{35}{64} \frac{t^9}{9!} + \frac{63}{128} \frac{t^{11}}{11!}  ...
\end{equation}

\begin{equation}
   s_5 = \frac{t^3}{3!} + \frac{3}{4}\frac{t^5}{5!} + \frac{5}{8} \frac{t^7}{7!} + \frac{35}{64} \frac{t^9}{9!} + \frac{63}{128} \frac{t^{11}}{11!}  ...
\end{equation}

\begin{equation}
    {\left(C - \int \cos\left(\theta\left(t\right)\right) e^{\left(\int \sin\left(\theta\left(t\right)\right)\,{d t}\right)}\,{d t}\right)} e^{\left(-\int \sin\left(\theta\left(t\right)\right)\,{d t}\right)}
\end{equation}


\subsection{Other examples of driving charts}\label{sec:meodc}

\subsubsection{a constant k}

\begin{equation}
    0 = \cos \theta + k \sin \theta
\end{equation}

\begin{equation}
    \theta = n\pi - \arctan \frac{1}{k}
\end{equation}

\subsubsection{identity}

\begin{equation}
    \cos \theta + t \sin \theta - 1 = 0
\end{equation}

\begin{equation}
    \theta = 2k\pi
\end{equation}
\begin{equation}
    \theta = 2k\pi + \tan^{-1} t
\end{equation}

\subsubsection{inverse of t}

\begin{equation}
    \frac{d(t^{-1})}{dt} = \cos \theta + t^{-1} \sin \theta
\end{equation}

\begin{equation}
    t^2 \cos \theta + t \sin \theta + 1 = 0
\end{equation}

\subsubsection{square of t}

\begin{equation}
    \frac{d(t^2)}{dt} = \cos \theta + t^2 \sin \theta
\end{equation}

\begin{equation}
    \cos \theta + t^2 \sin \theta - 2 t = 0
\end{equation}

\subsubsection{cube of t}

\begin{equation}
    \frac{d(t^3)}{dt} = \cos \theta + t^3 \sin \theta
\end{equation}

\begin{equation}
    \cos \theta + t^3 \sin \theta - 3 t^2 = 0
\end{equation}

\subsubsection{sin of t}

\begin{equation}
    \frac{d(\sin t)}{dt} = \cos \theta + \sin t \sin \theta
\end{equation}

\begin{equation}
     \cos \theta + \sin t \sin \theta - \cos t = 0
\end{equation}

\subsubsection{exp of t}

\begin{equation}
    \frac{d(e^t)}{dt} = \cos \theta + e^t \sin \theta
\end{equation}

\begin{equation}
     \cos \theta + e^t \sin \theta - e^t = 0
\end{equation}

\end{document}
