\documentclass{article}

\usepackage{arxiv}

\usepackage[utf8]{inputenc} % allow utf-8 input
\usepackage[T1]{fontenc}    % use 8-bit T1 fonts
\usepackage{hyperref}       % hyperlinks
\usepackage{url}            % simple URL typesetting
\usepackage{booktabs}       % professional-quality tables
\usepackage[english]{babel}
\usepackage{amsthm}
\usepackage{amsfonts}       % blackboard math symbols
\usepackage{nicefrac}       % compact symbols for 1/2, etc.
\usepackage{microtype}      % microtypography

\usepackage{graphicx}
\usepackage{amsmath}

\newtheorem{definition}{Definition}
\newtheorem{lemma}{Lemma}
\newtheorem{proposition}{Proposition}
\newtheorem{corollary}{Corollary}
\newtheorem{program}{Program}
\newtheorem{convention}{Convention}
\newtheorem{theorem}{Theorem}

\usepackage{stmaryrd}
\usepackage{mathtools}

\usepackage{tikz}

\DeclareMathSymbol{\mathinvertedexclamationmark}{\mathclose}{operators}{'074}
\DeclareMathSymbol{\mathexclamationmark}{\mathclose}{operators}{'041}
\makeatletter
\newcommand{\raisedmathinvertedexclamationmark}{%
  \mathclose{\mathpalette\raised@mathinvertedexclamationmark\relax}%
}
\newcommand{\raised@mathinvertedexclamationmark}[2]{%
  \raisebox{\depth}{$\m@th#1\mathinvertedexclamationmark$}%
}
\begingroup\lccode`~=`! \lowercase{\endgroup
  \def~}{\@ifnextchar`{\raisedmathinvertedexclamationmark\@gobble}{\mathexclamationmark}}
\mathcode`!="8000
\makeatother


\title{Suan: A geometry of computation}

%\date{September 9, 1985}	% Here you can change the date presented in the paper title
%\date{} 					% Or removing it

\author{
  Mingli~Yuan \\
  AI Lab \\
  ColorfulClouds Tech.\\
  Beijing, 100083 \\
  \texttt{mingli.yuan@caiyunapp.com} \\
  %% examples of more authors
  %% \AND
  %% Coauthor \\
  %% Affiliation \\
  %% Address \\
  %% \texttt{email} \\
  %% \And
  %% Coauthor \\
  %% Affiliation \\
  %% Address \\
  %% \texttt{email} \\
  %% \And
  %% Coauthor \\
  %% Affiliation \\
  %% Address \\
  %% \texttt{email} \\
}

% Uncomment to remove the date
%\date{}

% Uncomment to override  the `A preprint' in the header
%\renewcommand{\headeright}{Technical Report}
%\renewcommand{\undertitle}{Technical Report}

\begin{document}
\maketitle

\begin{abstract}
    After assigning all arithmetical expressions into hyperbolic surface, we define a local polar coordinate
    and then derive a flow equation, we show the consistency among the discrete constructions, infinitesimal calculation
    and the hyperbolic geometrical structures by several interesting cases.
    These lead to the foundation problem which is still unsolved, and we propose several ways to attack it.
    If the foundation problem is true, we can define a new mathematical system Suan or $\mathfrak{S}$, and reformulate other
    traditional mathematical object in a new way. Several important question arosed by $\mathfrak{S}$ are proposed,
    and further study directions are also included.
    We introduce a new construction can expand calculus theory to heterogeneous accumulation process.
\end{abstract}

\keywords{hyperbolic surface \and addition-multiplication tree \and infinitesimal calculation \and Suan system \and geometry of computation}

\setcounter{tocdepth}{2}
\tableofcontents

\section{Background and preliminaries}\label{sec:background}

\subsection{A beginning question}\label{sec:question}

It is a fundamental task to describe an evolving process in mathematic. After the introduction of concepts of variable and function,
Newton and Leibniz set up the area of calculus. It uses linearized small pieces of variation to reflect the changing rate and accumulate
these small pieces into a larger part, these are derivative and integral accordingly.

Back to basic arithmetical facts, as shown in some trivial examples, this kind of accumulative evolving process is happening like:

$$ 8 = 2 + 4 + 2 $$

reflects a changing sequence of $[2, 6, 8]$, or

$$ 8 = 1 \times 2 \times 4 $$

reflects a changing sequence of $[1, 2, 8]$.

In both cases, we use the same additive or multiplicative operators to apply them on different operands step by step,
so it is called a homogeneous accumulation process.

Alternatively, we have other cases like:

$$ 8 = (1 \times 2 + 1) \times 2 + 1 = (1 + 2) \times 2 + 2 = \ldots $$

In these cases, we combine both additive and multiplicative operators on a sequence of operands,
which is called a heterogeneous accumulation process.

Can we expand the calculus theory to those heterogeneous accumulation process?
That set the beginning question of our paper.

\subsection{Poincaré disk model}\label{sec:disk}

Poincaré disk model is a model of hyperbolic plane $H^2$.
Considering a unit disk $D$ on Euclid plane, $\partial D$ is the boundary and $D^o$ is the interior,


\subsection{Circles on hyperbolic plane}\label{sec:circle}


\subsection{Order-4 apeirogonal tiling}\label{sec:tiling}



\section{Arithmetical expressions}\label{sec:expressions}

We consider the arithmetical expression set $A$ which is freely generated from
\begin{itemize}
    \item initial operand: $0$
    \item operator: $\oplus_n: x \mapsto x + 2^{-n}$
    \item operator: $\ominus_n: x \mapsto x - 2^{-n}$
    \item operator: $\otimes_n: x \mapsto x \cdot e^{2^{-n}}$
    \item operator: $\oslash_n: x \mapsto x / e^{2^{-n}}$
\end{itemize}
where e is the Euler's number.

With $A_n$ defined as above, we have a chain of inclusion
\begin{lemma}
\label{lemma:chainofinclusion1}
We have
$$ A_0 \subset A_1 \subset A_2 \subset \ldots \subset A_n \subset \ldots $$
hold for all $n$.
\end{lemma}

\subsection{Literial structure}\label{sec:literial}


\subsection{Syntactical structure}\label{sec:syntactical}

In this section, we discus the conclusions that is drawn only from the syntactical side of the definition of $A_n$ and
do not borrow any semantic properties from the calculation result of the expressions.

The inverse notion can be introduced
\begin{itemize}
    \item $\oplus_n^{-1} = \ominus_n$
    \item $\ominus_n^{-1} = \oplus_n$
    \item $\otimes_n^{-1} = \oslash_n$
    \item $\oslash_n^{-1} = \otimes_n$
\end{itemize}

Suppose we have a operand $\alpha \in A_n $ and a series of operators $a_1, a_2, ... a_{n-1}, a_n$,
we introduce the path notion:
$$\alpha a_1 a_2 ... a_{n-1} a_n = a_n( a_{n-1}( ... a_2( a_1(\alpha) ) ... ) )$$

Then we have the calculation rule of inverse.

\begin{lemma}
\label{lemma:inverserule}
If we have
$$\beta = \alpha a_1 a_2 ... a_{n-1} a_n$$
then
$$\alpha = \beta a_n^{-1} a_{n-1}^{-1} ... a_2^{-1} a_1^{-1}$$
\end{lemma}

\begin{proof}
Notice that $\oplus_n, \ominus_n, \otimes_n, \oslash_n$ are all bijections, we know that $\alpha a_1 a_2 ... a_{n-1} a_n$ is a composition of functions

$$\beta = a_n( a_{n-1}( ... a_2( a_1(\alpha) ) ... ) )$$

Considering the inverse of a composition, we have

$$\alpha = a_1^{-1}( a_2^{-1}( ... a_{n-1}^{-1}( a_n^{-1}(\beta) ) ... ) )$$

Or under the path notion

$$\alpha = \beta a_n^{-1} a_{n-1}^{-1} ... a_2^{-1} a_1^{-1}$$

\qedhere
\end{proof}

After intruduce of a zero operator $\odot$ which is its own inverse:
\begin{itemize}
    \item $\odot: x \mapsto x$
    \item $\odot^{-1} = \odot$
\end{itemize}

Informally wee can divide a long calculation into pieces, or composite small pieces of calculations into a longer one,
for example

$$\gamma = 0 a_1 a_2 ... a_{n-1} a_n b_1 b2 ... b_{m-1} b_m$$

can be rewritten as

$$\gamma = 0 ((\odot a_1 a_2 ... a_{n-1} a_n) \circ (\odot b_1 b_2 ... b_{m-1} b_m))$$

And if we treat $\odot$ and $0$ as the same, and we define

$$\alpha = 0 a_1 a_2 ... a_{n-1} a_n$$
$$\beta = 0 b_1 b2 ... b_{m-1} b_m$$

We have

$$\gamma = \alpha \circ \beta$$

where $\alpha$, $\beta$ and $\gamma$ are all well defined in of $A_n$.

It is easy to see the algebraic structure $\mathcal{P}_n = (A_n, \circ)$ is a group, and we call it \emph{path group}.

\begin{figure}[ht]
\centering
\begin{tikzpicture}

\filldraw [black] (0,0) circle (2pt) node[align=center, below] {0};
\filldraw [black] (4,1) circle (2pt) node[align=center, below] {$\alpha$};
\filldraw [black] (2,3) circle (2pt) node[align=center, below] {$\beta$};
\filldraw [black] (5,3) circle (2pt) node[align=center, below] {$\gamma$};
\filldraw [black] (-2,2) circle (2pt) node[align=center, below] {$\psi$};
\filldraw [black] (1,2) circle (2pt) node[align=center, below] {$\phi$};
\filldraw [gray] (3,-0.4) circle (0pt) node[align=center, below] {$x$};
\filldraw [gray] (3,1.8) circle (0pt) node[align=center, below] {$y$};
\filldraw [gray] (4.7,2) circle (0pt) node[align=center, below] {$z$};
\draw [gray] (0,0) to[out=300,in=240] (4,1);
\draw [gray] (4,1) to[out=120,in=300] (2,3);
\draw [gray] (4,1) to[out=60,in=240] (5,3);
\draw [gray] (0,0) to[out=120,in=300] (-2,2);
\draw [gray] (0,0) to[out=60,in=240] (1,2);
\filldraw [gray] (-1,0.8) circle (0pt) node[align=center, below] {$y$};
\filldraw [gray] (0.7,1) circle (0pt) node[align=center, below] {$z$};

\end{tikzpicture}
\caption{Factorization by greatest common divisor}
\end{figure}

The generating order is a partial order over $A_n$, we assume the later generated one is greater. By using the generating order,
we can define the concept of greatest common divisor: for any $\beta, \gamma \in A_n$, the greatest common divisor
$gcd(\beta, \gamma)$ is the greatest common ancestor in the generating tree.
So coprime of $\psi, \phi \in A_n$ means $gcd(\psi, \phi) = 0$

\begin{lemma}
\label{lemma:coprimes}
When $\alpha = gcd(\beta, \gamma)$, and $\beta = \alpha \circ \psi$, $\gamma = \alpha \circ \phi$ hold, then $\psi$ and $\phi$ are coprimes.
\end{lemma}

\subsection{Semantical structure}\label{sec:semantical}

In this section, we study the conclusions related with the calculation results for each expression of $A_n$.

We define $S_n = \{2^{-n}k | k \in Z\}$, it is easy to know $S_n$ is a commutative ring, and we also have a chain of inclusion

\begin{lemma}
\label{lemma:chainofinclusion2}
We have
$$ S_0 \subset S_1 \subset S_2 \subset \ldots \subset S_n \subset \ldots \subset Q$$
hold for all $n$, where $Q$ is the field of rational numbers.
\end{lemma}

\begin{lemma}
\label{lemma:arithmeticalalgebra}
All arithmetical expressions $A_n$ is an algebra over $S_n$, or $\mathcal{A}_n = (A_n, S_n, \cdot)$.
\end{lemma}

\begin{proof}
It is easy to verify the three axioms of an algebra is hold:
for any arithmetical expressions $x, y, z \in A_n$ and $a, b \in S_n$, we have
\begin{itemize}
    \item right distributivity: $(x + y) \cdot z = x \cdot z + y \cdot z$
    \item left distributivity: $z \cdot (x + y) = z \cdot x + z \cdot y$
    \item compatibility with scalars: $(ax) \cdot (by) = (ab) (x \cdot y)$
\end{itemize}
\qedhere
\end{proof}

\begin{lemma}
\label{lemma:chainofinclusion3}
We have
$$ \mathcal{A}_0 \subset \mathcal{A}_1 \subset \mathcal{A}_2 \subset \ldots \subset \mathcal{A}_n \subset \ldots $$
hold for all $n$.
\end{lemma}

An equivalent formulation of Lindemann-Weierstrass theorem is given as below

\begin{proposition}
\label{proposition:LindemannWeierstrass}
If $\alpha_1, \cdots, \alpha_n$ are distinct algebraic numbers, then $e^{\alpha_1}, \cdots, e^{\alpha_n}$ are linearly independent over the algebraic numbers.
\end{proposition}

This proposition gives the transcendental nature of $e$.

Now, we can study the structure of the values.
Due to Lindemann-Weierstrass theorem, we have all the values should take the form:
$$
a_1 e^{\alpha_1} + \cdots + a_m e^{\alpha_m}
$$
where $a_i \in S_n$ and $\alpha_i \in S_n$, $i \in Z$. And we denote

$$
N_n = \{ a_1 e^{\alpha_1} + \cdots + a_m e^{\alpha_m} | a_i \in S_n, \alpha_i \in S_n \}
$$

\begin{lemma}
\label{lemma:valuealgebra}
All values from $N_n$ is an algebra over $S_n$, or $\mathcal{N}_n = (N_n, S_n, \cdot)$.
\end{lemma}

\begin{proof}
We can verify $N_n$ is closed under $+$ and $\cdot$. Also the three axioms of an algebra is hold:
for any value $x, y, z \in N_n$ and $a, b \in S_n$, we have
\begin{itemize}
    \item right distributivity: $(x + y) \cdot z = x \cdot z + y \cdot z$
    \item left distributivity: $z \cdot (x + y) = z \cdot x + z \cdot y$
    \item compatibility with scalars: $(ax) \cdot (by) = (ab) (x \cdot y)$
\end{itemize}
\qedhere
\end{proof}

The result of each calculation is a surjection $V_n$ from $A_n$ to $N_n$
$$ V_n: A_n \to N_n $$

So, we can verify path group $\mathcal{P}_n$ is non-abelian by calculating the result at point $0$.

\begin{lemma}
\label{lemma:nonabelian}
The path group $\mathcal{P}_n = (A_n, \circ)$ is non-abelian
\end{lemma}

\begin{proof}
Notice that $(0 + 2^{-n}) \cdot e^{2^{-n}} \neq 0 \cdot e^{2^{-n}} + 2^{-n}$, and since $V_n$ is surjective, we know $0 \oplus_n \otimes_n \neq 0 \otimes_n \oplus_n$
\qedhere
\end{proof}

Let's consider an embedding $H_n$ from $N_n$ to $A_n$
$$ H_n: N_n \to A_n $$
which is given by below procedure:

For any
$$
a_1 e^{\alpha_1} + \cdots + a_m e^{\alpha_m} \in N_n
$$
Assume we had ordered the terms by ascending order of $\alpha_i$, i.e. we have
$$
\alpha_1  < \alpha_2 \cdots < \alpha_m
$$
and also
$$
a_1 \neq 0, a_2 \neq 0, \cdots a_m \neq 0
$$

We can rewrite it into another form by Horner method, we call this form as Horner norm

$$
b_m + e^{\beta_m} (b_{m-1} + e^{\beta_{m-1}} ( \cdots (b_1 + b_0 e^{\beta_1}) \cdots ))
$$

and then we can translate it into path notion equivalently:

$$
0 \oplus^{b_0} \otimes^{\beta_1} \oplus^{b_1} \cdots \otimes^{\beta_{m-1}} \oplus^{b_{m-1}} \otimes^{\beta_m} \oplus^{b_m}
$$
where we improve the notion as
\begin{itemize}
    \item operator: $\oplus^k: x \mapsto x + k$, hence $\oplus^k = \oplus_n^{k2^n}$
    \item operator: $\ominus^k: x \mapsto x - k$, hence $\ominus^k = \ominus_n^{k2^n}$
    \item operator: $\otimes^k: x \mapsto x \cdot e^k$, hence $\otimes^k = \otimes_n^{k2^n}$
    \item operator: $\oslash^k: x \mapsto x / e^k$, hence $\oslash^k = \oslash_n^{k2^n}$
\end{itemize}

We call this embedding $H_n$ as Horner embedding.

At the end of this section, we discuss some combinatorial properties of the Horner norm and the path notion.

\subsection{Homomorphisms}\label{sec:homomorphisms}

We can consider two subgroup $\mathcal{N}_n$ and $\mathcal{Q}_n$ of $\mathcal{P}_n$.

The subgroup $\mathcal{N}_n$ are all expressions freely generated from
\begin{itemize}
    \item initial operand: $0$
    \item operator: $\otimes_n: x \mapsto x \cdot e^{2^{-n}}$
    \item operator: $\oslash_n: x \mapsto x / e^{2^{-n}}$
\end{itemize}

The subgroup $\mathcal{Q}_n$ of $\mathcal{P}_n$ are all expressions freely generated from
\begin{itemize}
    \item initial operand: $0$
    \item operator $\oplus_n: x \mapsto x + 2^{-n}$
    \item operator: $\ominus_n: x \mapsto x - 2^{-n}$
\end{itemize}

\begin{figure}[ht]
\centering
\begin{tikzpicture}

\filldraw [black] (0,0) circle (2pt) node[align=center, below] {0};
\filldraw [black] (0,4) circle (2pt) node[align=center, right] {$\alpha = \beta^{-1} \circ \alpha \circ \beta$};
\filldraw [black] (2,1) circle (2pt) node[align=center, right] {$\beta$};
\filldraw [black] (-2, -1) circle (2pt) node[align=center, left] {$\beta^{-1}$};
\filldraw [black] (-2, 3) circle (2pt) node[align=center, left] {$\beta^{-1} \circ \alpha$};
\filldraw [gray] (-0.4, 2) circle (0pt) node[align=center, below] {$n$};
\filldraw [gray] (-2.4, 1) circle (0pt) node[align=center, below] {$n$};
\filldraw [gray] (1, 0.3) circle (0pt) node[align=center, below] {$x$};
\filldraw [gray] (-1, -0.5) circle (0pt) node[align=center, below] {$x^{-1}$};
\draw [gray] (0,0) to[out=90,in=270] (0,4);
\draw [gray] (0,0) to[out=330,in=180] (2,1);
\draw [gray] (0,0) to[out=150,in=0] (-2,-1);
\draw [gray] (-2,-1) to[out=90,in=270] (-2,3);
\draw [gray] (-2,3) to[out=330,in=180] (0,4);

\end{tikzpicture}
\caption{If $\mathcal{N}_n $ is normal}
\end{figure}

\begin{lemma}
\label{lemma:normalofn}
The subgroup $\mathcal{N}_n $ is a normal subgroup of $\mathcal{P}_n$.
\end{lemma}

\begin{proof}
?
\qedhere
\end{proof}

\begin{figure}[ht]
\centering
\begin{tikzpicture}

\filldraw [black] (0,0) circle (2pt) node[align=center, left] {0};
\filldraw [black] (4,0) circle (2pt) node[align=center, right] {$\alpha = \beta^{-1} \circ \alpha \circ \beta$};
\filldraw [black] (1,2) circle (2pt) node[align=center, below] {$\beta$};
\filldraw [black] (-1,-2) circle (2pt) node[align=center, below] {$\beta^{-1}$};
\filldraw [black] (3,-2) circle (2pt) node[align=center, below] {$\beta^{-1} \circ \alpha$};
\filldraw [gray] (2,-0.4) circle (0pt) node[align=center, below] {$q$};
\filldraw [gray] (1,-2.4) circle (0pt) node[align=center, below] {$q$};
\filldraw [gray] (0.7,1) circle (0pt) node[align=center, below] {$x$};
\filldraw [gray] (-0.1,-1) circle (0pt) node[align=center, below] {$x^{-1}$};
\draw [gray] (0,0) to[out=0,in=180] (4,0);
\draw [gray] (0,0) to[out=90,in=240] (1,2);
\draw [gray] (0,0) to[out=-90,in=60] (-1,-2);
\draw [gray] (-1,-2) to[out=0,in=180] (3,-2);
\draw [gray] (3,-2) to[out=90,in=240] (4,0);

\end{tikzpicture}
\caption{If $\mathcal{Q}_n $ is normal}
\end{figure}

\begin{lemma}
\label{lemma:normalofq}
The subgroup $\mathcal{Q}_n $ is a normal subgroup of $\mathcal{P}_n$.
\end{lemma}

\begin{proof}
?
\qedhere
\end{proof}

\subsection{About canonical forms}


\subsection{Properties of Horner norm}


\section{An embedding into hyperbolic surface}\label{sec:embeding}

\subsection{The embedding}

\subsection{The transfomation group}

\subsubsection{Literial structure}

\subsubsection{Syntactical structure}

\subsubsection{Semantical structure}






$A_0$ can be embedded into Hyperbolic plane $H_2$ as a regular tree which is an infinite tree of degree 4.

\begin{figure}[ht]
\centering
\begin{tikzpicture}
    \draw (0, 0) node[inner sep=0] {\includegraphics[width=6in]{images/t4096.png}};
    \draw (-4.40, +4.60) node[inner sep=1pt] (s) {$0 - 1$};
    \draw (-2.00, +2.80) node[inner sep=1pt] (z) {$0$};
    \draw (+2.90, +1.60) node[inner sep=1pt] (a) {$0 + 1$};
    \draw (+6.00, +1.60) node[inner sep=1pt] (aa) {$0 + 1 + 1$};
    \draw (+3.00, +5.00) node[inner sep=1pt] (am) {$(0 + 1) \cdot e$};
    \draw (+4.00, -2.00) node[inner sep=1pt] (ad) {$(0 + 1) / e$};
    \draw (-0.80, +6.20) node[inner sep=1pt] (m) {$0 \cdot e$};
    \draw (-4.90, +0.40) node[inner sep=1pt] (d) {$0 / e$};
\end{tikzpicture}
\caption{Embedding of the expressions}
\end{figure}

\subsection{On geometrical structure}\label{sec:geostructure}

\newpage

\section{A new kind of infinitesimal calculation}\label{sec:ankoic}

\subsection{Local polar coordinate}\label{sec:flpc}

\subsection{The flow equation}\label{sec:lpcafe}

Given any point $x_0$ on $\mathfrak{S}$, an additional axis $A$ pass $x_0$, then for any nearby point $x_1$, two
horocycle $B$ and $\bar{B}$ exists to connect with $x_0$ and $x_1$, so we can consider a local polar coordinate
$\mathfrak{C}_{x_0}$, in which $x_1$ decided by:
\begin{itemize}
    \item the angle $\theta$ between $A$ and $B$
    \item the length $\rho$ of arc between $x_0$ and $x_1$ along $B$
\end{itemize}

$\mathfrak{C}_{x_0}$ has a counterpart $\bar{\mathfrak{C}_{x_0}}$, in which $x_1$ decided by:
\begin{itemize}
    \item the angle $\bar{\theta}$ between $A$ and $\bar{B}$
    \item the length $\bar{\rho}$ of arc between $x_0$ and $x_1$ along $\bar{B}$
\end{itemize}

Question: what dose the two horocycles mean under the computation perspective? how dose the two horocycles contact?

Now we study infinitesimal calculation and horocycle flow on $\mathfrak{S}$. By the perspective of
addition-multiplication computation, under a local coordinate $\mathfrak{C}_{x_0}$,
consider a point moving from $x_0$ and towards $x_{\delta}$ with the angle $\theta$ and a small step $\epsilon$, we have

\begin{equation}
    x_{\delta} = (x_0 + \epsilon \cos \theta)(1 + \epsilon \sin \theta)
\end{equation}

or

\begin{equation}
    x_{\delta} = x_0 (1 + \epsilon \sin \theta) + \epsilon \cos \theta
\end{equation}

And both can be simplified to

\begin{equation}
    x_{\delta} = x_0 + \epsilon (x_0 \sin \theta + \cos \theta)
\end{equation}

So we have

\begin{equation}
    \frac{1}{\delta} (x_{\delta} - x_0) = \frac{\epsilon}{\delta} (\cos \theta + x_0 \sin \theta)
\end{equation}

We denote the left side of the above equation as $dx / dt$ when $\delta$ and $\epsilon$ approach 0.

\begin{equation}
    \frac{dx}{dt} = u (\cos \theta + x \sin \theta)
    \label{eqn:flow}
\end{equation}

where $u$ is the moving speed.

\subsubsection{background defined by flow equation}\label{sec:}

By assuming $u = 1$ and costant angle $\theta$, we have a background which can be descibed by a simpler formula,

\begin{equation}
    \frac{dx}{dt} = \cos \theta + x \sin \theta
\end{equation}

Then we have the solution:

\begin{equation}
   x = x_0 e^{t \sin \theta} + (e^{t \sin \theta} - 1) \cot \theta
\end{equation}

And then expand to

\begin{equation}
   x =  x_0 e^{t \sin \theta} + [1 + t \sin \theta + \frac{1}{2!} t\sin^2 \theta  + \frac{1}{3!} t \sin^3 \theta + ... - 1] \cot \theta
\end{equation}

Or

\begin{equation}
   x =  x_0 e^{t \sin \theta} + t \cos \theta + \frac{1}{2} \sin 2\theta (\frac{t^2}{2!} + \frac{t^3}{3!} \sin \theta + \frac{t^4}{4!} \sin^2 \theta + ...)
\end{equation}

\subsection{Example I: Steady flow}

We guess constant angle flow is horocycle flow and there is a sign.

Two examples of constant angle flow

Assuming $\theta = \pi / 4$, we can conclude that

\begin{equation}
    \frac{dx}{dt} = \frac{\sqrt{2}}{2} (1 + x)
\end{equation}

\begin{equation}
    x = c e^{\frac{t}{\sqrt{2}}} - 1
\end{equation}

Then, we have below formula

\begin{equation}
    x = (x_0 + 1)e^{\frac{t}{\sqrt{2}}} - 1
\end{equation}

And a recurrence relation with $t_n = \sqrt{2} n$

\begin{equation}
    x_{n+1} = (x_n + 1) e - 1, n = 0 ... k ...
\end{equation}

or in simplified Polish notation

\begin{equation}
    x_{n+1} = [+*-] x_n, n = 0 ... k ...
\end{equation}

Geometricly, this recurrence relation defines a horecycle on $H^2$.

Assuming $\theta = \pi / 6$, we can conclude that

\begin{equation}
    \frac{dx}{dt} = \frac{1}{2} (\sqrt{3} x + 1)
\end{equation}

\begin{equation}
    x = c e^{\frac{\sqrt{3}}{2}t} - \frac{\sqrt{3}}{3}
\end{equation}

Then, we have below formula

\begin{equation}
    x = (x_0 + \frac{\sqrt{3}}{3})e^{\frac{\sqrt{3}}{2}t} - \frac{\sqrt{3}}{3}
\end{equation}

And a recurrence relation with $t_n = \frac{2\sqrt{3}}{3} n$

\begin{equation}
    x_{n+1} = (x_n + \frac{\sqrt{3}}{3}) e - \frac{\sqrt{3}}{3}, n = 0 ... k ...
\end{equation}

Steady flow theorem

\begin{theorem}
\label{l1}
A recurrence relation

\begin{equation}
    x_{n+1} = (x_n + \tan \theta) e - \tan \theta, n = 0 ... k ...
\end{equation}

is hold for a flow with constant angle $\theta$ and constant speed $\csc \theta$

\end{theorem}

\begin{proof}

We follow below solution of \eqref{eqn:flow}

\begin{equation}
   x = (\cot \theta + x_0) e^{t \sin \theta} - \cot \theta
\end{equation}

and re-write it as

\begin{equation}
   x = (\cot \theta + x_0) e^{t \sin \theta} - \cot \theta
\end{equation}

\end{proof}

\subsection{Example II: Brachistochrone curve}\label{sec:brachistochrone}


\section{The foundation problem}


\section{Reformulating other mathematical objects by $\mathfrak{S}$}

\subsection{Power series}\label{sec:powerserials}

\subsection{Complex function}\label{sec:complexfunction}

holomorphic: the line integral around a closed path of a function that is holomorphic everywhere inside the area bounded by the closed path is always zero, as is stated by the Cauchy integral theorem.

The values of such a holomorphic function inside a disk can be computed by a path integral on the disk's boundary (as shown in Cauchy's integral formula).

Path integrals in the complex plane are often used to determine complicated real integrals, and here the theory of residues among others is applicable (see methods of contour integration).

A "pole" (or isolated singularity) of a function is a point where the function's value becomes unbounded, or "blows up".

If a function has such a pole, then one can compute the function's residue there, which can be used to compute path integrals involving the function; this is the content of the powerful residue theorem.

The remarkable behavior of holomorphic functions near essential singularities is described by Picard's Theorem.

Functions that have only poles but no essential singularities are called meromorphic.

Laurent series are the complex-valued equivalent to Taylor series, but can be used to study the behavior of functions near singularities through infinite sums of more well understood functions, such as polynomials.

\subsection{Conformal transformation}\label{sec:conformalmapping}

\section{New questions arosed by $\mathfrak{S}$}

\subsection{Shortcuts of computation path}\label{sec:shortcut}

\subsection{Complexification}\label{sec:complex}

\subsection{Lie group and category theory}\label{sec:categorytheory}

the duality of number and expressions in the language of category theory

\section{Further discussion}\label{sec:gioc}

\subsection{New insights on AD and NAS}\label{sec:computationalgraph}

\newpage
\bibliographystyle{unsrt}
\bibliography{references}

\newpage
\appendix

\section{Derivation of the flow equation}

\begin{equation}
    \frac{dx}{dt} = \cos \theta + x \sin \theta
\end{equation}

We can get the solution by steps:

\begin{equation}
    \frac{dx}{\cos \theta + x \sin \theta} = dt
\end{equation}

\begin{equation}
    \frac{1}{\sin \theta} \frac{d(\cos \theta + x \sin \theta)}{\cos \theta + x \sin \theta} = dt
\end{equation}

\begin{equation}
    \frac{1}{\sin \theta} ln(\cos \theta + x \sin \theta) = t + C
\end{equation}

\begin{equation}
    \cos \theta + x \sin \theta = C e^{t \sin \theta}
\end{equation}

\begin{equation}
    \cos \theta + x_0 \sin \theta = C
\end{equation}

\begin{equation}
   x = \frac{\cos \theta + x_0 \sin \theta}{\sin \theta} e^{t \sin \theta} - \cot \theta
\end{equation}

\begin{equation}
   x = (\cot \theta + x_0) e^{t \sin \theta} - \cot \theta
\end{equation}

\begin{equation}
   x =  x_0 e^{t \sin \theta} + (e^{t \sin \theta} - 1) \cot \theta
\end{equation}

\begin{equation}
   x =  x_0 e^{t \sin \theta} + [1 + t \sin \theta + \frac{1}{2!} t\sin^2 \theta  + \frac{1}{3!} t \sin^3 \theta + ... - 1] \cot \theta
\end{equation}

\begin{equation}
   x =  x_0 e^{t \sin \theta} + t \cos \theta + \sin \theta \cos \theta (\frac{t^2}{2!} + \frac{t^3}{3!} \sin \theta + \frac{t^4}{4!} \sin^2 \theta + ...)
\end{equation}

\begin{equation}
   x =  x_0 e^{t \sin \theta} + t \cos \theta + \frac{1}{2} \sin 2\theta (\frac{t^2}{2!} + \frac{t^3}{3!} \sin \theta + \frac{t^4}{4!} \sin^2 \theta + ...)
\end{equation}

\begin{equation}
   x =  x_0 e^{t \sin \theta} + t \cos \theta + \frac{1}{2} \sin 2\theta (\frac{t^2}{2!} + \frac{t^3}{3!} \sin \theta + \frac{t^4}{4!} (\frac{1 - \cos 2\theta}{2}) + ...)
\end{equation}

When $\theta = \frac{k \pi}{2}, k = 0, 1, 2, 3...$, we have

\begin{equation}
    x = x_0 e^{t \sin \theta} + t \cos \theta
\end{equation}

rewrite it:

\begin{equation}
   x =  x_0 e^{t \sin \theta} + t \cos \theta + \frac{1}{2} \sin 2\theta \Psi(t)
\end{equation}

\begin{equation}
   \Psi(t) =  \frac{t^2}{2!} + \frac{t^3}{3!} \sin \theta + \frac{t^4}{4!} (\frac{1 - \cos 2\theta}{2}) + \frac{t^5}{5!} (\frac{3 \sin \theta - \sin 3\theta}{4}) + \frac{t^6}{6!} (\frac{3 - 4 \cos 2\theta - \cos 4\theta}{8}) + ...
\end{equation}

\begin{equation}
   \Psi(t) = c_0 + s_1 \sin \theta + c_2 \cos 2\theta + s_3 \sin 3\theta + c_4 \cos 4\theta + s_5 \sin 5\theta + c_6 \cos 6\theta + ...
\end{equation}

\begin{equation}
   c_0 = \frac{t^2}{2!} + \frac{1}{2}\frac{t^4}{4!} + \frac{3}{8} \frac{t^6}{6!} + \frac{5}{16} \frac{t^8}{8!} + \frac{35}{128} \frac{t^{10}}{10!}  ...
\end{equation}

\begin{equation}
   s_1 = \frac{t^3}{3!} + \frac{3}{4}\frac{t^5}{5!} + \frac{5}{8} \frac{t^7}{7!} + \frac{35}{64} \frac{t^9}{9!} + \frac{63}{128} \frac{t^{11}}{11!}  ...
\end{equation}

\begin{equation}
   c_2 = \frac{t^3}{3!} + \frac{3}{4}\frac{t^5}{5!} + \frac{5}{8} \frac{t^7}{7!} + \frac{35}{64} \frac{t^9}{9!} + \frac{63}{128} \frac{t^{11}}{11!}  ...
\end{equation}

\begin{equation}
   s_3 = \frac{t^3}{3!} + \frac{3}{4}\frac{t^5}{5!} + \frac{5}{8} \frac{t^7}{7!} + \frac{35}{64} \frac{t^9}{9!} + \frac{63}{128} \frac{t^{11}}{11!}  ...
\end{equation}

\begin{equation}
   c_4 = \frac{t^3}{3!} + \frac{3}{4}\frac{t^5}{5!} + \frac{5}{8} \frac{t^7}{7!} + \frac{35}{64} \frac{t^9}{9!} + \frac{63}{128} \frac{t^{11}}{11!}  ...
\end{equation}

\begin{equation}
   s_5 = \frac{t^3}{3!} + \frac{3}{4}\frac{t^5}{5!} + \frac{5}{8} \frac{t^7}{7!} + \frac{35}{64} \frac{t^9}{9!} + \frac{63}{128} \frac{t^{11}}{11!}  ...
\end{equation}

\begin{equation}
    {\left(C - \int \cos\left(\theta\left(t\right)\right) e^{\left(\int \sin\left(\theta\left(t\right)\right)\,{d t}\right)}\,{d t}\right)} e^{\left(-\int \sin\left(\theta\left(t\right)\right)\,{d t}\right)}
\end{equation}


\section{Other examples of driving charts}\label{sec:meodc}

\subsubsection{a constant k}

\begin{equation}
    0 = \cos \theta + k \sin \theta
\end{equation}

\begin{equation}
    \theta = n\pi - \arctan \frac{1}{k}
\end{equation}

\subsubsection{identity}

\begin{equation}
    \cos \theta + t \sin \theta - 1 = 0
\end{equation}

\begin{equation}
    \theta = 2k\pi
\end{equation}
\begin{equation}
    \theta = 2k\pi + \tan^{-1} t
\end{equation}

\subsubsection{inverse of t}

\begin{equation}
    \frac{d(t^{-1})}{dt} = \cos \theta + t^{-1} \sin \theta
\end{equation}

\begin{equation}
    t^2 \cos \theta + t \sin \theta + 1 = 0
\end{equation}

\subsubsection{square of t}

\begin{equation}
    \frac{d(t^2)}{dt} = \cos \theta + t^2 \sin \theta
\end{equation}

\begin{equation}
    \cos \theta + t^2 \sin \theta - 2 t = 0
\end{equation}

\subsubsection{cube of t}

\begin{equation}
    \frac{d(t^3)}{dt} = \cos \theta + t^3 \sin \theta
\end{equation}

\begin{equation}
    \cos \theta + t^3 \sin \theta - 3 t^2 = 0
\end{equation}

\subsubsection{sin of t}

\begin{equation}
    \frac{d(\sin t)}{dt} = \cos \theta + \sin t \sin \theta
\end{equation}

\begin{equation}
     \cos \theta + \sin t \sin \theta - \cos t = 0
\end{equation}

\subsubsection{exp of t}

\begin{equation}
    \frac{d(e^t)}{dt} = \cos \theta + e^t \sin \theta
\end{equation}

\begin{equation}
     \cos \theta + e^t \sin \theta - e^t = 0
\end{equation}


\section{Accknowledgement}\label{sec:accknowledgement}

\end{document}
