\documentclass[a4paper,12pt]{article}
\usepackage{amsmath, amsthm, amssymb}
\usepackage{datetime}
\usepackage{tikz} % 添加绘图
\usepackage{tabularx}
\usepackage{enumitem}
\usepackage{fancyref}

\usepackage[top=1in,bottom=1in,left=1in,right=1in]{geometry} % 用于设置页面布局
\usepackage{xeCJK} % 用于使用本地字体
\usepackage[super, square, sort&compress]{natbib} % 处理参考文献
\usepackage{titlesec, titletoc} % 设置章节标题及页眉页脚
%\usepackage{xCJKnumb} % 中英文数字转换
\usepackage{amssymb}
\usepackage{amsmath} % 在公式中用\text{文本}输入中文
\usepackage{diagbox}
\usepackage{multirow} % 表格中使用多行
\usepackage{booktabs} % 表格中使用\toprule等命令
\usepackage{rotating} % 使用sidewaystable环境旋转表格
\usepackage{tabularx}
\usepackage{graphicx} % 处理图片
\usepackage{footnote} % 增强的脚注功能,可添加表格脚注
\usepackage{threeparttable} % 添加真正的表格脚注,示例见README
\usepackage{hyperref} % 添加pdf书签

\usepackage{tikz}
\usetikzlibrary{shapes,arrows,shadows}

% 字体设置
\setmainfont{Times New Roman}
\setsansfont[Scale=MatchLowercase,Mapping=tex-text]{PT Sans}
\setmonofont[Scale=MatchLowercase]{PT Mono}
\setCJKmainfont[ItalicFont={Kaiti SC}, BoldFont={Heiti SC}]{Songti SC}
\setCJKsansfont{Heiti SC}
\setCJKmonofont{Songti SC}
% \setCJKmainfont[BoldFont={FZXiaoBiaoSong-B05S}]{Songti SC}
% \setCJKfamilyfont{kai}[BoldFont=Heiti SC]{Kaiti SC}
% \setCJKfamilyfont{song}[BoldFont=Heiti SC]{Songti SC}
% \setCJKfamilyfont{hei}[BoldFont=Heiti SC]{Heiti SC}
% \setCJKfamilyfont{fsong}[BoldFont=Heiti SC]{Songti SC}
% \newcommand{\kai}[1]{{\CJKfamily{kai}#1}}
% \newcommand{\hei}[1]{{\CJKfamily{hei}#1}}
% \setromanfont[Mapping=tex-text]{TeXGyrePagella}
% \setsansfont[Scale=MatchLowercase,Mapping=tex-text]{TeXGyrePagella}
% \setmonofont[Scale=MatchLowercase]{Courier New}
%%设置常用中文字号,方便调用
\newcommand{\erhao}{\fontsize{22pt}{\baselineskip}\selectfont}
\newcommand{\xiaoerhao}{\fontsize{18pt}{\baselineskip}\selectfont}
\newcommand{\sanhao}{\fontsize{16pt}{\baselineskip}\selectfont}
\newcommand{\xiaosanhao}{\fontsize{15pt}{\baselineskip}\selectfont}
\newcommand{\sihao}{\fontsize{14pt}{\baselineskip}\selectfont}
\newcommand{\xiaosihao}{\fontsize{12pt}{\baselineskip}\selectfont}
\newcommand{\wuhao}{\fontsize{10.5pt}{\baselineskip}\selectfont}
\newcommand{\xiaowuhao}{\fontsize{9pt}{\baselineskip}\selectfont}
\newcommand{\liuhao}{\fontsize{7.5pt}{\baselineskip}\selectfont}

% 章节标题显示方式及页眉页脚设置
% \item xCJKnumb是自己额外安装的包
% \item titleformat命令定义标题的形式
% \item titlespacing定义标题距左、上、下的距离
\titleformat{\section}{\raggedright\large\bfseries}{\thesection}{1em}{}
\titleformat{\subsection}{\raggedright\normalsize\bfseries}{\thesubsection}{1em}{}
\titlespacing{\section}{0pt}{*0}{*2}
\titlespacing{\subsection}{0pt}{*0}{*1}
% 由于默认的2em缩进不够,所以我手动调整了,但是在windows下似乎2.2就差不多了,或者是article中没有这个问题
\setlength{\parindent}{2.2em}

% 设置表格标题前后间距
\setlength{\abovecaptionskip}{0pt}
\setlength{\belowcaptionskip}{0pt}


\renewcommand{\refname}{\bfseries{参~考~文~献}} %将Reference改为参考文献(用于 article)
% \renewcommand{\bibname}{参~考~文~献} %将bibiography改为参考文献(用于 book)
\renewcommand{\baselinestretch}{1.38} %设置行间距
\renewcommand{\figurename}{\small\ttfamily 图}
\renewcommand{\tablename}{\small\ttfamily 表}


\usepackage{stmaryrd}
\usepackage{mathtools}

\title{Effective Processes and Natural Law}
\date{September 9, 1985}
\author{Robert Rosen}

\begin{document}
\maketitle

\setcounter{tocdepth}{2}
\tableofcontents

\section{Introduction}

One of the most remarkable confluences of ideas in modern scientific history occurred in the few short years between the publication of Godel's
original papers on formal undecidability in 1931, and the work of McCulloch and Pitts on neural networks, which appeared in 1943.
During these twelve years, fundamental interrelationships were established between logic, mathematics, the theory of the brain,
and the possibilities of digital computation, which still literally takes one's breath away to contemplate in their full scope.
It was believed at that time, and still is today, over half a century later, that these ideas presage a revolution as fundamental
as that achieved by Newton three centuries earlier.

The name of Alan Turing is preeminent in the history of these astonishing developments. For it was Turing, in his seminal paper of 1936,
who first really juxtaposed the relevant ideas through the construction of the class of "machines" which bear his name.
These Turing machines were, on the one hand, explicitly extrapolated from the mental processes of a human being engaged in a mathematical computation;
on another hand, they represented a formal embodiment of logical or algorithmic processes as manifested in mathematics;
and on yet another hand, through the use of the term "machine" they suggested both the harnessing of material processes to extend
our own mathematical capabilities (soon to be realized through the creation of digital computers) and, at an even deeper level, a new and powerful
metaphor for exploring life itself.

At a purely mathematical/logical level it was quickly recognized that the Turing machines were one of a number of equivalent formalisms for
embodying the concept, of an algorithm. In its turn, an algorithm is regarded as the epitome of an effective process for solving a problem.
Now an effective process connotes an idea of absolute necessity; it is an inferential chain which in every case must lead from appropriate initial data to the
corresponding answer or solution. Moreover, an algorithm is a rote process which, once set in motion, requires no further intervention, no reflection,
and no thought in its relentless progression from data to solution. That is why it seems, in retrospect, so natural to embody it in a "machine", as Turing did.

Since the notion of "effective process" is an informal, intuitive concept, while the notion of "algorithm" is a precisely formalized mathematical one,
it was early suggested that the latter can replace the former. This is precisely the substance of Church's Thesis (cf. Kleene 1952),
which asserts that any process one would want to call "effective" can already be carried out by some properly programmed Turing machine.

Now, strictly speaking, all of the developments described so far are entirely formal; they take place in a logical and mathematical universe of
propositions and production rules. In a sense, they are "all software". However, much of their interest resides in the fact that terms like "machine"
or "effective" have connotations which are nonmathematical, which pertain to the external world of natural, material phenomena.
Indeed, as we have already noted, Turing himself designed his mathematical machines as abstractions from a real-world phenomenon,
a human being performing a computation. The suggestion is thus irresistibly conveyed that if this aspect of human mental
activity can be "mechanized", why not others? Why not all? Likewise, if mental processes, involving what happens in a material
brain (i.e., in hardware) can be represented entirely formally, why can there not be other kinds of material systems (i.e., other hardware) which can be
made to do the same thing the brain does? Here we see, in embryo, the field of "artificial intelligence" in its widest ramifications, and much else besides.

But once we have admitted a material significance to words like "machine" or "effective", we have left the world of mathematics and entered
the world of (in the broadest sense) physics. And whereas the Turing machines are, as we have noted, "all software", physics is by contrast "all hardware".
We have thus, for good and ill, introduced a fundamental distinction between hardware and software which is, in itself,
not part of the formal theory with which we began; nor, by the same token, is it part of physics either. This distinction,
as we shall see, is vital, but it is insidious; it is camouflaged in the all-encompassing, umbrella term "machine".  We can
see how insidiously the distinction creeps in from the following quotation taken from Martin Davis's book Computability and Unsolvability (1958):

For how can we ever exclude the possibility of our being presented someday (perhaps by some extraterrestrial visitor) with
a (perhaps extremely complex) device or "oracle" that "computes" a noncomputable function?

This was clearly meant to be the most rhetorical of rhetorical questions, presented in the context of an entirely formal mathematical development.
But we see here clearly the equivocation on the term "machine", resting on a tacit distinction between real hardware and logical software.

The present author has been troubled for a long time by the deep epistemological ramifications of this question.  In particular,
once we admit "hardware", or material systems, into our discussion (and as we have already noted, that was from the beginning the clear if tacit intention),
what happens to the idea of an "effective process"?  A long time ago (Rosen 1962) I considered the question of what Church's Thesis
means in this new context. Specifically: Is Church's Thesis a fundamental restriction on material nature (akin to the exclusion of the perpetuum
mobile by the Laws of Thermodynamics), or not? And what happens to recursiveness if Church's Thesis can be violated by natural processes?

It may be useful to review the salient points of out earlier argument here. Suppose we are given a physical system S (perhaps the alien "computer" of Davis).
Its behavior is governed by physical laws, which we learn about by doing experiments. A typical experiment will involve either doing something to the system
(i.e, perturbing it from outside) or else letting the system do something to its environment, and then observing or measuring the result.
Clearly, both the experimenter's intervention and the measured results are material events. Events are described or characterized by means of
numbers, whose values are determined by the application of suitable meters (cf. Rosen 1978). Suppose for simplicity that our experimental
intervention a is characterized by such a number r(a), and that the resultant behavior of our system is characterized by another such number.
In this way, our experimenter can generate a table of values which defines a function f from numbers to numbers. The reader will recognize
this procedure as a typical input-output characterization of our system S.  The form of the function f clearly tells us something about the laws
which govern the behavior of S. This is, after all, the whole function of experiment in science.

Now surely this experimentation process is in some sense effective. Indeed, as we shall argue at great length below, sequences of events in the
material world (e.g., in the system S) are governed by causal relations, which bind them together quite as inexorably as implication relations
bind propositions. Thus, if our experimental procedures are repeatable (which meansthat the same causal sequence in S can be recreated at will),
then Church's Thesis must mean that any input-output function 1, generated as we have described from any material system S,
must also be recursive or computable. Otherwise, the system S would be precisely the "computer" which Davis assured us is excluded from possibility.

Seen in this light, Church's Thesis is an attempt to draw inferences or conclusions about hardware (physics) from premises about software
(algorithms). Another well-known attempt to do the same thing is embodied in von Neumann's arguments about "self-reproduction"
(Burks 1966; Arbib, this volume; cf. Section 5 below). Here again, the intent is to learn something about the behavior of material systems
(especially organisms) from a formal theory of computation. As might be expected, such enterprises are risky in the extreme;
but if it could be successfully pulled off, the rewards would be great indeed.

At the very least, we can perhaps already see that introducing an idea of "hardware" into formal theory will have some peculiar
ramifications. Complementary peculiarities arise from the other side, when we attempt to introduce ideas of "software" into physics.
However, these very peculiarities promise to tell us something interesting about both. In the remainder of the present paper, we shall
explore some of these possibilities.

\section{Church's Thesis in Formal Systems}

The essence of Church's Thesis is that it identifies logical inference in any
formal system with string processing. In its turn, string processing, or word
processing, is a purely syntactic activity.  String processing is, of course,
precisely what the Turing machines do. Nevertheless, it seems on the face
of it to be a very strong, perhaps excessively strong condition to require
that every inference in a formal system should be expressible in syntactic
terms alone; i.e., that every trajectory from premises to conclusion should be
navigated entirely through the manipulation of the symbols in which these
propositions are encoded.

Nevertheless, a rather strong case can be built to support this rather
unlikely-looking Thesis. It distills a trend towards formalization which
began with Euclid, became a matter of urgency in the confusion following the
discovery of non-Euclidean geometries (i.e., geometries which, as formal
systems, were consistent as God-given Euclid), and of absolute desperation
when the paradoxes in naive set theory were revealed. The formalistic response
to this situation, pioneered by David Hilbert, was precisely to empty
mathematics of any semantic content whatsoever, arguing in effect that it
was a needless semantics which was at the root of the difficulties.
This in effect turned all of mathematics into a kind of game in which meaningless
symbols were manipulated according to (a finite family of) arbitrary syntactical rules.
Indeed, the whole point of Hilbertian Formalism is to create systems in which there is
nothing but syntax.

Perhaps the clearest statement of this kind of Formalist program was given by Kleene 1952:

This step (axiomatization) will not be finished until all the properties of the
undefined or technical terms of the theory which matter for the deduction
of theorems have been expressed by axioms. Then it should be possible to
perform the deductions treating the technical terms as words in themselves
without meaning.  For to say that they have meanings necessary to the deduction
of the theorems, other than what they derive from the axioms which
govern them, amounts to saying that not all of their properties which matter
for the deductions have been expressed by axioms. When the meanings
of the technical terms are thus left out of account, we have arrived at the
standpoint of formal axiomatics \ldots Since we have abstracted entirely from
the content matter, leaving only the form, we say that the original theory has
been formalized. In this structure, the theory is no longer a system of meaningful
propositions, but one of sentences as sequences of words, which are
in turn sequences of letters.  We say by reference to the form alone which
combinations of words Itre sentences, which sentences are axioms, and which
sentences follow as immediate consequences of others.

Clearly, the idea here is that it is always possible to replace semantics ("meanings")
with syntactics, so that logically no information is lost; any inference involving semantics
possesses a purely syntactical image in the formalization.

In such a formal system, we start with the idea that the axioms are true.
This notion of truth is hereditary; if the axioms are true, then so also are the
symbol sequences obtained by applying the inferential rules of the system to them;
thus truth passes from axioms to theorems. So far, we never need to import a notion of "truth"
into the system from outside, as it were; we simply construct true propositions (theorems)
as we go along.

The troubles embodied in Gliders celebrated theorems (GOdel 1931)
arise from trying to compare this constructive notion of internal truth with
preassigned external truth-value in formal arithmetic; as Godel showed, they
do not match. Furthermore, one way of looking at Turing's theorem on decidability (Turing 1936-7)
is that there is no internal inferential mechanism for deciding whether a proposition
in the given formalization is a theorem(i.e., true) or not. Thus, we know that the Formalist program,
in which only purely syntactic inferences are allowed, is too impoverished to even play the game of number theory.
That is, we must either allow into our system some "informal" inferential procedures, which according to Godes Theorem
cannot be reduced to syntactics even in principle, or else restrict ourselves forever to mere fragments of number theory.
Such "informal" inferential procedures are what are disallowed by Church's Thesis; they are ineffective.

Let us put the above discussion into more familiar terms. In ordinary
("Platonic") mathematics, which of course has both a syntactic and a semantic aspect,
we know that if we are given a set, a variety of other sets
can always be built from S via canonical constructions. For example, we
have the power set 28; we have the free algebraic structures (semigroup,
group, etc.)  generated by S, we have the set H(S, S) of all maps from
S to S; we have the Cartesian product S x S, etc. These associated sets
give us inferential capabilities which have the power of logical relations,
without necessarily being expressible in terms of the formal inferential laws
which govern the mathematical system from which S came. For instance,
if Q : S 	S is an automorphism of S (i.e., an element of H(S, S)), and
s E S, we may say that Q(s) = s' establishes an implication relation between and s'.
But this "implication" need not coincide with any we can draw
from the production rules governing the system; i.e., need not follow from
system syntactics alone. If it does, we may say that our Q is computable in
the system; otherwise, not computable. In the latter case, we would have to
say that the mapping Q is not effective, according to Church's Thesis. But
clearly, if Q has any meaning or existence at all, once it is given to us, it is
effective.

In the "all-software" world of formal systems, we can of course restrict ourselves in any way we like.
Thus, we can agree not to allow ourselves any automorphisms Q which cannot be expressed in purely syntactical
terms. In such a world, and only in such a world, could Church's Thesis hold unrestrictedly.
Whether such a formal world would be at all interesting is, of course, another question.
And as we shall see, the situation gets even worse when we allow "hardware" into our world.

Before turning to material systems, we should say a word about the encoding of propositions
in a formal system onto Turing machine tapes, and decoding tapes back into propositions in the system.
It is fairly clear that we may use the word "effective" in its usual intuitive sense in connection
with encoding and decoding; the familiar Godel numbering, for example, is
clearly an effective mapping from syntax to arithmetic and back. Indeed,
in showing that a process in some formal system is effective, or recursive,
we can clearly combine the encoding, the computation, and the decoding
into a single Turing machine which does all three. However, we merely note
here for future reference that the encoding and decoding are logically distinct from each other
and from the actual computation; if these are in some sense "ineffective", then Church's Thesis
may appear to fail, even though the computation itself is completely recursive.

\section{Implication and Causality}

As we have seen, in the realm of formal systems, Church's Thesis identifies the intuitive notion of "effective process" with the purely syntactic idea of string processing.

That is, any implication which can be "effectively" performed within the system can already be carried out by means of a finite set of production rules,
operating on finite strings of symbols taken from a f i nite alphabet. We have also pointed out that when we deal with the material world
(as opposed to the formal ones of mathematics and logic) ideas of implication are replaced by ideas of causality. Nevertheless, we can still
retain the idea of an "effective" process. In this context, Church's Thesis means that any causal sequence can be represented by a corresponding
recursive process; i.e., any causal sequence can be described by purely syntactic means. If this is true, it of course places severe
limitations on what physics can be like. The question is no less than whether the Laws of Nature can themselves be formulated in purely
syntactical terms, or whether they can possess an inherent semantic component which cannot be finitistically formalized.
It should be noted that the urge to formalization (i.e., to pure syntactics) in mathematics is exactly parallel to similar trends in theoretical science.
Indeed, the whole thrust of atomic theory (or nowadays, the theory of "elementary particles") is to reduce all material processes to the motion of
ultimate constituent units, devoid of any internal structure ("meaning"), possessing only an instantaneous position ("configuration")
and the temporal derivatives of position. The forces which push these ultimate units around are the precise analogs of the production rules in a formal system.
Hence the paths or trajectories traced out by a material system under the influence of given forces are the analogs of formal theorems,
with initial conditions as axioms. The idea that causal relations between events in material systems can be related to implication relations between
propositions describing those events is the sine qua non of theoretical science. Indeed, the belief in what used to be called Natural Law requires (a)
that the sequences of events we perceive in the external world are not arbitrary or whimsical, but are governed by definite rules (this is Causality),
and (b) that these rules can be articulated in such a way that they can be grasped by the human mind. Taken together, this formulation of Natural Law
asserts precisely that causal relations in material systems can be brought into congruence with implications in a
formal (ultimately, mathematical) system of propositions about those events.

This situation can be most succinctly expressed in terms of a diagram (see Figure 1).

We say that a modeling relation exists between the natural system on the left of the diagram, and the formal system on the right, when the following

commutativity holds:
0 = 02 + 0 + 	(1)
That is, we get the same answer whether we simply sit as observers, and watch the unfolding sequence of events in the natural system, or whether we (a) encode some properties of the natural system into the formalism, (b) use the implicative structure of the formal systhm to derive theorems, and then (c) decode these theorems into propositions (predictions) about the natural system itself.  When the diagram commutes, we have established a congruence between (some of) the causal features of the natural system and the implicative structure of the formal system. We can then say that the formal system is a model of the natural one, or alternatively, that the natural system is a realization of the formal one.
   These little diagrams themselves possess a number of rich and important epistemological properties, which we cannot enter into here; for a fuller discussion, see e.g. Rosen 1985.
Once we have thus constructed a formal system which is a model for
some natural process, we have left the realm of science and entered that
of mathematics. We can then treat a model as we would any other formal
system. In particular, we can look at its purely syntactic aspects, which we can immediately identify with the "effective" processes of Church's Thesis,
and ask whether these exhaust the implicative resources of the system itself.

In this way, we can construct a purely syntactic "machine" model of our original natural system, as indicated in Figure 2.

This can be seen by looking only at the outer two systems, and forgetting about our original model, which now plays the role of a "transducer"
between them.

However, when we do this, the following points must be explicitly noticed: (a) the encoding and decoding arrows between them (the
dotted arrows in Figure 2) cannot be described as effective in any formal
sense, and (b) these encoding and decoding arrows involve exclusively the
input and output strings of the machines inhabiting the right-most box. Both
of these observations are important. We shall briefly discuss them each in
turn.

In applying Church's Thesis to formal systems, we noted above that the
corresponding encoding and decoding arrows themselves represented formal processes, which could in fact be amalgamated into
the Thesis itself. However, when we wish to compare a natural system, governed by causality, with a formal system, governed by
implication, this is no longer the case. The encoding instruments, or transducers, are now themselves material systems;
i.e., governed by causality and not by implications. As noted above, they are (in the broadest sense) meters. Since these meters are governed by
causality, anything they do is effective in a material sense. But it is clear that a formalization of the encoding process would require more models;
these would in turn require their own encoding and decoding processes, which would require more models, etc., an infinite regress.
It is precisely this fact which makes "the measurement problem" so hard in physics. The question as to whether this potential infinite regress
can be terminated at some finite point is a deep epistemological question about the nature of the world, with close ties to such things as
reductionism. We cannot of course enter into such matters here; we will merely assume that a set of meters or other transducers
from the natural world to input tapes is given, and that Church's Thesis can be investigated relative to these encodings
(which are then, by hypothesis, effective in the material sense).

Our second observation is also epistemologically important. It says that
all relevant features of a material system, and of the model into which it
was originally encoded (cf. Figure 1), are to be expressed as input strings to
be processed by a machine whose structure itself encodes nothing. That is to
say, the rules governing the operation of these machines, and hence the entire
inferential structure of the string-processing system themselves, have no
relation at all to the material system being encoded. The'only requirement
is that the requisite commutativity hold, as expressed in Section 1 above,
between the encoding on input strings and the decoding of the resultant
output strings. As we shall see, this is the essence of simulation.

We have already noted that Church's Thesis amounts to asserting that
all causal relationships can be expressed in purely syntactic terms. We can
formulate the Thesis still more sharply now: relative to any given encoding
of a natural system into input strings, the string-processing machinery itself
must not encode any aspect of the material system.  That is to say, the
"hardware" of the machines must be totally independent of the "hardware"
generating the strings to be processed. If this cannot be done, then Church's
Thesis cannot be true.

Given the above, Church's Thesis asserts that the Turing machines constitute
a class of universal simulators for all material processes. There are
then two questions to be asked: (a) is it true? and (b) if so, what does it
mean?

\section{Is Church's Thesis Physically True?}

The upshot of the argument of the preceding sections is the following. Na-
ture provides us with a plethora of material processes which we would want
to call "effective". These processes are (we suppose) governed by causality,
and not by implication or production rules, as in a formal system. In these
terms, Church's Thesis can be expressed as follows. Given any such process,
we can encode appropriate propositions about the natural system generating
it onto a set of input tapes to a Turing machine; and the corresponding output
tapes can be decoded so as to perfectly simulate the process in question.
Equivalently, Church's Thesis asserts that all "information" about material
processes, and hence all of Natural Law, can be expressed in purely syntactic
terms.

We know from Gliders Theorem that sufficiently rich formal systems
always contain inferences which cannot be obtained syntactically.  More
specifically, given any encoding of propositions of the system onto input
tapes to Turing machines, there will be propositions of the system which are
"true" but will never appear on any output tape. The processes by which the
"truth" of such inferences are established are thus ineffective; they cannot
in principle be simulated by any Turing machine in the given encoding. We
can always change the encoding, of course, but this will not change Gliders
conclusion.

Thus, in formal systems, we already find that a purely syntactical encoding
will in some sense lose information. The information lost must then
pertain to an irreducible, unformalizable semantic component in the original
inferential structure. By changing the encodings, we can shift to some
extent where this semantic information resides, but we cannot eliminate it.

By itself, this result of Godel does not bear on the physical truth of
Church's Thesis, since it is a purely formal result. But it is in fact suggestive
of how the physical font of Church's Thesis might be verified or falsified.

As we have seen above, the manner in which we compare material processes with formal ones is through the establishment of
modeling relations, as diagrammed in Figure 1 above. Formal models of material systems are
then perfectly good formal systems, whose inferential structures by definition reflect causal processes in the natural system being modeled.
Thus, if a model, arising in this fashion, should fall within the purview of Gliders
argument, this would at least be strong evidence that Church's Thesis is
false as a physical proposition. Stated another way, there would exist physical processes which could effectively compute
nonrecursive functions. It would also mean that Natural Law cannot be expressed entirely in syntactical terms.

The obvious thing to look for, then, is a model of a material system
which is rich enough as a formalism to "do arithmetic". The formalisms
which physics provides, as models of purely physical systems, are unfortunately
extremely impoverished, considered simply as formal systems. But,
as we have argued elsewhere, these formalisms are in fact highly nongeneric
and do not suffice to image material systems like organisms (cf. Rosen, in
press). One way of expressing this nongenericity is precisely in terms of the
way they image causal structures. When this nongenericity is lifted, a new
class of (potential) models is obtained in which the image of causal structure
is infinitely richer and more complicated. Since it is precisely the formal
imaging of causal structures which is at the heart of Church's Thesis, we may perhaps find in these formalisms many processes
which are causally effective, but mathematically ineffective. This would mean that the behavior of such systems must contain
an irreducible semantic component, one intimately related to the complexity of the system.

A different approach was taken long ago by John Myhill (cf. Myhill 1966).
He was able to show that, modulo some idealizations regarding measurement and performance tolerances, there are already
classical analog devices(analog computers) which could "compute" nonrecursive functions. Such systems would thus already
manifest behaviors which could not be predicted by any purely syntactic encoding, and hence would also have an irreducible
semantic aspect.

Finally, we have already mentioned that Newtonian particle mechanics, and more recently, the unified physical theories based on elementary particles,
are in themselves an attempt to express the Laws of Nature in purely syntactic terms. Insofar as any material system is comprised of such "meaningless"
(i.e., structureless) elementary subunits, these theories at heart assert that to understand any behavior of any
such system it suffices to describe it in terms of these subunits and their interactions. As noted earlier, this is the essence of reductionism.

In these terms, the familiar Laplacian Spirit is a purely syntactical concept: the embodiment of Church's Thesis if he could exist.  As a matter
of fact, he could not exist (or at any rate, not as a material system built of particles himself), for reasons we have already indicated.
But even if he could exist, he would be a very poor biologist, for example; organisms, and open systems in general, are constantly turning over their
constituent particles. Thus, to even find an organism, let alone follow it in time, he would need to supplement his purely syntactical
information with other (semantic) information not formalizable within his system.

Thus, for a variety of reasons, there is cause to believe that Church's Thesis fails as a physical proposition. Nevertheless,
as we have seen, to state and analyze the Thesis in material terms touches on some of the deepest and most basic aspects of theoretical science.

\section{The Role of Simulations}

We have already alluded above to the use of the Turing machines both
as metaphors for the material world and as effective descriptions of that
world.  In both cases, though in different ways, we seek to draw conclusions
about material processes from a purely syntactic formalism. In this final section,
I will very briefly consider one well-known example of this:
the "self-reproducing automata" of von Neumann (Burks 1966; Arbib, this volume).

The basis of von Neumann's argument was the inference of the existence of a "universal constructor" from Turing's
argument for a universal simulator (computer).  On the purely formal side, von Neumann constructed
a universe ("cellular space") of intercommunicating Turing machines arranged along some regular geometric array,
like the cells in a multicellular organism, or ,the neurons in a brain, or atoms in a crystal. Each machine
communicates with its nearest neighbors in the array. In the obvious fashion, the array as a whole changes "state" in time.
The question was whether some sub-array ("tessellation automaton") could induce certain interesting behaviors in its complement,
which could be interpreted as construction, replication, growth, development, evolution, and so on.

At the same time, von Neumann clearly believed that a "universal constructor" could exist as hardware. He envisaged equipping a Turing machine
with sensors, so that it could "read" a blueprint, and with effectors,
so that it could extract physical components from its environment, and
assemble them as instructed by the blueprint being read. (An "effector",
in this context, is a transducer from numbers to things, i.e., an "inverse
meter".) The idea was -that both computation and construction were algorithmic processes,
and therefore whatever was true of the one must be true
of the other.

Von Neumann also felt that the cellular spaces were not merely formal constructs, but actually comprised models of real-world
constructors and their activities. In this, he was tacitly assuming Church's Thesis in its strongest form.

We have argued elsewhere (cf. Rosen 1985), on grounds of causality, that
any inference regarding a material universal constructor from the existence
of a formal universal computer is unjustified.  In that argument, we essentially
showed that there was no intrinsic way of distinguishing between
those input strings which encode "real-world information" and those which
do not. Thus there was no way to distinguish between a computation which
could be claimed to simulate a "real-world" process, and one which has no
such realization.

Conversely, we can also see that the falsity of Church's Thesis means that there are aspects of material processes which cannot be formalized with any given encoding.
Thus there are (a) formal constructions without material counterpart, and conversely, (b) m material constructions without formal counterpart.
Therefore, on both counts, von Neumann's argument is without material content, and merely involves the familiar equivocation on
the terms "automaton" (or "machine") and "construction".

These considerations show how dangerous it can be to extrapolate unrestrictedly from formal systems to material ones.
The danger arises precisely from the fact that computation involves only simulation, which allows the establishment of
no congruence between causal processes in material systems and inferential processes in the simulator. We therefore lack precisely
those essential features of encoding and decoding which are required for such extrapolations. Thus, although formal simulators can be of
great practical and heuristic value, their theoretical significance is very sharply circumscribed, and they must be used with the greatest caution.

There are, of course, many other ramifications of Church's Thesis which we cannot touch on in this brief space. Its main role, as we have seen,
is to separate out what is syntactic in a formal system from what is not; when the formal system is also a model of a material system,
Church's Thesis does the same for causal relations. The Thesis in fact raises a host of deep questions about Natural Law, about causality,
about modeling, and about the material realization of formalisms. Its central feature, the Turing machines, embody the essence of syntactics
or string processing in a single, conceptually rich package. Even if (as I believe) Church's Thesis fails, it does so in a most instructive way.
Its implications for the material sciences, and especially for biology, have barely begun to be explored.

\end{document}




