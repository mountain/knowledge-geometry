\newglossaryentry{图灵机}
{
  sort=TLJ,
  name=图灵机,
  description={Turing machine}
}
\newglossaryentry{算法}
{
  sort=SF,
  name=算法,
  description={algorithm}
}
\newglossaryentry{算法过程}
{
  sort=SFGC,
  name=算法过程,
  description={algorithmic process}
}
\newglossaryentry{有效过程}
{
  sort=YXGC,
  name=有效过程,
  description={effective process}
}
\newglossaryentry{邱奇论题}
{
  sort=QQLT,
  name=邱奇论题,
  description={Church Thesis}
}
\newglossaryentry{神谕}
{
  sort=SY,
  name=神谕,
  description={oracle}
}
\newglossaryentry{软件}
{
  sort=RJ,
  name=软件,
  description={software}
}
\newglossaryentry{硬件}
{
  sort=YJ,
  name=硬件,
  description={hardware}
}
\newglossaryentry{可计算}
{
  sort=KJS,
  name=可计算,
  description={computable}
}
\newglossaryentry{不可计算}
{
  sort=BKJS,
  name=不可计算,
  description={non-computable}
}
\newglossaryentry{人工智能}
{
  sort=RGZN,
  name=人工智能,
  description={Artificial Intelligent}
}
\newglossaryentry{因果}
{
  sort=YG,
  name=因果,
  description={causality}
}
\newglossaryentry{因果关系}
{
  sort=YGGX,
  name=因果关系,
  description={causal relation}
}
\newglossaryentry{因果序列}
{
  sort=YGXL,
  name=因果序列,
  description={causal sequence}
}
\newglossaryentry{因果过程}
{
  sort=YGGC,
  name=因果过程,
  description={causal process}
}
\newglossaryentry{因果结构}
{
  sort=YGJG,
  name=因果结构,
  description={causal structure}
}
\newglossaryentry{逻辑}
{
  sort=LJ,
  name=逻辑,
  description={logic}
}
\newglossaryentry{逻辑学}
{
  sort=LJX,
  name=逻辑学,
  description={logic}
}
\newglossaryentry{逻辑关系}
{
  sort=LJGX,
  name=逻辑关系,
  description={logical relation}
}
\newglossaryentry{逻辑推理}
{
  sort=LJTL,
  name=逻辑推理,
  description={logical inference}
}
\newglossaryentry{字符串处理}
{
  sort=ZFCCL,
  name=字符串处理,
  description={string processing}
}
\newglossaryentry{语法}
{
  sort=YF,
  name=语法,
  description={syntactics}
}
\newglossaryentry{语法项}
{
  sort=YFX,
  name=语法项,
  description={syntactic term}
}
\newglossaryentry{语法规则}
{
  sort=YFGZ,
  name=语法规则,
  description={syntactical rule}
}
\newglossaryentry{语法信息}
{
  sort=YFXX,
  name=语法信息,
  description={syntactical information}
}
\newglossaryentry{语法形式}
{
  sort=YFXS,
  name=语法形式,
  description={syntactic formalism}
}
\newglossaryentry{语法编码}
{
  sort=YFBM,
  name=语法编码,
  description={syntactical encoding}
}
\newglossaryentry{语义}
{
  sort=YY,
  name=语义,
  description={semantics}
}
\newglossaryentry{语义内容}
{
  sort=YYNR,
  name=语义内容,
  description={semantic content}
}
\newglossaryentry{语义成分}
{
  sort=YYCF,
  name=语义成分,
  description={semantic component}
}
\newglossaryentry{语义信息}
{
  sort=YYXX,
  name=语义信息,
  description={semantic information}
}
\newglossaryentry{干预}
{
  sort=GY,
  name=干预,
  description={manipulates input data, and provides output in a useful format}
}
\newglossaryentry{测量}
{
  sort=CL,
  name=测量,
  description={measurement}
}
\newglossaryentry{测量问题}
{
  sort=CLWT,
  name=测量问题,
  description={the measurement problem}
}
\newglossaryentry{事件}
{
  sort=SJ,
  name=事件,
  description={event}
}
\newglossaryentry{事件序列}
{
  sort=SJXL,
  name=事件序列,
  description={sequences of event}
}
\newglossaryentry{观察}
{
  sort=GC,
  name=观察,
  description={observation}
}
\newglossaryentry{观察者}
{
  sort=GCZ,
  name=观察者,
  description={observer}
}
\newglossaryentry{科学实验}
{
  sort=KXSY,
  name=科学实验,
  description={experiment in science}
}
\newglossaryentry{可复现}
{
  sort=KFX,
  name=可复现,
  description={repeatable}
}
\newglossaryentry{形式}
{
  sort=XS,
  name=形式,
  description={formal}
}
\newglossaryentry{形式化}
{
  sort=XSH,
  name=形式化,
  description={formalization}
}
\newglossaryentry{形式理论}
{
  sort=XSLL,
  name=形式理论,
  description={formal theory}
}
\newglossaryentry{形式系统}
{
  sort=XSXT,
  name=形式系统,
  description={formal system}
}
\newglossaryentry{形式主义}
{
  sort=XSZY,
  name=形式主义,
  description={Formalism}
}
\newglossaryentry{形式算术}
{
  sort=XSSS,
  name=形式主义,
  description={formal arithmetic}
}
\newglossaryentry{形式体系}
{
  sort=XSTX,
  name=形式体系,
  description={formalism}
}
\newglossaryentry{形式过程}
{
  sort=XSGC,
  name=形式过程,
  description={formal process}
}
\newglossaryentry{形式构造}
{
  sort=XSGZ,
  name=形式构造,
  description={formal construction}
}
\newglossaryentry{物质}
{
  sort=WZ,
  name=物质,
  description={material}
}
\newglossaryentry{物质过程}
{
  sort=WZGC,
  name=物质过程,
  description={material process}
}
\newglossaryentry{物质现象}
{
  sort=WZXX,
  name=物质现象,
  description={material phenomena}
}
\newglossaryentry{物质系统}
{
  sort=WZXT,
  name=物质系统,
  description={material system}
}
\newglossaryentry{物质构造}
{
  sort=WZGZ,
  name=物质构造,
  description={material construction}
}
\newglossaryentry{物质世界}
{
  sort=WZSJ,
  name=物质系统,
  description={material world}
}
\newglossaryentry{物质事件}
{
  sort=WZSJ,
  name=物质事件,
  description={material event}
}
\newglossaryentry{物质科学}
{
  sort=WZKX,
  name=物质科学,
  description={material science}
}
\newglossaryentry{公理}
{
  sort=GL,
  name=公理,
  description={axiom}
}
\newglossaryentry{公理化}
{
  sort=GLH,
  name=公理化,
  description={axiomatization}
}
\newglossaryentry{编码}
{
  sort=BM,
  name=编码,
  description={encoding}
}
\newglossaryentry{编码工具}
{
  sort=BMGJ,
  name=编码工具,
  description={encoding instrument}
}
\newglossaryentry{编码过程}
{
  sort=BMGC,
  name=编码过程,
  description={encoding process}
}
\newglossaryentry{解码}
{
  sort=JM,
  name=解码,
  description={decoding}
}
\newglossaryentry{解码过程}
{
  sort=JMGC,
  name=解码过程,
  description={decoding process}
}
\newglossaryentry{转换器}
{
  sort=ZHQ,
  name=转换器,
  description={transducer}
}
\newglossaryentry{哥德尔编码}
{
  sort=GDEBM,
  name=哥德尔编码,
  description={G{\"o}del numbering}
}
\newglossaryentry{自然法则}
{
  sort=ZRFZ,
  name=自然法则,
  description={Natural Laws, Laws of Nature}
}
\newglossaryentry{建模关系}
{
  sort=JMGX,
  name=建模关系,
  description={modeling relation}
}
\newglossaryentry{模型}
{
  sort=MX,
  name=模型,
  description={model}
}
\newglossaryentry{推理}
{
  sort=TL,
  name=推理,
  description={inference}
}
\newglossaryentry{推理链条}
{
  sort=TLLT,
  name=推理链条,
  description={inferential chain}
}
\newglossaryentry{推理规则}
{
  sort=TLGZ,
  name=推理规则,
  description={inferential rule}
}
\newglossaryentry{推理机制}
{
  sort=TLJZ,
  name=推理机制,
  description={inferential mechanism}
}
\newglossaryentry{推理能力}
{
  sort=TLNL,
  name=推理能力,
  description={manipulates input data, and provides output in a useful format}
}
\newglossaryentry{推理结构}
{
  sort=TLJG,
  name=推理结构,
  description={inferential structure}
}
\newglossaryentry{推理过程}
{
  sort=TLGC,
  name=推理过程,
  description={inferential process}
}
\newglossaryentry{蕴涵}
{
  sort=YH,
  name=蕴涵,
  description={implication}
}
\newglossaryentry{蕴涵关系}
{
  sort=YHGX,
  name=蕴涵关系,
  description={implication relation}
}
\newglossaryentry{蕴涵结构}
{
  sort=YHJG,
  name=蕴涵结构,
  description={implicative structure}
}
\newglossaryentry{蕴涵性资源}
{
  sort=YHXZY,
  name=蕴涵性资源,
  description={implicative resources}
}
\newglossaryentry{生成}
{
  sort=SC,
  name=生成,
  description={generation}
}
\newglossaryentry{产生规则}
{
  sort=CSGZ,
  name=产生规则,
  description={production rule}
}
\newglossaryentry{预测}
{
  sort=YC,
  name=预测,
  description={prediction}
}
\newglossaryentry{模拟}
{
  sort=MN,
  name=模拟,
  description={simulation}
}
\newglossaryentry{模拟器}
{
  sort=MNQ,
  name=模拟器,
  description={simulator}
}
\newglossaryentry{通用模拟器}
{
  sort=TYMNQ,
  name=通用模拟器,
  description={universal simulator}
}
\newglossaryentry{模拟计算}
{
  sort=MNJS,
  name=模拟计算,
  description={manipulates input data, and provides output in a useful format}
}
\newglossaryentry{模拟设备}
{
  sort=MNSB,
  name=模拟设备,
  description={analog device}
}
\newglossaryentry{模拟计算机}
{
  sort=MNJSJ,
  name=模拟计算机,
  description={analog computer}
}
\newglossaryentry{数字计算}
{
  sort=SZJS,
  name=数字计算,
  description={digital computation}
}
\newglossaryentry{数字计算机}
{
  sort=SZJSJ,
  name=数字计算机,
  description={digital computer}
}
\newglossaryentry{命题}
{
  sort=MT,
  name=命题,
  description={proposition}
}
\newglossaryentry{定理}
{
  sort=MT,
  name=命题,
  description={theorem}
}
\newglossaryentry{物理命题}
{
  sort=WLMT,
  name=物理命题,
  description={physical proposition}
}

