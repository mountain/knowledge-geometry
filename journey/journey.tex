\documentclass[a4paper,12pt]{article}
\usepackage{amsmath, amsthm}
\usepackage{datetime}
\usepackage{framed}
\usepackage{enumitem}
\usepackage{fancyref}
\usepackage{wrapfig}
\usepackage{pifont}
\usepackage{appendix}
\usepackage{caption}
\usepackage{xcolor}

\usepackage{amsthm}
\usepackage{amssymb}
\usepackage{amsfonts}
\usepackage{amsmath}
\usepackage{mathtools}

\usepackage{tikz}
\usepackage{qtree}
\usetikzlibrary{fit,arrows,calc}
\usepackage{pgfplots}

\usepackage{csquotes}
\renewcommand{\mkbegdispquote}[2]{\itshape}

\newdateformat{nianyueri}{ \THEYEAR 年 \THEMONTH 月 \THEDAY 日 }

\usepackage{data/circledsteps}
\usepackage[top=1in,bottom=1in,left=1in,right=1in]{geometry} % 用于设置页面布局
\usepackage{xeCJK} % 用于使用本地字体
\usepackage[super, square, sort&compress]{natbib} % 处理参考文献
\usepackage{titlesec, titletoc} % 设置章节标题及页眉页脚
\usepackage{amssymb}
\usepackage{amsmath} % 在公式中用\text{文本}输入中文
\usepackage{diagbox}
\usepackage{multirow} % 表格中使用多行
\usepackage{booktabs} % 表格中使用\toprule等命令
\usepackage{rotating} % 使用sidewaystable环境旋转表格
\usepackage{tabularx}
\usepackage{graphicx} % 处理图片
\usepackage{footnote} % 增强的脚注功能,可添加表格脚注
\usepackage{threeparttable} % 添加真正的表格脚注,示例见README
\usepackage{hyperref} % 添加pdf书签

\usepackage{tikz}
\usetikzlibrary{shapes,arrows,shadows}


% 字体设置
\setmainfont{Times New Roman}
\setsansfont[Scale=MatchLowercase,Mapping=tex-text]{PT Sans}
\setmonofont[Scale=MatchLowercase]{PT Mono}
\setCJKmainfont[ItalicFont={FZKai-Z03}, BoldFont={FZHei-B01}]{FZShuSong-Z01}
\setCJKsansfont{FZHei-B01}
\setCJKmonofont{FZShuSong-Z01}

\newcommand{\song}{\CJKfamily{song}} % 宋体
\newcommand{\fs}{\CJKfamily{fs}} % 仿宋体
\newcommand{\kai}{\CJKfamily{kai}} % 楷体
\newcommand{\hei}{\CJKfamily{hei}} % 黑体
\newcommand{\li}{\CJKfamily{li}} % 隶书
\newcommand{\you}{\CJKfamily{you}} % 幼圆
\def\songti{\song}
\def\fangsong{\fs}
\def\kaishu{\kai}
\def\heiti{\hei}
\def\lishu{\li}
\def\youyuan{\you}

%%设置常用中文字号,方便调用
\newcommand{\chuhao}{\fontsize{42pt}{\baselineskip}\selectfont}
\newcommand{\xiaochu}{\fontsize{36pt}{\baselineskip}\selectfont}
\newcommand{\yihao}{\fontsize{26pt}{\baselineskip}\selectfont}
\newcommand{\xiaoyi}{\fontsize{24pt}{\baselineskip}\selectfont}
\newcommand{\erhao}{\fontsize{22pt}{\baselineskip}\selectfont}
\newcommand{\xiaoer}{\fontsize{18pt}{\baselineskip}\selectfont}
\newcommand{\sanhao}{\fontsize{16pt}{\baselineskip}\selectfont}
\newcommand{\xiaosan}{\fontsize{15pt}{\baselineskip}\selectfont}
\newcommand{\sihao}{\fontsize{14pt}{\baselineskip}\selectfont}
\newcommand{\xiaosi}{\fontsize{12pt}{\baselineskip}\selectfont}
\newcommand{\wuhao}{\fontsize{10.5pt}{\baselineskip}\selectfont}
\newcommand{\xiaowu}{\fontsize{9pt}{\baselineskip}\selectfont}
\newcommand{\liuhao}{\fontsize{7.5pt}{\baselineskip}\selectfont}
\newcommand{\xiaoliu}{\fontsize{6.5pt}{\baselineskip}\selectfont}
\newcommand{\qihao}{\fontsize{5.5pt}{\baselineskip}\selectfont}
\newcommand{\bahao}{\fontsize{5pt}{\baselineskip}\selectfont}

% 章节标题显示方式及页眉页脚设置
% \item xCJKnumb是自己额外安装的包
% \item titleformat命令定义标题的形式
% \item titlespacing定义标题距左、上、下的距离
\titleformat{\section}{\raggedright\large\bfseries}{\thesection}{1em}{}
\titleformat{\subsection}{\raggedright\normalsize\bfseries}{\thesubsection}{1em}{}
\titlespacing{\section}{0pt}{*2}{*0}
\titlespacing{\subsection}{0pt}{*1}{*0}

% 由于默认的2em缩进不够,所以我手动调整了,但是在windows下似乎2.2就差不多了,或者是article中没有这个问题
\setlength{\parindent}{0em}
\setlength{\parskip}{0.25em}

% 设置表格标题前后间距
\setlength{\abovecaptionskip}{0pt}
\setlength{\belowcaptionskip}{0pt}

% 设置列表项目前后间距
\setlength\itemsep{0em}

\renewcommand{\refname}{\bfseries{参~考~文~献}} %将Reference改为参考文献(用于 article)
% \renewcommand{\bibname}{参~考~文~献} %将bibiography改为参考文献(用于 book)

\renewcommand{\baselinestretch}{1.4} %设置行间距
\renewcommand{\figurename}{\small\ttfamily 图}
\renewcommand{\tablename}{\small\ttfamily 表}


\usepackage{stmaryrd}
\usepackage{mathtools}
\usepackage{wasysym}

\newtheorem{definition}{定义}
\numberwithin{definition}{section}
\newtheorem{lemma}{引理}
\numberwithin{lemma}{section}
\newtheorem{proposition}{命题}
\numberwithin{proposition}{section}
\newtheorem{theorem}{定理}
\numberwithin{theorem}{section}
\newtheorem{grammar}{文法}
\numberwithin{grammar}{section}
\newtheorem{program}{程序}
\numberwithin{program}{section}
\newtheorem{convention}{约定}
\numberwithin{convention}{section}
\newtheorem{corollary}{推论}
\numberwithin{corollary}{section}
\renewcommand*{\proofname}{证明}

\xeCJKsetwidth{‘’“”}{1em}

\title{数字丛林里的远足\\
\large 从词向量到算术表达式的几何}
\date{2021 年 6 月}
\author{苑明理}

\begin{document}
\maketitle{}
\centerline{\rule{13cm}{0.4pt}}
\renewcommand{\contentsname}{\hfill\bfseries 目录\hfill}
\setcounter{tocdepth}{2}
\tableofcontents
\centerline{\rule{13cm}{0.4pt}}
\newpage

\section{引言}

几千年来,最基本的数学对象之一—算术表达式,一直都是以一种离散的方式存在。数通过四则运算连接在一起,构成了算术表达式,
除了一些代数变形可以连接不同的表达式,两个表达式 $A$ 和 $B$ 之间没有好的、几何的连接方式。
那么究竟有没有算术表达式的几何化方式呢?在本文我们将会给出一个思路。

事情可以追溯到 2015 年底,当时我在上海短居。在考虑用词向量技术表达整数的时候,我发现可以构造一种有趣的、双曲平面上的离散结构。
当看到那些蜿蜒到无穷的折线和均匀的数值分布时,我意识到可能发现了一个有着丰富内涵的数学结构。其后的几年我一直在反复琢磨它。

2019 年 10 月底我去美国出差,11 月初在回国的飞机上,我偶然发现了这种构造方式可以推广到无穷小。
一两天后,我得到了双曲平面上无穷小生成结构的流方程。和中山大学的蒋文峰老师讨论后,他给到我很多好的建议,鼓励我一步步走下去。

2019 年底到 2020 年夏,我在努力把这个结果应用到一种重要的神经网络上—残差网络,可以得到一类特殊的网络架构来解决演化问题。
几个月的努力,逐步得到了一些有趣的神经网络架构,虽然在一些演化问题上它还没有达到工业界的最佳水平,但也有着良好的性能,
至少可以说明这种几何化的想法在实践中是可行的。这些应用角度的努力,也可以从理论角度给予解读,是在探索另外一种微积分的可能性。

2020 年底,在和英国的刘宇老师讨论后,他问了我几个特别好的问题,促使我重新思考,结果发现无穷小生成元的选择是自由的,
这样就导出了一个非常深刻的管型构造,这个管型构造能和多项式联系起来,是一种纤维结构。
但是这些几何构造还仅仅存在于设想中,它还欠缺一个严密的理论基础。

于是,在2021 年初,我从公司(彩云科技)争取了三个月的时间,在实习生张乐和张怀公的帮助下,我们一起找到了第一个严格的、
可以解算的实例(张乐的工作),也找到了一般情况下的表述方法(张怀公的工作)。

本文会沿着真实思路发展的线索,陈述这个算术表达式几何化的想法。

有一种理解,算术表达式的计算过程比它的计算结果包含了更为丰富的信息,简单说,就是“过程比结果更重要”。
在此,我也想表达,如果我们的历程是数字丛林里的一次探险,那么探险的事业是永远没有止境的,它的过程远比一个探险报告有趣的多。

\newpage

\section{词向量与数}

在计算语言学的研究中,从 1978 年 Salton 的 VSM 模型起 \cite{Almeida2019WordEA},人们为了方便计算,
常常把词汇表示成向量,并称这种表示为词嵌入。 2013 年 Mikolov 的论文 \cite{Mikolov2013EfficientEO} 发表后,
人们开始认识到,一类称为正则的嵌入方式是特别值得关注的,它们拥有一种组合性\cite{Mikolov2013DistributedRO}。
这里,我们把这种组合性归纳为:
\begin{enumerate}
\item 词和词之间的关系是有意义的
\item 同义的关系是平行的
\item 词和关系可以通过向量运算组合起来,形成一个格网
\end{enumerate}

\begin{figure}[ht]
\centering
\begin{tikzpicture}[x=0.5cm,y=0.5cm,z=0.3cm,>=stealth]
\draw[->] (xyz cs:x=-7.0) -- (xyz cs:x=7.0) node[above] {$x_0$};
\draw[->] (xyz cs:y=0) -- (xyz cs:y=7.0) node[right] {$x_n$};
\draw[->] (xyz cs:z=-7.0) -- (xyz cs:z=7.0) node[above] {$x_i$};

\node[fill,circle,inner sep=1.5pt,label={left:$king$}] (p) at (xyz cs:x=-3.0, y=3.0, z=-3.0) {};
\node[fill,circle,inner sep=1.5pt,label={right:$man$}] (q) at (xyz cs:x=2.0, y=-3.0, z=3.0) {};
\node[fill,circle,inner sep=1.5pt,label={left:$queen$}] (r) at (xyz cs:x=-3.0, y=3.0, z=3.0) {};
\node[fill,circle,inner sep=1.5pt,label={right:$woman$}] (s) at (xyz cs:x=2.0, y=-3.0, z=9.0) {};
\draw[dashed, blue] (p) -- (q);
\draw[dashed, blue] (r) -- (s);
\draw[dashed, red] (p) -- (r);
\draw[dashed, red] (q) -- (s);
\end{tikzpicture}
\caption{正则词嵌入的组合性}
\label{fig:compositionality}
\end{figure}

上图的示例可以简述为:男人、国王、女人、女王四个词形成一个平行四边形,其中有两对同义的关系
\begin{enumerate}
\item femalize:女性化的维度,从男人到女人、从国王到女王
\item royalize:皇权的维度,从男人到国王、女人到女王
\end{enumerate}

而几何上的平行四边形也可以通过$royalize$与$femalize$两个运算的可交换性来表达:
$$
    royalize(femalize(man)) = femalize(royalize(man))
$$

或者通过下面的交换子为零来表示
$$
    royalize(femalize(man)) - femalize(royalize(man)) = 0
$$

因为机器学习模型的统计特点,这里的组合性只是一种近似关系。机器学习界指出了组合性的存在,但并没有沿这个方向走太远。

整数和整数之间存在很多关系,但是如果我们把 $\cdot + 1$ 和 $\cdot \times 2$ 视为基本关系,我们能得到什么? 此时,我们发现:

$$
[(x + 1) \times 2] - (x \times 2 + 1) = 1
$$

这说明平行四边形永远无法存在。我们知道双曲空间是一种不存在平行四边形的空间,
简单尝试就可以把$\cdot + 1$关系、 $\cdot \times 2$关系和所有整数安排到双曲空间的一个网格上。

\begin{figure}[ht]
\centering
\includegraphics[width=4in]{images/assignment2.png}
\caption{$\cdot + 1$关系、 $\cdot \times 2$关系的赋值网格}
\end{figure}

这个网格的构造规则如下
\begin{enumerate}
    \item 网格呈现为四阶无限边形铺嵌
    \item 加乘轴相互垂直、交替出现
    \item 加轴、乘轴的增长方向按右手系排布
\end{enumerate}

我们需要论证依照此程序,可以无矛盾的构造几何构形,直至无限边形的无穷远处。

\begin{theorem}
    存在一组加轴、乘轴增长方向的右手系(或左手系)排布
\end{theorem}

\begin{proof}
对四阶无限边形铺嵌的节点、面染色
\begin{enumerate}
    \item 节点可以在黑、白两种选择中交错染色,且可以无矛盾地拓广至无穷远处;
    \item 面可以在红、蓝两种选择中交错染色,且可以无矛盾地拓广至无穷远处;
\end{enumerate}

\begin{figure}[ht]
\centering
\includegraphics[width=4in]{images/H2_tiling_24i-1.png}
\caption{染色与定向}
\end{figure}

在上述染色的基础上,我们对边可以依照下述规则形成右手系排布:

\end{proof}

\newpage

\section{代数的观点}

从数的角度讲,加减乘除四则运算的代数结构是域;但是从算术表达式的角度讲,四则运算对应到的代数结构是群。
由此引向算术表达式的路径群与典范形式的讨论。

对数域 $\mathcal{F}$,取 $\mu, \lambda \in \mathcal{F}$,我们考虑表达式集合 $E(\mu, \lambda)$ 如下自由产生
\begin{itemize}
    \item 初始值: $0$
    \item 运算符: $\odot: x \mapsto x$
    \item 运算符: $\oplus_\mu: x \mapsto x + \mu$
    \item 运算符: $\ominus_\mu: x \mapsto x - \mu$
    \item 运算符: $\otimes_\lambda: x \mapsto x \cdot e^\lambda$
    \item 运算符: $\oslash_\lambda: x \mapsto x / e^\lambda$
\end{itemize}

这里 $\mu$ 是加法生成元,而 $\lambda$ 是乘法生成元。在上下文清晰的地方,我们会从运算符的下标里省略 $\mu$ 和 $\lambda$。

采纳了不同的 $E(\mu, \lambda)$ 上的等值关系,我们会得到不同的结构:
\begin{itemize}
    \item 字面结构:根据表达式字面上是否相等
    \item 语法结构:根据表达式在特定公理下是否相等,比如加减互逆、乘除互逆、分配律等
    \item 语义结构:根据表达式运算结果的值否相等
\end{itemize}

最后,我们指出,引入态射后,这些不同的结构之间可以构成范畴,而字面结构是初始对象,语义结构是终止对象。

\subsection{字面结构}\label{subsec:literial}

当我们采纳了字面上的等值关系,我们得到了一个出度为 5 的生成树。它对应一个六阶无限边形铺嵌,是否有对应右手系下的符号排布,是个有趣的问题。

\begin{itemize}
    \item 根:初始操作子 $0$
    \item 边:每个节点通过对应的边 $\odot$、$\oplus$、$\ominus$、$\otimes$和$\oslash$,生成 $5$ 个子节点。
\end{itemize}

从根节点到任意节点的一条路径对应一个算术表达式,不同的表达式对应不同的路径。

\begin{figure}[ht]
\centering
\Tree
[.0
    [.$\odot$   [.$\odot$ ] [.$\oplus$ ] [.$\ominus$ ] [.$\otimes$ ] [.$\oslash$ ] ]
    [.$\oplus$  [.$\odot$ ] [.$\oplus$ ] [.$\ominus$ ] [.$\otimes$ ] [.$\oslash$ ] ]
    [.$\ominus$ [.$\odot$ ] [.$\oplus$ ] [.$\ominus$ ] [.$\otimes$ ] [.$\oslash$ ] ]
    [.$\otimes$ [.$\odot$ ] [.$\oplus$ ] [.$\ominus$ ] [.$\otimes$ ] [.$\oslash$ ] ]
    [.$\oslash$ [.$\odot$ ] [.$\oplus$ ] [.$\ominus$ ] [.$\otimes$ ] [.$\oslash$ ] ]
]
\caption{字面结构下的生成树}
\end{figure}

Suppose we have a series of operators $a_1, a_2, \cdots a_{n-1}, a_n$, we introduce a \emph{path notion}.

\begin{definition}
\label{definition:path}
    a path of operators $a_1 a_2 \cdots a_{n-1} a_n$ is a function over $\mathcal{F}$ which apply operators on an operand in turn
    $$a_1 a_2 \cdots a_{n-1} a_n (x) \coloneqq a_n( a_{n-1}( \cdots a_2( a_1(x) ) \cdots ) ), \forall x \in \mathcal{F}$$
\end{definition}

Any expressions in $E(\mu, \lambda)$ can be written as a path with operand $0$. And now we verify the operators is associative inside a path.

\begin{lemma}
\label{lemma:associative}
    The operators inside a path is associative, i.e. $$x (y z) = (x y) z$$
\end{lemma}

\begin{proof}
    Follow the definition, we have
$$(x (y z)) [\alpha] = (y z) (x[\alpha]) = z(y(x[\alpha]))$$
$$((x y) z) [\alpha] = z ((x y)[\alpha]) = z(y(x[\alpha]))$$
\qedhere
\end{proof}

\begin{definition}
\label{definition:concatenate}
    Concatenate of paths $p_1 \cdot p_2$ is defined as the composite of functions
    $$p_1 \cdot p_2 \coloneqq p_2 \circ p_1 $$
\end{definition}

\begin{theorem}
\label{theorem:semigroup}
the algebraic structure $\mathcal{P} = (E, \cdot)$ is a semigroup.
\end{theorem}

\subsection{Syntactical structure}\label{subsec:syntactical}

We can introduce the inverse notion
\begin{itemize}
    \item $\oplus^{-1} = \ominus$
    \item $\ominus^{-1} = \oplus$
    \item $\otimes^{-1} = \oslash$
    \item $\oslash^{-1} = \otimes$
    \item $\odot^{-1} = \odot$
\end{itemize}

Then we have the calculation rule of inverse.

\begin{lemma}
\label{lemma:inverserule}
If we have
$$\beta = \alpha a_1 a_2 \cdots a_{n-1} a_n$$
then
$$\alpha = \beta a_n^{-1} a_{n-1}^{-1} \cdots a_2^{-1} a_1^{-1}$$
\end{lemma}

\begin{proof}
Notice that $\oplus, \ominus, \otimes, \oslash$ are all bijections, we know that $\alpha a_1 a_2 \cdots a_{n-1} a_n$ is a composition of functions

$$\beta = a_n( a_{n-1}( \cdots a_2( a_1(\alpha) ) \cdots ) )$$

Considering the inverse of a composition, we have

$$\alpha = a_1^{-1}( a_2^{-1}( \cdots a_{n-1}^{-1}( a_n^{-1}(\beta) ) \cdots ) )$$

Or under the path notion

$$\alpha = \beta a_n^{-1} a_{n-1}^{-1} \cdots a_2^{-1} a_1^{-1}$$

\qedhere
\end{proof}

Informally we can divide a long calculation into pieces, or composite small pieces of calculations into a longer one,
for example

$$\gamma = 0 a_1 a_2 \cdots a_{n-1} a b_1 b2 \cdots b_{m-1} b_m$$

can be rewritten as

$$\gamma = 0 ((\odot a_1 a_2 \cdots a_{n-1} a) \circ (\odot b_1 b_2 \cdots b_{m-1} b_m))$$

And if we treat $\odot$ and $0$ as the same, and we define

$$\alpha = 0 a_1 a_2 \cdots a_{n-1} a$$
$$\beta = 0 b_1 b2 \cdots b_{m-1} b_m$$

We have

$$\gamma = \alpha \circ \beta$$

where $\alpha$, $\beta$ and $\gamma$ are all well defined in $E$.

Under such definition, the algebraic structure $\mathcal{P} = (E, \circ)$ is a group, and we call it \emph{path group}.

\begin{theorem}
\label{theorem:pathgroup}
the algebraic structure $\mathcal{P} = (E, \circ)$ is a group
\end{theorem}

\begin{proof}
    Follow the definition of group, we have
\qedhere
\end{proof}

The cayley tree of path group $\mathcal{P}$ is a tree structure as below

\begin{figure}[ht]
\centering
\includegraphics[width=3in]{images/cayley}
\caption{Cayley tree of syntactical structure}
\end{figure}

\begin{figure}[ht]
\centering
\begin{tikzpicture}

\filldraw [black] (0,0) circle (2pt) node[align=center, below] {0};
\filldraw [black] (4,1) circle (2pt) node[align=center, below] {$\alpha$};
\filldraw [black] (2,3) circle (2pt) node[align=center, below] {$\beta$};
\filldraw [black] (5,3) circle (2pt) node[align=center, below] {$\gamma$};
\filldraw [black] (-2,2) circle (2pt) node[align=center, below] {$\psi$};
\filldraw [black] (1,2) circle (2pt) node[align=center, below] {$\phi$};
\filldraw [gray] (3,-0.4) circle (0pt) node[align=center, below] {$x$};
\filldraw [gray] (3,1.8) circle (0pt) node[align=center, below] {$y$};
\filldraw [gray] (4.7,2) circle (0pt) node[align=center, below] {$z$};
\draw [gray] (0,0) to[out=300,in=240] (4,1);
\draw [gray] (4,1) to[out=120,in=300] (2,3);
\draw [gray] (4,1) to[out=60,in=240] (5,3);
\draw [gray] (0,0) to[out=120,in=300] (-2,2);
\draw [gray] (0,0) to[out=60,in=240] (1,2);
\filldraw [gray] (-1,0.8) circle (0pt) node[align=center, below] {$y$};
\filldraw [gray] (0.7,1) circle (0pt) node[align=center, below] {$z$};

\end{tikzpicture}
\caption{Factorization by greatest common divisor}
\end{figure}

The generating order is a partial order over $E$, we assume the later generated one is greater in generating.
By the generating order, we can define the concept of greatest common divisor: for any $\beta, \gamma \in A$,
the greatest common divisor $gcd(\beta, \gamma)$ is the greatest common ancestor in the generating tree.
So coprime of $\psi, \phi \in A$ means $gcd(\psi, \phi) = 0$

\begin{lemma}
\label{lemma:coprimes}
When $\alpha = gcd(\beta, \gamma)$, and $\beta = \alpha \circ \psi$, $\gamma = \alpha \circ \phi$ hold, then $\psi$ and $\phi$ are coprimes.
\end{lemma}

\subsection{Semantical structure}\label{sec:semantical}

Now we study the evaluation result, and the structure under the evaluation equality.
Giving an element of $E(\mu, \lambda)$ with a generating path $0 a_1 a_2 \cdots a_n$,
and the evaluation can be expressed by a function $\nu: E(\mu, \lambda) \to R$,
and we can proof by structure induction that all evaluation result keep the form

$$
\nu(0 a_1 a_2 \cdots a_n) = \sum_{i} k_i \mu e^{l_i \lambda} \in R
$$
where $k_i \in N, l_i \in N$, and obviously $\sum_{i} k_i \mu e^{l_i \lambda}$ is an evaluation of a polynomial over $\mathcal{F}$ at $e^{\lambda}$.
Comparing with path notion, we call this an expanded notion, or evaluated notion if we emphasising on evaluated value.

\begin{lemma}
\label{lemma:arithmeticalalgebra}
All arithmetical expressions $E(\mu, \lambda)$ is an algebra over numbers field $\mathcal{F}$, or $\mathcal{A} = (E, \mathcal{F}, \cdot)$.
\end{lemma}

\begin{proof}
By using the expanded notion, it is easy to verify the three axioms of an algebra is hold:
for any arithmetical expressions $x, y, z \in E$ and $a, b \in \mathcal{F}$, we have
\begin{itemize}
    \item right distributivity: $(x + y) \cdot z = x \cdot z + y \cdot z$
    \item left distributivity: $z \cdot (x + y) = z \cdot x + z \cdot y$
    \item compatibility with scalars: $(ax) \cdot (by) = (ab) (x \cdot y)$
\end{itemize}
\qedhere
\end{proof}

Since exponent appears in multiplication generators, we now study an equivalent formulation of Lindemann-Weierstrass theorem

\begin{theorem}
\label{theorem:LindemannWeierstrass}
If $\alpha_1, \cdots, \alpha_m$ are distinct algebraic numbers, then $e^{\alpha_1}, \cdots, e^{\alpha_m}$ are linearly independent over the algebraic numbers.
\end{theorem}

This theorem gives the transcendental nature of $e$.
Now we can verify path group $\mathcal{P}$ is non-abelian by calculating the result at point $0$ when the field $\mathcal{F}$ is algebraic.

\begin{corollary}
\label{corollary:nonabelian}
The path group $\mathcal{P} = (E, \circ)$ is non-abelian when the field $\mathcal{F}$ is algebraic
\end{corollary}

\begin{proof}
Because of Lindemann-Weierstrass theorem, $\mu$ have a covering structure and each layer of $\nu$ is a injectiion to the codomain.
Notice that $(0 + \mu) \cdot e^\lambda \neq 0 \cdot e^\lambda + \mu$, then we have $0 \oplus \otimes \neq 0 \otimes \oplus$
\qedhere
\end{proof}


最后,我们指出交换子的值和空间的弯曲程度相关。

\section{流方程}

假设在一个无穷小范围内存在算术表达式生成结构的一种几何,它会是什么样呢?本节我们探讨这个话题。

算术表达式空间中一点 $p$,$p$ 作为表达式有估值 $x$,假设在 $p$ 点存在加轴和乘轴构成的标架 $F_i$,
$p$ 有一个运动,在标架 $F_i$ 中表现为速度 $u$ 和角度 $\theta$ , 我们对 $p$ 的运动带来的估值 $x$ 的变化,
通过下述推导,可以建立方程

\begin{equation}
    x_{\delta} = (x_0 + \mu \epsilon \cos \theta)e^{\lambda \epsilon \sin \theta}
\end{equation}

或者

\begin{equation}
    x_{\delta} = x_0 e^{\lambda \epsilon \sin \theta} + \mu \epsilon \cos \theta
\end{equation}

两者都可以简化为

\begin{equation}
    x_{\delta} = x_0 + \epsilon (x_0 \lambda \sin \theta + \mu \cos \theta)
\end{equation}

于是

\begin{equation}
    \frac{1}{\delta} (x_{\delta} - x_0) = \frac{\epsilon}{\delta} (\mu \cos \theta + x_0 \lambda \sin \theta)
\end{equation}

当 $\delta$ 和 $\epsilon$ 同时趋于零时,我们就得到了 $dx / dt$,即有

\begin{equation}
    \frac{dx}{dt} = u (\mu \cos \theta + x \lambda \sin \theta)
\end{equation}

最后,我们指出流方程的推导是抽象的,我们在推导过程中并没有用上前面章节里具体的几何结构。

\newpage

\section{管型构造}

\newpage

\section{新的实例}

\newpage

\section{可能的框架}

\newpage
\phantomsection
\addcontentsline{toc}{section}{参考文献}
\bibliographystyle{ieeetr}
\bibliography{biblio/article}

\end{document}


