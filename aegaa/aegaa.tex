\documentclass[a4paper,12pt]{article}
\usepackage{datetime}
\usepackage{framed}
\usepackage{enumitem}
\usepackage{fancyref}
\usepackage{wrapfig}
\usepackage{pifont}
\usepackage{appendix}
\usepackage{caption}
\usepackage{xcolor}
\usepackage[stable]{footmisc}
\usepackage{multicol}
\usepackage{csquotes}
\usepackage{pdfpages}

\usepackage{amsmath}
\usepackage{amsthm}
\usepackage{amssymb}
\usepackage{amsfonts}
\usepackage{mathtools}

\usepackage{tikz}
\usepackage{pgf}
\usepgflibrary{fpu}
\usetikzlibrary{angles,fit,arrows,calc,math,intersections,through,backgrounds,cd}
\usepackage{pgfplots}
\usepackage{tkz-euclide}

\usepackage{qtree}

\usepackage{listings}
\lstset{
  basicstyle=\itshape,
  xleftmargin=3em,
  literate={->}{$\rightarrow$}{2}
           {α}{$\alpha$}{1}
           {δ}{$\delta$}{1}
}

\usepackage{csquotes}
\renewcommand{\mkbegdispquote}[2]{\itshape}

\newdateformat{nianyueri}{修订于 \THEYEAR 年 \THEMONTH 月 \THEDAY 日 }

\usepackage{xstring}
\usepackage{catchfile}
\CatchFileDef{\HEAD}{../.git/refs/heads/master}{}
\newcommand{\gitrevision}{%
  \StrLeft{\HEAD}{7}%
}

\usepackage{stmaryrd}
\usepackage{mathtools}
\usepackage{wasysym}
\usepackage{textcomp}
\usepackage{blindtext}
\usepackage{subfiles}

\usepackage{data/quiver}
\usepackage{data/circledsteps}
\usepackage[top=1in,bottom=1in,left=1in,right=1in]{geometry} % 用于设置页面布局
\usepackage{xeCJK} % 用于使用本地字体
\usepackage[super, square, sort&compress]{natbib} % 处理参考文献
\usepackage{titlesec, titletoc} % 设置章节标题及页眉页脚
%\usepackage{xCJKnumb} % 中英文数字转换
\usepackage{amssymb}
\usepackage{amsmath} % 在公式中用\text{文本}输入中文
\usepackage{diagbox}
\usepackage{multirow} % 表格中使用多行
\usepackage{booktabs} % 表格中使用\toprule等命令
\usepackage{rotating} % 使用sidewaystable环境旋转表格
\usepackage{tabularx}
\usepackage{graphicx} % 处理图片
\usepackage{footnote} % 增强的脚注功能,可添加表格脚注
\usepackage{threeparttable} % 添加真正的表格脚注,示例见README
\usepackage{hyperref} % 添加pdf书签

\usepackage{tikz}
\usetikzlibrary{shapes,arrows,shadows}

% 字体设置
\setmainfont{Times New Roman}
\setsansfont[Scale=MatchLowercase,Mapping=tex-text]{PT Sans}
\setmonofont[Scale=MatchLowercase]{PT Mono}
\setCJKmainfont[ItalicFont={Kaiti SC}, BoldFont={Heiti SC}]{Songti SC}
\setCJKsansfont{Heiti SC}
\setCJKmonofont{Songti SC}
% \setCJKmainfont[BoldFont={FZXiaoBiaoSong-B05S}]{Songti SC}
% \setCJKfamilyfont{kai}[BoldFont=Heiti SC]{Kaiti SC}
% \setCJKfamilyfont{song}[BoldFont=Heiti SC]{Songti SC}
% \setCJKfamilyfont{hei}[BoldFont=Heiti SC]{Heiti SC}
% \setCJKfamilyfont{fsong}[BoldFont=Heiti SC]{Songti SC}
% \newcommand{\kai}[1]{{\CJKfamily{kai}#1}}
% \newcommand{\hei}[1]{{\CJKfamily{hei}#1}}
% \setromanfont[Mapping=tex-text]{TeXGyrePagella}
% \setsansfont[Scale=MatchLowercase,Mapping=tex-text]{TeXGyrePagella}
% \setmonofont[Scale=MatchLowercase]{Courier New}
%%设置常用中文字号,方便调用
\newcommand{\erhao}{\fontsize{22pt}{\baselineskip}\selectfont}
\newcommand{\xiaoerhao}{\fontsize{18pt}{\baselineskip}\selectfont}
\newcommand{\sanhao}{\fontsize{16pt}{\baselineskip}\selectfont}
\newcommand{\xiaosanhao}{\fontsize{15pt}{\baselineskip}\selectfont}
\newcommand{\sihao}{\fontsize{14pt}{\baselineskip}\selectfont}
\newcommand{\xiaosihao}{\fontsize{12pt}{\baselineskip}\selectfont}
\newcommand{\wuhao}{\fontsize{10.5pt}{\baselineskip}\selectfont}
\newcommand{\xiaowuhao}{\fontsize{9pt}{\baselineskip}\selectfont}
\newcommand{\liuhao}{\fontsize{7.5pt}{\baselineskip}\selectfont}

% 章节标题显示方式及页眉页脚设置
% \item xCJKnumb是自己额外安装的包
% \item titleformat命令定义标题的形式
% \item titlespacing定义标题距左、上、下的距离
\titleformat{\section}{\raggedright\large\bfseries}{\thesection}{1em}{}
\titleformat{\subsection}{\raggedright\normalsize\bfseries}{\thesubsection}{1em}{}
\titlespacing{\section}{0pt}{*0}{*2}
\titlespacing{\subsection}{0pt}{*0}{*1}
% 由于默认的2em缩进不够,所以我手动调整了,但是在windows下似乎2.2就差不多了,或者是article中没有这个问题
\setlength{\parindent}{2.2em}

% 设置表格标题前后间距
\setlength{\abovecaptionskip}{0pt}
\setlength{\belowcaptionskip}{0pt}


\renewcommand{\refname}{\bfseries{参~考~文~献}} %将Reference改为参考文献(用于 article)
% \renewcommand{\bibname}{参~考~文~献} %将bibiography改为参考文献(用于 book)
\renewcommand{\baselinestretch}{1.38} %设置行间距
\renewcommand{\figurename}{\small\ttfamily 图}
\renewcommand{\tablename}{\small\ttfamily 表}


\newtheorem{problem}{问题}
\numberwithin{problem}{section}
\newtheorem{definition}{定义}
\numberwithin{definition}{section}
\newtheorem{lemma}{引理}
\numberwithin{lemma}{section}
\newtheorem{proposition}{命题}
\numberwithin{proposition}{section}
\newtheorem{theorem}{定理}
\numberwithin{theorem}{section}
\newtheorem{grammar}{文法}
\numberwithin{grammar}{section}
\newtheorem{program}{程序}
\numberwithin{program}{section}
\newtheorem{convention}{约定}
\numberwithin{convention}{section}
\newtheorem{corollary}{推论}
\numberwithin{corollary}{section}
\renewcommand*{\proofname}{证明}

\xeCJKsetwidth{‘’“”}{1em}

\makeatletter
\DeclareRobustCommand{\Hyp}[1]{\mathpunct{{\LeftP}{#1}{\RightP}}}
\newcommand{\LeftP}{%
  \setlength{\unitlength}{\fontcharht\font`T}%
  \begin{picture}(1,1)
  \Line(1,1)(0.8,1)(0.8,0)(1,0)
  \end{picture}%
}
\newcommand{\RightP}{%
  \setlength{\unitlength}{\fontcharht\font`T}%
  \begin{picture}(1,1)
  \Line(0,1)(0.2,1)(0.2,0)(0,0)
  \end{picture}%
}
\makeatother

\makeatletter
\DeclareRobustCommand{\Intg}[1]{\mathbin{{\UpperP}{#1}{\LowerP}}}
\newcommand{\UpperP}{%
  \setlength{\unitlength}{\fontcharht\font`T}%
  \begin{picture}(1,1)
  \Line(0.8,1)(0.8,1.2)(1.8,1.2)(1.8,1)
  \end{picture}%
}
\newcommand{\LowerP}{%
  \setlength{\unitlength}{\fontcharht\font`T}%
  \begin{picture}(1,1)
  \Line(-0.9,0)(-0.9,-0.2)(0.1,-0.2)(0.1,0)
  \end{picture}%
}
\makeatother

\title{算术表达式几何与分析}
\date{\nianyueri\today}
\author{苑明理}

\begin{document}

\begin{center}
  \sihao \em 过程比结果更重要
\end{center}

\begingroup
\let\newpage\relax
\maketitle
\endgroup

\centerline{\rule{13cm}{0.4pt}}
\renewcommand{\contentsname}{\hfill\bfseries 目录\hfill}
\setcounter{tocdepth}{2}
\tableofcontents
\centerline{\rule{13cm}{0.4pt}}

\newpage
\begin{displayquote}
在这个(数制发明的)历史过程里,人们遗忘了一个隐藏的巨大世界—那就是表达式,或者说计算过程。
数作为一种表征,本来就是计算过程的高度凝练的表达,但它一旦符号化后,就成为我们追求的目标。
于是,我们往往关注计算过程的结果,而忽视了过程本身蕴含的众多信息。过程比结果更重要。
\end{displayquote}
\newpage

\section{超算子的引入}

\subsection{一阶运算}

\begin{itemize}
  \item 一阶单位:$0$
  \item 左顺:$a \Hyp{+^1} b = a + b$,即左加
  \item 右顺:$a \Hyp{{}^1+} b = b + a$,即右加
  \item 左违:$a \Hyp{-^1} b = a - b$,即左减
  \item 右违:$a \Hyp{{}^1-} b = b - a$,即右减
  \item 左正:$\Hyp{+^1} b = + b$,零元在左
  \item 右正:$a \Hyp{{}^1+} = + a$,零元在右
  \item 左负:$\Hyp{-^1} b = - b$,零元在左
  \item 右负:$a \Hyp{{}^1-} = - a$,零元在右
\end{itemize}

顺违相抵
$$a \Hyp{+^1} b \Hyp{-^1} b = a$$
$$a \Hyp{{}^1+} b \Hyp{{}^1-} b = a$$

正负相消
$$\Hyp{-^1} b \Hyp{+^1} \Hyp{+^1} b = 0$$
$$b \Hyp{{}^1+} \Hyp{{}^1+}  b \Hyp{{}^1-} = 0$$

左右(违)互逆

$$ \Hyp{-^1} ( a \Hyp{-^1} b ) = a \Hyp{{}^1-} b $$
$$ ( a \Hyp{{}^1-} b ) \Hyp{{}^1-} = a \Hyp{-^1} b $$
$$ \Hyp{-^1} ( a \Hyp{{}^1-} b ) = a \Hyp{-^1} b $$
$$ ( a \Hyp{-^1} b ) \Hyp{{}^1-} = a \Hyp{{}^1-} b $$

\subsection{二阶运算}

\begin{itemize}
  \item 二阶单位:$1$
  \item 左顺:$a \Hyp{+^2} b = ab$,即左乘
  \item 右顺:$a \Hyp{{}^2+} b = ba$,即右乘
  \item 左违:$a \Hyp{-^2} b = a / b$,即左除
  \item 右违:$a \Hyp{{}^2-} b = b / a$,即右除
  \item 左正:$\Hyp{+^2} b = b$,幺元在左
  \item 右正:$a \Hyp{{}^2+} = a$,幺元在右
  \item 左负:$\Hyp{-^2} b = - b$,幺元在左,右侧数的倒数
  \item 右负:$a \Hyp{{}^2-} = - a$,幺元在右,左侧数的倒数
\end{itemize}

顺违相抵、正负相消、左右互逆依然成立

\subsection{三阶运算}

\begin{itemize}
  \item 三阶单位:左幺 $1$,右幺 $?$,
  \item 左顺:$a \Hyp{+^3} b = a^b$,即左侧乘方
  \item 右顺:$a \Hyp{{}^3+} b = b^a$,即右侧乘方
  \item 左违:$a \Hyp{-^3} b = \sqrt[b]{a}$,即开方
  \item 右违:$a \Hyp{{}^3-} b = \log_b a$,即对数
  \item 左正:$\Hyp{+^3} b = - b$,幺元在左,右侧数的倒数
  \item 右正:$a \Hyp{{}^3+} = - a$,幺元在右,左侧数的倒数
  \item 左负:$\Hyp{-^3} b = - b$,幺元在左,右侧数的倒数
  \item 右负:$a \Hyp{{}^3-} = - a$,幺元在右,左侧数的倒数
\end{itemize}

顺违相抵依然成立

\subsection{恒等式}

\subsubsection{分配律}

同底的乘方相乘,等价于该底上的指数相加

$$a \Hyp{+^3} b \Hyp{+^2} a \Hyp{+^3} c = a \Hyp{+^3} (b \Hyp{+^1} c)$$

本质上和加乘的分配律是一样的

$$a \Hyp{+^2} b \Hyp{+^1} a \Hyp{+^2} c = a \Hyp{+^2} (b \Hyp{+^1} c)$$

所以猜测有一般化的分配律

$$a \Hyp{+^{k+1}} b \Hyp{+^k} a \Hyp{+^{k+1}} c = a \Hyp{+^{k+1}} (b \Hyp{+^1} c)$$

\subsubsection{简化律}

对数把乘法化加减

$$(a \Hyp{{}^2+} b) \Hyp{{}^3-} c = a \Hyp{{}^3-} c \Hyp{{}^1+} b \Hyp{{}^3-} c$$

所以猜测有一般化的简化律

$$(a \Hyp{{}^{k+1}+} b) \Hyp{{}^{k+2}-} c = a \Hyp{{}^{k+2}-} c \Hyp{{}^1+} b \Hyp{{}^{k+2}-} c$$

\section{超算子的算术}

\section{不定式与微积分}

微商:
积分:

几何观点下的分析学,表达式空间边界上的代数与几何

\section{算术表达式几何}

\subsection{算术表达式}

\section{超越算术表达式几何}

\subsection{路径几何空间}

\end{document}
