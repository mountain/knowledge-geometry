\documentclass[a4paper,12pt]{article}
\usepackage{amsmath, amsthm}
\usepackage{datetime}
\usepackage{framed}
\usepackage{enumitem}
\usepackage{fancyref}
\usepackage{wrapfig}
\usepackage{pifont}
\usepackage{appendix}
\usepackage{caption}
\usepackage{xcolor}
\usepackage[stable]{footmisc}
\usepackage{multicol}

\usepackage{amsthm}
\usepackage{amssymb}
\usepackage{amsfonts}
\usepackage{amsmath}
\usepackage{mathtools}

\usepackage{tikz}
\usepackage{pgf}
\usepgflibrary{fpu}
\usepackage{qtree}
\usetikzlibrary{angles,fit,arrows,calc,math,intersections,through,backgrounds}
\usepackage{pgfplots}
\usepackage{tkz-euclide}

\usepackage{csquotes}
\renewcommand{\mkbegdispquote}[2]{\itshape}

\newdateformat{nianyueri}{ \THEYEAR 年 \THEMONTH 月 \THEDAY 日 }

\usepackage{xstring}
\usepackage{catchfile}
\CatchFileDef{\HEAD}{../.git/refs/heads/master}{}
\newcommand{\gitrevision}{%
  \StrLeft{\HEAD}{7}%
}

\usepackage{data/quiver}
\usepackage{data/circledsteps}
\usepackage[top=1in,bottom=1in,left=1in,right=1in]{geometry} % 用于设置页面布局
\usepackage{xeCJK} % 用于使用本地字体
\usepackage[super, square, sort&compress]{natbib} % 处理参考文献
\usepackage{titlesec, titletoc} % 设置章节标题及页眉页脚
%\usepackage{xCJKnumb} % 中英文数字转换
\usepackage{amssymb}
\usepackage{amsmath} % 在公式中用\text{文本}输入中文
\usepackage{diagbox}
\usepackage{multirow} % 表格中使用多行
\usepackage{booktabs} % 表格中使用\toprule等命令
\usepackage{rotating} % 使用sidewaystable环境旋转表格
\usepackage{tabularx}
\usepackage{graphicx} % 处理图片
\usepackage{footnote} % 增强的脚注功能,可添加表格脚注
\usepackage{threeparttable} % 添加真正的表格脚注,示例见README
\usepackage{hyperref} % 添加pdf书签

\usepackage{tikz}
\usetikzlibrary{shapes,arrows,shadows}

% 字体设置
\setmainfont{Times New Roman}
\setsansfont[Scale=MatchLowercase,Mapping=tex-text]{PT Sans}
\setmonofont[Scale=MatchLowercase]{PT Mono}
\setCJKmainfont[ItalicFont={Kaiti SC}, BoldFont={Heiti SC}]{Songti SC}
\setCJKsansfont{Heiti SC}
\setCJKmonofont{Songti SC}
% \setCJKmainfont[BoldFont={FZXiaoBiaoSong-B05S}]{Songti SC}
% \setCJKfamilyfont{kai}[BoldFont=Heiti SC]{Kaiti SC}
% \setCJKfamilyfont{song}[BoldFont=Heiti SC]{Songti SC}
% \setCJKfamilyfont{hei}[BoldFont=Heiti SC]{Heiti SC}
% \setCJKfamilyfont{fsong}[BoldFont=Heiti SC]{Songti SC}
% \newcommand{\kai}[1]{{\CJKfamily{kai}#1}}
% \newcommand{\hei}[1]{{\CJKfamily{hei}#1}}
% \setromanfont[Mapping=tex-text]{TeXGyrePagella}
% \setsansfont[Scale=MatchLowercase,Mapping=tex-text]{TeXGyrePagella}
% \setmonofont[Scale=MatchLowercase]{Courier New}
%%设置常用中文字号,方便调用
\newcommand{\erhao}{\fontsize{22pt}{\baselineskip}\selectfont}
\newcommand{\xiaoerhao}{\fontsize{18pt}{\baselineskip}\selectfont}
\newcommand{\sanhao}{\fontsize{16pt}{\baselineskip}\selectfont}
\newcommand{\xiaosanhao}{\fontsize{15pt}{\baselineskip}\selectfont}
\newcommand{\sihao}{\fontsize{14pt}{\baselineskip}\selectfont}
\newcommand{\xiaosihao}{\fontsize{12pt}{\baselineskip}\selectfont}
\newcommand{\wuhao}{\fontsize{10.5pt}{\baselineskip}\selectfont}
\newcommand{\xiaowuhao}{\fontsize{9pt}{\baselineskip}\selectfont}
\newcommand{\liuhao}{\fontsize{7.5pt}{\baselineskip}\selectfont}

% 章节标题显示方式及页眉页脚设置
% \item xCJKnumb是自己额外安装的包
% \item titleformat命令定义标题的形式
% \item titlespacing定义标题距左、上、下的距离
\titleformat{\section}{\raggedright\large\bfseries}{\thesection}{1em}{}
\titleformat{\subsection}{\raggedright\normalsize\bfseries}{\thesubsection}{1em}{}
\titlespacing{\section}{0pt}{*0}{*2}
\titlespacing{\subsection}{0pt}{*0}{*1}
% 由于默认的2em缩进不够,所以我手动调整了,但是在windows下似乎2.2就差不多了,或者是article中没有这个问题
\setlength{\parindent}{2.2em}

% 设置表格标题前后间距
\setlength{\abovecaptionskip}{0pt}
\setlength{\belowcaptionskip}{0pt}


\renewcommand{\refname}{\bfseries{参~考~文~献}} %将Reference改为参考文献(用于 article)
% \renewcommand{\bibname}{参~考~文~献} %将bibiography改为参考文献(用于 book)
\renewcommand{\baselinestretch}{1.38} %设置行间距
\renewcommand{\figurename}{\small\ttfamily 图}
\renewcommand{\tablename}{\small\ttfamily 表}


\usepackage{stmaryrd}
\usepackage{mathtools}
\usepackage{wasysym}
\usepackage{textcomp}
\usepackage{subfiles}

\newtheorem{definition}{定义}
\numberwithin{definition}{section}
\newtheorem{lemma}{引理}
\numberwithin{lemma}{section}
\newtheorem{proposition}{命题}
\numberwithin{proposition}{section}
\newtheorem{theorem}{定理}
\numberwithin{theorem}{section}
\newtheorem{grammar}{文法}
\numberwithin{grammar}{section}
\newtheorem{program}{程序}
\numberwithin{program}{section}
\newtheorem{convention}{约定}
\numberwithin{convention}{section}
\newtheorem{corollary}{推论}
\numberwithin{corollary}{section}
\newtheorem{principle}{原理}
\numberwithin{principle}{section}
\renewcommand*{\proofname}{证明}

\xeCJKsetwidth{‘’“”}{1em}

\title{如一的组合子逻辑及其几何}
\date{\nianyueri\today}
\author{苑明理}

\begin{document}

\begingroup
\let\newpage\relax
\maketitle
\centerline{ 版本号:\gitrevision }
\endgroup

\centerline{\rule{13cm}{0.4pt}}
\renewcommand{\contentsname}{\hfill\bfseries 目录\hfill}
\setcounter{tocdepth}{2}
\tableofcontents
\centerline{\rule{13cm}{0.4pt}}
\newpage

\section{二叉树与 Catalan 数}

我们递归的定义二叉树结构 $T$ 如下:
\begin{itemize}
  \item $T \leftarrow () $,如此的项称为叶节点
  \item $T \leftarrow (a \hspace{0.6em} b), a \in T, b \in T$
\end{itemize}

二叉树有多种表示方法,可以简单的图示如下

\Tree [. $a$ $b$ ]

或者通过文本的来表示

\begin{itemize}
  \item 1个叶节点:()
  \item 2个叶节点:()()、(())
  \item 3个叶节点:()()()、()(())、(())()、(()())、((()))
  \item ...
\end{itemize}

如上的文本被称为 Dyck 字,它也可以用左右括号的恰当匹配来刻画。一个自然的问题是 Dyck 字如何生成?又如何计数?
为此我们首先引入 Dyck 字的文法:
\begin{grammar}
  $\mathbb{T} \leftarrow \epsilon | (\mathbb{T})\mathbb{T}$
\end{grammar}

由这个文法引出的递归关系,很容易导出如下的计数序列,被称为 Catalan 数。

\begin{align}
C_0 & = 1 \\
C_1 & = 1 \\
C_{n+1} & = \sum_{i=0}^n C_i C_{n-i}
\end{align}

和如下完全同构的递归构造过程

\begin{align}
\mathbb{T}_0 & = {\epsilon} \\
\mathbb{T}_1 & = { () } \\
\mathbb{T}_{n+1} & = \cup_{i=0}^n (\mathbb{T}_i) \mathbb{T}_{n-i}
\end{align}

这里的乘法被解释为字串的连接操作,$\epsilon$ 是空字串。

\section{组合子逻辑}

我们可以考察二叉树集合$T$上的自映射,对这个映射的研究构成了组合子逻辑研究的基本背景。

下面我们不严格的引入组合子,简单来理解,可以认为组合子是一些特定的组合单元,每个组合单元可以实现一种二叉树上的简单映射;
然后再把不同的组合单元组装起来,可以构造非常复杂的二叉树上的映射。

$I$ 组合子


$K$ 组合子


$S$ 组合子



\section{Catalan 群与$\iota$组合子}

Catalan 群 $\mathbb{K}_n$ 是给定了 $n$ 个叶节点的树的自映射群,即有
$$
\mathbb{K}_n = \{ \alpha | \alpha \in \mathbb{T}_n \to \mathbb{T}_n \}
$$

我们要指出在 $\iota$ 组合子的帮助下

$$
\mathbb{K}_0 \leftarrow \mathbb{K}_1 \leftarrow \mathbb{K}_2 \cdots
$$

如上构成一个正合列。

\section{Catalan 群的几何}

\section{如一的组合子逻辑}

二叉树的叶节点上操作的是几何体,$(a b)$ 读作把 $a$、$b$ 在 $a$ 上寻找和 $b$ 匹配的位置并黏在一起,于是 $Sxyz$ 解读为通过 $z$ 把 $x$、$y$黏在一起。
$Kxy$ 解读为遗忘掉时空里的历史$xy$,只取到最后的结果 $x$,其实就是从蕴含或者因果链条了取结果。如果 $K$ 单置,可以解读为时间。

\begin{principle}
恒者恒在
\end{principle}

$$I = \iota \iota = \iota S K = \cdots $$

$\iota \iota$ 是一个有 2 个 $\iota$ 为叶节点的二叉树。
我们要研究 2 个 叶节点的二叉树到 2 个 叶节点的二叉树的映射。
它是我们得到的第一个封闭集。

如下的两个 Catalan-Hankel 矩阵
$\begin{bmatrix}1 & 0 \\ 0 & 1\end{bmatrix}$
$\begin{bmatrix}1 & 1 \\ 1 & 2\end{bmatrix}$

我们可以验证

$\begin{bmatrix}1 & 1 \\ 1 & 2\end{bmatrix}\begin{bmatrix}1 & 1 \\ 1 & 2\end{bmatrix} = \begin{bmatrix}2 & 3 \\ 3 & 5\end{bmatrix}$

\Tree [. $\iota$ $\iota$ ]

$\iota S K$ 是一个有 10 个 $\iota$ 为叶节点的二叉树。

\Tree [. $\iota$ $\iota$ ]

\newpage

\section{一种新表示法}


\section{从如一到数}


\section{从如一到逻辑}



\newpage

\appendix
\newpage

\section{数量关系表}

我们有如下数量关系表

\begin{table}[]
\begin{tabular}{|c|c|c|c|c|c|c|}
\hline
 n & elm &  cat     & leaf & perm & morph & struct \\ \hline
   1 &  0  &    $1$     &  1   &  1 & 1 & 不封闭   \\ \hline
   2 &  1  &    $1$     &  2   &  2 & 2 & 群 \\ \hline
   2 &  2  &    $2$     &  3   &  6 & $12 = 2^2 \cdot 3$ & 不封闭 \\ \hline
   3 &  3  &    $5$     &  4   &  24 & $120 = 2^3 \cdot 3 \cdot 5$ & 不封闭 \\ \hline
   3 &  4  &   $14 = 2 \cdot 7$ &  5   &  120 & $1680 = 2^4 \cdot 3 \cdot 5 \cdot 7$ & 不封闭  \\ \hline
   4 &  5  &   $42 = 2 \cdot 3 \cdot 7$ &  6   &  720 & $30240 = 2^5 \cdot 3^3 \cdot 5 \cdot 7 $ & 不封闭 \\ \hline
   4 &  6  &  $132 = 2^2 \cdot 3 \cdot 11$ &  7   &  5040 & $665280 = 2^6 \cdot 3^3 \cdot 5 \cdot 7 \cdot 11 $ & 。   \\ \hline
   5 &  7  &  $429 = 3 \cdot 11 \cdot 13$ &  8   &  40320 & & 。   \\ \hline
   5 &  8  & $1430 = 2 \cdot 5 \cdot 11 \cdot 13$ &  9   &  362880 & & 。   \\ \hline
   6 &  9  & $4862 = 2 \cdot 11 \cdot 13 \cdot 17$ &  10  & 3628800 & & 。   \\ \hline
   6 & 10  &$16796 = 2^2 \cdot 13 \cdot 17 \cdot 19$ &  11  & 39916800 & & 。   \\ \hline
   7 & 11  &$58786 = 2 \cdot 7 \cdot 13 \cdot 17 \cdot 19$ &  12  & 479001600 & & 。   \\ \hline
\end{tabular}
\end{table}

\section{$K_1$}

\textbf{树表示}

\Tree []

\textbf{$\iota$ 核}

单独的 $\iota$,不封闭

\textbf{路径表示}

$$p_0 = \{ \iota \} $$

\section{$K_2$}

\textbf{树表示}

\Tree [. $a$ $b$ ]

\textbf{$\iota$ 核}

\Tree [. $\iota$ $\iota$ ]

也即有

$$\iota \iota = I$$

\textbf{结构}

成群

\textbf{路径表示}

$$p_1 = I$$

\section{$K_3$}

\textbf{树表示}

\Tree [. [. $a$ $b$ ] $c$ ]

\Tree [. $a$ [. $b$ $c$ ] ]

\textbf{$\iota$ 核}

\Tree [. [. $\iota$ $\iota$ ] $\iota$ ]

$$\iota \iota \iota = \iota$$

\Tree [. $\iota$ [. $\iota$ $\iota$ ] ]

$$\iota (\iota \iota) = \iota I = SK$$

\textbf{路径表示}

$$p_2 = p_0 p_1 + p_1 p_0$$

$$p_2 = p_0 p_1 + p_0$$

$$p_0 p_1 = \{ SK \} $$

\section{$K_4$}

\textbf{树表示}

\Tree [. [. [. $a$ $b$ ] $c$ ] $d$ ]

\Tree [. [. $a$ $b$ ] [. $c$ $d$ ] ]

\Tree [. [. $a$ [. $b$ $c$ ] ] $d$ ]

\Tree [. $a$ [. [. $b$ $c$ ] $d$ ] ]

\Tree [. $a$ [. $b$ [. $c$ $d$ ] ] ]

\textbf{$\iota$ 核}

\Tree [. [. [. $\iota$ $\iota$ ] $\iota$ ] $\iota$ ]

$$\iota \iota \iota \iota = I \iota \iota = \iota \iota = I$$

\Tree [. [. $\iota$ $\iota$ ] [. $\iota$ $\iota$ ] ]

$$(\iota \iota) (\iota \iota) = II = I$$

\Tree [. [. $\iota$ [. $\iota$ $\iota$ ] ] $\iota$ ]

$$(\iota (\iota \iota)) \iota = (\iota I) \iota = S K \iota$$

\Tree [. $\iota$ [. [. $\iota$ $\iota$ ] $\iota$ ] ]

$$(\iota (\iota \iota) \iota) = (\iota (I \iota)) = \iota \iota = I$$

\Tree [. $\iota$ [. $\iota$ [. $\iota$ $\iota$ ] ] ]

$$(\iota (\iota (\iota \iota))) = (\iota (\iota I)) = (\iota I) S K = (I S K) S K = (S K) S K = S K S K = K$$

\textbf{路径表示}

$$p_3 = p_0 p_2 + p_1 p_1 + p_2 p_0$$

$$p_3 = p_0 (p_0 p_1 + p_1 p0) + p_1 p_1 + (p_0 p_1 + p_1 p0) p_0$$

$$p_3 = p_0 p_0 p_1 + p_1 + p_0 p_1 p_0$$

$$p_0 p_0 p_1 = \{ K \} $$

\section{$K_5$}

\textbf{树表示}

\Tree [. [. [. [. $a$ $b$ ] $c$ ] $d$ ] $e$ ]

\Tree [. [. [. $a$ [. $b$ $c$ ] ] $d$ ] $e$ ]

\Tree [. [. [. $a$ $b$ ] [. $c$ $d$ ] ] $e$ ]

\Tree [. [. [. $a$ $b$ ] $c$ ] [. $d$ $e$ ] ]

\Tree [. [. $a$ [. $b$ $c$ ] ] [. $d$ $e$ ] ]

\Tree [. [. $a$ [. $b$ [. $c$ $d$ ] ] ] $e$ ]

\Tree [. [. $a$ $b$ ] [. [. $c$ $d$ ] $e$ ] ]

\Tree [. [. $a$ $b$ ] [. $c$ [. $d$ $e$ ] ] ]

\Tree [. $a$ [. [. [. $b$ $c$ ] $d$ ] $e$ ] ]

\Tree [. $a$ [. [. $b$ [. $c$ $d$ ] ] $e$ ] ]

\Tree [. $a$ [. [. $b$ $c$ ] [. $d$ $e$ ] ] ]

\Tree [. $a$ [. $b$ [. [. $c$ $d$ ] $e$ ] ] ]

\Tree [. $a$ [. $b$ [. $c$ [. $d$ $e$ ] ] ] ]

\textbf{$\iota$ 核}

\Tree [. [. [. [. $\iota$ $\iota$ ] $\iota$ ] $\iota$ ] $\iota$ ]

$$\iota \iota \iota \iota \iota = I \iota \iota \iota = \iota $$

\Tree [. [. [. $\iota$ [. $\iota$ $\iota$ ] ] $\iota$ ] $\iota$ ]

$$(\iota (\iota \iota)) \iota \iota = \iota I \iota \iota = S K \iota \iota = K \iota I = \iota$$

\Tree [. [. [. $\iota$ $\iota$ ] [. $\iota$ $\iota$ ] ] $\iota$ ]

$$ ((\iota \iota) (\iota \iota)) \iota = \iota$$

\Tree [. [. [. $\iota$ $\iota$ ] $\iota$ ] [. $\iota$ $\iota$ ] ]

$$ ((\iota \iota) \iota)) (\iota \iota) = \iota I = S K$$

\Tree [. [. $\iota$ [. $\iota$ $\iota$ ] ] [. $\iota$ $\iota$ ] ]

$$ (\iota (\iota \iota)) (\iota \iota) = (\iota I) I = S K I$$

\Tree [. [. $\iota$ [. $\iota$ [. $\iota$ $\iota$ ] ] ] $\iota$ ]

$$ (\iota (\iota (\iota \iota))) \iota  = (\iota (\iota I)) \iota = (\iota I) S K \iota = S K S K \iota = K \iota$$

\Tree [. [. $\iota$ $\iota$ ] [. [. $\iota$ $\iota$ ] $\iota$ ] ]

$$ (\iota \iota) ((\iota \iota) \iota) = \iota $$

\Tree [. [. $\iota$ $\iota$ ] [. $\iota$ [. $\iota$ $\iota$ ] ] ]

$$ (\iota \iota) (\iota (\iota \iota)) = S K $$

\Tree [. $\iota$ [. [. [. $\iota$ $\iota$ ] $\iota$ ] $\iota$ ] ]

$$ (\iota \iota) (\iota (\iota \iota)) = S K $$

\Tree [. $\iota$ [. [. $\iota$ [. $\iota$ $\iota$ ] ] $\iota$ ] ]

$$ \iota (\iota (\iota \iota) \iota) = (\iota I \iota) S K = (S K \iota) S K $$

\Tree [. $\iota$ [. [. $\iota$ $\iota$ ] [. $\iota$ $\iota$ ] ] ]

$$ \iota ((\iota \iota) (\iota \iota)) = \iota $$

\Tree [. $\iota$ [. $\iota$ [. [. $\iota$ $\iota$ ] $\iota$ ] ] ]

$$ \iota (\iota ((\iota \iota) \iota)) = \iota I = S K $$

\Tree [. $\iota$ [. $\iota$ [. $\iota$ [. $\iota$ $\iota$ ] ] ] ]

$$ \iota (\iota (\iota (\iota \iota))) = \iota (\iota (\iota I)) = (S K S K) S K = K S K = S $$

\textbf{路径表示}

$$p_4 = p_0 p_3 + p_1 p_2 + p_2 p_1 + p_3 p_0$$

$$p_4 = p_0 (p_0 p_0 p_1 + p_1 + p_0 p_1 p_0) + p_1 (p_0 p_1 + p_0) + (p_0 p_1 + p_0) p_1 + (p_0 p_0 p_1 + p_1 + p_0 p_1 p_0) p_0$$

$$p_4 = p_0 (p_0 p_0 p_1 + p_1 + p_0 p_1 p_0) + p_1 (p_0 p_1 + p_0) + (p_0 p_1 + p_0) p_1 + (p_0 p_0 p_1 + p_1 + p_0 p_1 p_0) p_0$$

$$p_4 = p_0 (p_0 p_0 p_1 + p_1 + p_0 p_1 p_0) + p_1 (p_0 p_1 + p_0) + (p_0 p_1 + p_0) p_1 + (p_0 p_0 p_1 + p_1 + p_0 p_1 p_0) p_0$$

$$p_4 = p_0 p_0 p_0 p_1 + p_0 p_1 + p_0 p_0 p_1 p_0 + p_0 p_1 + p_0 + p_0 p_1 p_1 + p_0 p_1 + p_0 p_0 p_1 p_0 + p_1 p_0 + p_0 p_1 p_0 p_0$$

$$p_4 = p_0 + p_0 p_1 + p_0 p_0 p_0 p_1 + p_0 p_0 p_1 p_0 + p_0 p_1 p_1 + p_0 p_1 p_0 p_0$$


$$p_0 p_0 p_0 p_1 = \{ S \} $$

$$p_0 p_0 p_1 p_0 = \{ K \iota \} $$

$$p_0 p_1 p_0 p_0 = \{ S K \iota \iota \} $$

$$p_0 p_1 p_1 = \{ S K I \} $$

\section{$K_6$}

\textbf{路径表示}

$$p_2 = p_0 + p_0 p_1$$

$$p_3 = p_1 + p_0 p_0 p_1 + p_0 p_1 p_0$$

$$p_4 = p_0 + p_0 p_1 + p_0 p_0 p_0 p_1 + p_0 p_0 p_1 p_0 + p_0 p_1 p_1 + p_0 p_1 p_0 p_0$$

$$p_5 = p_0 p_4 + p_1 p_3 + p_2 p_2 + p_3 p_1 + p_4 p_0$$

$$p_5 = p_0(p_0 + p_0 p_1 + p_0 p_0 p_0 p_1 + p_0 p_0 p_1 p_0 + p_0 p_1 p_1 + p_0 p_1 p_0 p_0) + (p_0 + p_0 p_1 + p_0 p_0 p_0 p_1 + p_0 p_0 p_1 p_0 + p_0 p_1 p_1 + p_0 p_1 p_0 p_0)p_0 + p_1 p_3 + p_3 p_1 + p_2 p_2$$

$$p_5 = p_0 + p_1 + p_0 p_0 p_1 + p_0 p_1 p_0 + p_0 p_0 p_1 p_1 + p_0 p_1 p_0 p_1 + p_0 p_1 p_1 p_0 + p_0 p_0 p_0 p_0 p_1 + p_0 p_0 p_0 p_1 p_0 + p_0 p_1 p_0 p_0 p_0$$

$$p_5 = p_0 + p_1 + p_0^2 p_1 + p_0 p_1 p_0 + p_0^2 p_1^2 + p_0 p_1 p_0 p_1 + p_0 p_1^2 p_0 + p_0^4 p_1 + p_0^3 p_1 p_0 + p_0 p_1 p_0^3$$

\textbf{验证是否成群}

$p_0 p_1$ 这个组合不在里面,所以不封闭

\section{$K_7$}

$$p_0 = {\iota} $$

$$p_1 = p_0 p_0 = {I}$$

$$p_2 = p_0 + p_0 p_1$$

$$p_3 = p_1 + p_0^2 p_1 + p_0 p_1 p_0$$

$$p_4 = p_0 + p_0 p_1 + p_0^3 p_1 + p_0^2 p_1 p_0 + p_0 p_1^2 + p_0 p_1 p_0^2$$

$$p_5 = p_0 + p_1 + p_0^2 p_1 + p_0 p_1 p_0 + p_0^2 p_1^2 + (p_0 p_1)^2 + p_0 p_1^2 p_0 + p_0^4 p_1 + p_0^3 p_1 p_0 + p_0 p_1 p_0^3$$

$$p_6 = p_0 p_5 + p_1 p_4 + p_2 p_3 + p_3 p_2 + p_4 p_1 + p_5 p_0$$

$$p_2 p_3 = (p_0 + p_0 p_1) (p_1 + p_0^2 p_1 + p_0 p_1 p_0)$$

$$p_2 p_3 = p_0 (p_1 + p_0^2 p_1 + p_0 p_1 p_0) + p_0 p_1 (p_1 + p_0^2 p_1 + p_0 p_1 p_0)$$

$$p_2 p_3 =  p_0 p_1 + p_0^3 p_1 + p_0^2 p_1 p_0 + p_0 p_1^2 + p_0 p_1 p_0^2 p_1 + (p_0 p_1)^2 p_0$$

$$p_2 p_3 =  p_0 p_1 + p_0 p_1^2 + p_0^3 p_1 + p_0^2 p_1 p_0 + p_0 p_1 p_0^2 p_1 + (p_0 p_1)^2 p_0$$

$$p_3 p_2 = (p_1 + p_0^2 p_1 + p_0 p_1 p_0) (p_0 + p_0 p_1)$$

$$p_3 p_2 = p_1 p_0 + p_0^2 p_1 p_0 + p_0 p_1 p_0^2 + p_1 p_0 p_1 + p_0^2 p_1 p_0 p_1 + p_0 p_1 p_0^2 p_1 $$

$$p_4 p_1 = (p_0 + p_0 p_1 + p_0^3 p_1 + p_0^2 p_1 p_0 + p_0 p_1^2 + p_0 p_1 p_0^2) (p_0 p_0)$$

$$p_4 p_1 = p_0 + p_0 p_1 p_0^2 + p_0^3 p_1 p_0^2 + p_0^2 p_1 p_0^3 + p_0 p_1 p_0^4 + p_0 p_1^2 p_0^2$$

$$p_0 p_4 = p_1 + p_0^2 p_1 + p_0^4 p_1 + p_0^3 p_1 p_0 + p_0^2 p_1^2 + p_0^2 p_1 p_0^2 $$

$$p_4 p_0 = p_1 + p_0 p_1 p_0 + p_0^3 p_1 p_0 + p_0^2 p_1 p_0^2 + p_0 p_1^2 p_0 + p_0 p_1 p_0^3$$

\textbf{计算汇总}

$$p_2 p_3 = p_0 p_1 + p_0 p_1^2 + p_0^3 p_1 + p_0^2 p_1 p_0 + p_0 p_1 p_0^2 p_1 + (p_0 p_1)^2 p_0$$
$$p_3 p_2 = p_0 + p_0^2 p_1 p_0 + p_0 p_1 p_0^2 + p_0 p_1 + p_0 p_1 + p_0 p_1 p_0^2 p_1 $$
$$p_4 p_1 = p_0 + p_0 p_1 p_0^2 + p_0^3 p_1 p_0^2 + p_0^2 p_1 p_0^3 + p_0 p_1 p_0^4 + p_0 p_1^2 p_0^2$$
$$p_1 p_4 = p_0 + p_0 p_1 + p_0^3 p_1 + p_0^2 p_1 p_0 + p_0 p_1^2 + p_0 p_1 p_0^2$$
$$p_0 p_4 = p_1 + p_0^2 p_1 + p_0^4 p_1 + p_0^3 p_1 p_0 + p_0^2 p_1^2 + p_0^2 p_1 p_0^2 $$
$$p_4 p_0 = p_1 + p_0 p_1 p_0 + p_0^3 p_1 p_0 + p_0^2 p_1 p_0^2 + p_0 p_1^2 p_0 + p_0 p_1 p_0^3$$

$$
p_5 = \begin{Bmatrix}
    p_0 + p_1 \\
    p_0 p_1 \\
    p_0 p_1 p_0 + p_0 p_1^2 \\
    p_0^3 p_1 + p_0^2 p_1 p_0 + p_0 p_1 p_0^2\\
    p_0^4 p_1 + p_0^3 p_1 p_0 +
\end{Bmatrix}
$$

\section{路径计算-随时间遗忘的版本}

计算规则

$$p_0 p_0 = I$$
$$p_0^2 p_1 = K$$
$$p_0^3 p_1 = S$$
$$p_1 X = X$$

$$p_0 = \begin{Bmatrix} p_0 \end{Bmatrix}$$

$$p_1 = \begin{Bmatrix} p_1 \end{Bmatrix}$$

$$p_2 = \begin{Bmatrix}  p_0 \\  p_0 p_1 \end{Bmatrix}$$

$$\begin{array}{lcl}
p_3 & = & \begin{Bmatrix} p_0 \end{Bmatrix} \begin{Bmatrix}  p_0 \\  p_0 p_1 \end{Bmatrix} +
          \begin{Bmatrix} p_1 \end{Bmatrix}\begin{Bmatrix} p_1 \end{Bmatrix} +
          \begin{Bmatrix} p_0 \\  p_0 p_1 \end{Bmatrix} \begin{Bmatrix} p_0 \end{Bmatrix}\\
    & = & \begin{Bmatrix} p_1 \\ \cdot \\ p_0^2 p_1 \end{Bmatrix} +
          \begin{Bmatrix} p_1 \\ \cdot \\ \cdot \end{Bmatrix} +
          \begin{Bmatrix} p_1 \\ \cdot \\ p_0 p_1 p_0 \end{Bmatrix} \\
    & = & \begin{Bmatrix} 3 p_1 \\ \cdot \\ p_0^2 p_1 + p_0 p_1 p_0 \end{Bmatrix}
\end{array}
$$

$$\begin{array}{lcl}
p_4 & = & \begin{Bmatrix} p_0 \end{Bmatrix} \begin{Bmatrix} 3 p_1 \\ \cdot \\ p_0^2 p_1 + p_0 p_1 p_0  \end{Bmatrix} +
          \begin{Bmatrix} p_1 \end{Bmatrix} \begin{Bmatrix}  p_0 \\  p_0 p_1 \end{Bmatrix} +
          \begin{Bmatrix}  p_0 \\  p_0 p_1 \end{Bmatrix} \begin{Bmatrix} p_1 \end{Bmatrix} +
          \begin{Bmatrix} 3 p_1 \\ \cdot \\ p_0^2 p_1 + p_0 p_1 p_0 \end{Bmatrix}\begin{Bmatrix} p_0 \end{Bmatrix}\\
    & = & \begin{Bmatrix} \cdot \\ 3 p_0 p_1 \\ \cdot \\ p_0^3 p_1 + p_0^2 p_1 p_0 \end{Bmatrix} +
          \begin{Bmatrix} p_0 \\  p_0 p_1 \\ \cdot \\ \cdot \end{Bmatrix} +
          \begin{Bmatrix} \cdot \\ p_0 p_1 \\  p_0 p_1^2 \\ \cdot \end{Bmatrix} +
          \begin{Bmatrix} 3 p_0 \\ \cdot \\ \cdot \\ p_0^2 p_1 p_0 + p_0 p_1 p_0^2 \end{Bmatrix} \\
    & = & \begin{Bmatrix} 4 p_0 \\ 5 p_0 p_1 \\ p_0 p_1^2 \\ p_0^3 p_1 + 2 p_0^2 p_1 p_0 + p_0 p_1 p_0^2 \end{Bmatrix}
\end{array}
$$

$$\begin{array}{lcl}
p_5 & = & \begin{Bmatrix} p_0 \end{Bmatrix} \begin{Bmatrix} 4 p_0 \\ 5 p_0 p_1 \\ p_0 p_1^2 \\ p_0^3 p_1 + 2 p_0^2 p_1 p_0 + p_0 p_1 p_0^2 \end{Bmatrix} +
          \begin{Bmatrix} p_1 \end{Bmatrix} \begin{Bmatrix}  3 p_1 \\ \cdot \\ p_0^2 p_1 + p_0 p_1 p_0 \end{Bmatrix} +
          \begin{Bmatrix} p_0 \\  p_0 p_1 \end{Bmatrix} \begin{Bmatrix} p_0 \\  p_0 p_1 \end{Bmatrix} \\
    & + & \begin{Bmatrix} 3 p_1 \\ \cdot \\ p_0^2 p_1 + p_0 p_1 p_0 \end{Bmatrix} \begin{Bmatrix} p_1 \end{Bmatrix} +
          \begin{Bmatrix} 4 p_0 \\ 5 p_0 p_1 \\ p_0 p_1^2 \\ p_0^3 p_1 + 2 p_0^2 p_1 p_0 + p_0 p_1 p_0^2 \end{Bmatrix}\begin{Bmatrix} p_0 \end{Bmatrix}\\
    & = & \begin{Bmatrix} 4 p_1 + p_0\\ \cdot \\ 5 p_0^2 p_1 \\ p_0^2 p_1^2 \\ p_0^4 p_1 + 2 p_0^3 p_1 p_0 \end{Bmatrix} +
          \begin{Bmatrix} 3 p_1 \\ \cdot \\ p_0^2 p_1 + p_0 p_1 p_0 \\ \cdot \\ \cdot \end{Bmatrix} +
          \begin{Bmatrix} p_1 \\ \cdot \\  p_0^2 p_1  + p_0 p_1 p_0 \\  p_0 p_1 p_0 p_1 \\ \cdot \end{Bmatrix}
          \\
    & + & \begin{Bmatrix} 3 p_1 \\ \cdot \\ \cdot \\ p_0^2 p_1^2 + p_0 p_1 p_0 p_1 \\ \cdot \end{Bmatrix} +
          \begin{Bmatrix} 2 p_0 + 4 p_1 \\ \cdot \\ 5 p_0 p_1 p_0 \\ p_0 p_1^2 p_0 \\ p_0^3 p_1 p_0 + p_0 p_1 p_0^3 \end{Bmatrix}
          \\
    & = & \begin{Bmatrix} 3 p_0 + 15 p_1 \\ \cdot \\ 7 p_0 p_1 p_0 + \\ p_0 p_1^2 p_0 \\ p_0^3 p_1 p_0 + 2 p_0^2 p_1 p_0^2 + p_0 p_1 p_0^3\end{Bmatrix}
\end{array}
$$

\section{路径计算-不随时间遗忘的版本}

$$p_0 = \begin{Bmatrix} p_0 \end{Bmatrix}$$

$$p_1 = \begin{Bmatrix} \cdot \\  p_0 p_0 \end{Bmatrix}$$



\section{路径计算-复化的版本}

计算规则

$$p_0 = \frac{1 + i}{\sqrt{2}} \iota$$
$$q_0 = \frac{1 - i}{\sqrt{2}} \iota$$

$$\begin{array}{lcl}
[p_0 q_0] & = & \begin{Bmatrix} p_0 q_0 \end{Bmatrix} \\
          & = &
\end{array}$$

\end{document}
