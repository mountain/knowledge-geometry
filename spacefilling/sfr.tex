\documentclass{article}

\usepackage{arxiv}

\usepackage[utf8]{inputenc} % allow utf-8 input
\usepackage[T1]{fontenc}    % use 8-bit T1 fonts
\usepackage{hyperref}       % hyperlinks
\usepackage{url}            % simple URL typesetting
\usepackage{booktabs}       % professional-quality tables
\usepackage[english]{babel}
\usepackage{amsthm}
\usepackage{amsfonts}       % blackboard math symbols
\usepackage{nicefrac}       % compact symbols for 1/2, etc.
\usepackage{microtype}      % microtypography
\usepackage{amssymb}
\usepackage{gensymb}

\usepackage{graphicx}
\usepackage{amsmath}
\usepackage{leftidx}

\newtheorem{definition}{Definition}
\newtheorem{lemma}{Lemma}
\newtheorem{proposition}{Proposition}
\newtheorem{program}{Program}
\newtheorem{convention}{Convention}
\newtheorem{theorem}{Theorem}

\usepackage{stmaryrd}
\usepackage{mathtools}

\usepackage{tikz}
\usetikzlibrary{lindenmayersystems}
\usetikzlibrary[shadings]

\usepackage{float}

% this code block defines the new and custom floatbox float environment
\newfloat{lsystem}{thp}{lob}[section]
\floatname{lsystem}{Lindenmayer System}

\title{Hilbert spiral and Riemann mapping theorem}

%\date{September 9, 1985}	% Here you can change the date presented in the paper title
%\date{} 					% Or removing it

\author{
  Mingli~Yuan \\
  AI Lab \\
  ColorfulClouds Tech.\\
  Beijing, 100083 \\
  \texttt{mingli.yuan@caiyunapp.com} \\
  %% examples of more authors
  %% \AND
  %% Coauthor \\
  %% Affiliation \\
  %% Address \\
  %% \texttt{email} \\
  %% \And
  %% Coauthor \\
  %% Affiliation \\
  %% Address \\
  %% \texttt{email} \\
  %% \And
  %% Coauthor \\
  %% Affiliation \\
  %% Address \\
  %% \texttt{email} \\
}

% Uncomment to remove the date
%\date{}

% Uncomment to override  the `A preprint' in the header
%\renewcommand{\headeright}{Technical Report}
%\renewcommand{\undertitle}{Technical Report}

\begin{document}
\maketitle

\begin{abstract}
    Hilbert curve is a space-filling curve that fills the unit square.
    Based on a construction of a Hilbert-type curve $\leftidx{_4}{H}{_4}$ of Liu,
    we give the construction of Hilbert spiral which fills the whole Euclidean plane $E_2$,
    and a conterpart construction on hyperbolic plane $H_2$.
    By using the perpendicular and parallel relationship introduced by Hilbert spiral in $H_2$,
    we can define a special conformal flow which leads us to a new proof of Riemann mapping theorem.
\end{abstract}

\keywords{Space-filling curve \and Hilbert curve \and Hilbert spiral \and Riemann mapping theorem}

\setcounter{tocdepth}{2}
\tableofcontents

\section{Background}\label{sec:background}

Hilbert curves is a family of space-filling curves that fills the unit square. The construction of Hilbert curves can be
expressed as a Lindenmayer system.

For example, the Hilbert curve discovered by David Hilbert himself can be constructed by below rules:

\begin{lsystem}
    \caption{Hilbert curve}
    \begin{itemize}
      \item axiom: L
      \item angle: $90\degree$
      \item scale factor: 0.5
      \item rules: \begin{itemize}
          \item[$\circ$] $L \mapsto +RF-LFL-FR+ $
          \item[$\circ$] $R \mapsto -LF+RFR+FL- $
      \end{itemize}
    \end{itemize}
\end{lsystem}

With visual interpretions of the pure symbol
\begin{itemize}
  \item F: draw a line with one fix-length step
  \item +: turn right to a specific angle predefined
  \item -: turn left to a specific angle predefined
\end{itemize}

The first several steps of the construction of Hilbert curve can be visualized as below

\pgfdeclarelindenmayersystem{Hilbert curve}{
  \rule{L -> +RF-LFL-FR+}
  \rule{R -> -LF+RFR+FL-}}

\begin{figure}[ht]
\centering
\begin{tabular}{ccccc}
\begin{tabular}{c}
    \begin{tikzpicture}
        \shadedraw [bottom color=white, top color=white, draw=red!80!black]
        [l-system={Hilbert curve, axiom=L, order=1, step=32pt, angle=90}]
        lindenmayer system;
    \end{tikzpicture}
\end{tabular}
&
\begin{tabular}{c}
    \begin{tikzpicture}
        \shadedraw [bottom color=white, top color=white, draw=red!80!black]
        [l-system={Hilbert curve, axiom=L, order=2, step=16pt, angle=90}]
        lindenmayer system;
    \end{tikzpicture}
\end{tabular}
&
\begin{tabular}{c}
    \begin{tikzpicture}
        \shadedraw [bottom color=white, top color=white, draw=red!80!black]
        [l-system={Hilbert curve, axiom=L, order=3, step=8pt, angle=90}]
        lindenmayer system;
    \end{tikzpicture}
\end{tabular}
&
\begin{tabular}{c}
    \begin{tikzpicture}
        \shadedraw [bottom color=white, top color=white, draw=red!80!black]
        [l-system={Hilbert curve, axiom=L, order=4, step=4pt, angle=90}]
        lindenmayer system;
    \end{tikzpicture}
\end{tabular}
&
\begin{tabular}{c}
    \begin{tikzpicture}
        \shadedraw [bottom color=white, top color=white, draw=red!80!black]
        [l-system={Hilbert curve, axiom=L, order=5, step=2pt, angle=90}]
        lindenmayer system;
    \end{tikzpicture}
\end{tabular}
\\
\end{tabular}
\caption{the construction of Hilbert curve}
\end{figure}

With different initial axioms and rules, a family of Hilbert-type curves had been discovered by Hilbert, Moore,
Liu and Perez-Davidenko et al. For example, the construction of Hilbert-type curve $\leftidx{_4}{H}{_4}$ of Liu
can be constructed by

\begin{itemize}
  \item axiom: L
  \item scale factor: 0.5
  \item rules: \begin{itemize}
      \item[$\circ$] $L \mapsto +RF-LFL-FR+ $
      \item[$\circ$] $R \mapsto -LF+RFR+FL- $
  \end{itemize}
\end{itemize}

\pgfdeclarelindenmayersystem{Liu curve}{
  \rule{C -> +RF-LFL-FR+}
  \rule{L -> +RF-LFL-FR+}
  \rule{R -> -LF+RFR+FL-}}

\begin{figure}[ht]
\centering
\begin{tabular}{ccc}
\begin{tabular}{c}
    \begin{tikzpicture}
        \shadedraw [bottom color=white, top color=white, draw=red!80!black]
        [l-system={Liu curve, axiom=C, order=1, step=32pt, angle=90}]
        lindenmayer system;
    \end{tikzpicture}
\end{tabular}
&
\begin{tabular}{c}
    \begin{tikzpicture}
        \shadedraw [bottom color=white, top color=white, draw=red!80!black]
        [l-system={Liu curve, axiom=C, order=2, step=16pt, angle=90}]
        lindenmayer system;
    \end{tikzpicture}
\end{tabular}
&
\begin{tabular}{c}
    \begin{tikzpicture}
        \shadedraw [bottom color=white, top color=white, draw=red!80!black]
        [l-system={Liu curve, axiom=C, order=3, step=8pt, angle=90}]
        lindenmayer system;
    \end{tikzpicture}
\end{tabular}
\\
\end{tabular}
\caption{the construction ${}_4H_4$ of Liu}
\end{figure}

\section{Hilbert spirals}\label{sec:space-filling}

\section{A flow on $H_2$}\label{sec:flow}

\section{A new proof of Riemann mapping theorem}\label{sec:riemann-mapping}

\section{Conclusion}\label{sec:conclusion}

\end{document}
