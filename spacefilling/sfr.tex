\documentclass{article}

\usepackage{arxiv}

\usepackage[utf8]{inputenc} % allow utf-8 input
\usepackage[T1]{fontenc}    % use 8-bit T1 fonts
\usepackage{hyperref}       % hyperlinks
\usepackage{url}            % simple URL typesetting
\usepackage{booktabs}       % professional-quality tables
\usepackage[english]{babel}
\usepackage{amsthm}
\usepackage{amsfonts}       % blackboard math symbols
\usepackage{nicefrac}       % compact symbols for 1/2, etc.
\usepackage{microtype}      % microtypography

\usepackage{graphicx}
\usepackage{amsmath}
\usepackage{leftidx}

\newtheorem{definition}{Definition}
\newtheorem{lemma}{Lemma}
\newtheorem{proposition}{Proposition}
\newtheorem{program}{Program}
\newtheorem{convention}{Convention}
\newtheorem{theorem}{Theorem}

\usepackage{stmaryrd}
\usepackage{mathtools}

\title{Hilbert curve and Riemann mapping theorem}

%\date{September 9, 1985}	% Here you can change the date presented in the paper title
%\date{} 					% Or removing it

\author{
  Mingli~Yuan \\
  AI Lab \\
  ColorfulClouds Tech.\\
  Beijing, 100083 \\
  \texttt{mingli.yuan@caiyunapp.com} \\
  %% examples of more authors
  %% \AND
  %% Coauthor \\
  %% Affiliation \\
  %% Address \\
  %% \texttt{email} \\
  %% \And
  %% Coauthor \\
  %% Affiliation \\
  %% Address \\
  %% \texttt{email} \\
  %% \And
  %% Coauthor \\
  %% Affiliation \\
  %% Address \\
  %% \texttt{email} \\
}

% Uncomment to remove the date
%\date{}

% Uncomment to override  the `A preprint' in the header
%\renewcommand{\headeright}{Technical Report}
%\renewcommand{\undertitle}{Technical Report}

\begin{document}
\maketitle

\begin{abstract}
    Hilbert curve is a space-filling curve that fills the unit square. Based on a construction of Hilbert curve $\leftidx{_4}{H}{_4}$ of Liu,
    we give another upward procedure to combine with the standard downward procedure to fill the whole Euclidean plane $E_2$.
    By giving this space-filling curve, we can conformally transform it into hyperbolic plane $H_2$ in which the curve still fill
    the whole hyperbolic plane $H_2$. Using the perpendicular and parallel relationship of the space-filling curve in $H_2$,
    we can define a special parallel movement and this leads us to a new proof of Riemann mapping theorem.
\end{abstract}

\keywords{space-filling curve \and Hilbert curve \and Riemann mapping theorem}

\setcounter{tocdepth}{2}
\tableofcontents

\section{Space-filling curves fill $E_2$ and $H_2$}\label{sec:space-filling}

\section{A special parallel movement on $H_2$}\label{sec:parallel-movement}

\section{A new proof of Riemann mapping theorem}\label{sec:riemann-mapping}

\end{document}
